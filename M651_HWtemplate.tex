\documentclass{report}
\usepackage{amsthm, amsmath, amssymb,verbatim,xspace, graphicx}
\usepackage{pdfsync, listings, color,pgf,tikz}
\usepackage[all,cmtip]{xy}

\setlength{\oddsidemargin}{0in} \setlength{\textwidth}{6in}
\setlength{\topmargin}{-0.25in} \setlength{\textheight}{8.5in}

\newcommand{\assignmentNumber}[1]{Homework #1 Selected Solutions}
\newcommand{\duedate}[1]{#1}

\newcommand\R{{\mathbb R}}
\newcommand\Q{{\mathbb Q}}
\newcommand\N{{\mathbb N}}
\newcommand\Z{{\mathbb Z}}
\newcommand\Zn{\mathbb Z / n \mathbb Z}
\newcommand\eP{\varphi}
\newcommand\dsp{\displaystyle}


\theoremstyle{plain}
\newtheorem{thm}{Theorem}
\newtheorem{lem}[thm]{Lemma}
\newtheorem{prop}[thm]{Proposition}
\newtheorem{cor}[thm]{Corollary}
\newtheorem{conj}{Conjecture}
\newtheorem{quest}{Question}
\newtheorem*{rem}{Remark}

\date{August 24, 2018}

\begin{document} 

% Change to student's name
\rightline{N. Bourbaki}
% Change due date below
\rightline{\duedate{January 23, 2019}}

% Change assignment number below
\centerline{\sc \large Math 651: Homework 1 }
\smallskip


\begin{itemize}

% If a problem asks you to "show" or "prove" something, state a formal Proposition/Lemma/Theorem, etc. and give a proof. Do NOT state the problem as it was in the text.
\item[p. 83, \# 1]  
\begin{thm}Let $X$ be a topological space $X$, and $A\subseteq X$. Suppose for each $x\in A$ there is an open set $U$ with $x\in U\subseteq A$.
Then $A$ is open.
\end{thm}
\begin{proof} To see $A$ is open, observe that \dots
\end{proof}

% Not all problems require stating a Proposition. If that is not appropriate, state the problem (paraphrasing is ok) and simply do it.
\item[p. 83, \# 2]  Consider  the nine topologies of Example 1 of section 12, and name them $T_1$ through $T_{12}$ so the first row has $T_1,T_2, T_3$, etc.
Then for each of the 36 pairs, determine if they are comparable, and which is finer.

Answer: $T_1$ is comparable to all the others, which are all finer than it. $T_2$ \dots


% If a problem asks a question, always justify your answer. This is usually best done by stating and proving a proposition
\item[p. 83, \# 3] 
\begin{prop} The collection $\mathcal T_c$ in Example 4 of section 12 is a topology on $X$.
\end{prop}
\begin{proof}\dots
\end{proof}
\begin{prop} The collection $\mathcal T_\infty=\{U\mid X\smallsetminus U$ is infinite or empty or $X$\} is a topology on $X$.
\end{prop}
\begin{proof}\dots
\end{proof}

\end{itemize}


\end{document}