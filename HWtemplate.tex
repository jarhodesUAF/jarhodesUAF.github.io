\documentclass[12pt]{report}

\usepackage{amsmath,amssymb,amsthm}
\usepackage[margin=1.25in,footskip=0.25in]{geometry} % Please do not change margins


\theoremstyle{plain}
\newtheorem*{thm*}{Theorem}
\newtheorem*{lemma*}{Lemma}
\newtheorem*{prop*}{Proposition}
\newtheorem*{cor*}{Corollary}
\newtheorem*{conj*}{Conjecture}


\newcommand{\rank}{\operatorname{rank}} % feel free to define your own commands


\begin{document}

\hfill Nicolas Bourbaki % Put your name here (and Google "Bourbaki")

\hfill \today 

\centerline{\large{Math 405 Homework}}

\begin{enumerate}

\item[p. 53, \#2]
\begin{thm*} If $G$ is a group, its identity is unique.
\end{thm*}
\begin{proof}
Suppose $e_1$ and $e_2$ are both identities in $G$. Then
$$e_1=e_1e_2=e_2$$
where the first equality follows from $e_2$ being an identity, and the second from $e_1$ being an identity.
\end{proof}

 \item[p. 54 \#5] % Present problems in order, using numbering from the book
Let 
$$A=\begin{pmatrix}1&2&3&4\\5&6&7&8\end{pmatrix}.$$
Note $\rank{A}\le \min(2,4)=2,$ since $A$ is $2\times 4$.
Since the rows of $A$ are non-zero, and not multiples of one another, $\rank(A)\ne 0,1$, so $\rank(A)=2$.

\item[p. 63, \#1] % Include problems you failed to do (with apologetic groveling as you see fit)
I don't know how to do this, and foolishly did not realize that until it was too late to seek help. It will never happen again.


\end{enumerate}


\end{document}

\item[\#1] %Non-text problems, stated in class, will be numbered #1,#2, etc.
Suppose $A$ has singular value decomposition $A=U\Sigma V^T$, with $U,V$ orthogonal, $\Sigma$ a (possibly non-square) diagonal matrix.
Then
\begin{align*}
AA^T&=(U\Sigma V^T)(U\Sigma V^T)^T\\
&= (U\Sigma V^T)(V\Sigma^T U^T)\\
&= U\Sigma (V^TV) \Sigma U^T
\end{align*}
But by orthogonality, $V^TV=I$. Letting $\Lambda=\Sigma\Sigma^T$, which is both square and diagonal, yields 
$$AA^T=U\Lambda U^T.$$
Multiplying on the right by $U$, and using its orthogonality, gives
 $$(AA^T) U= U\Lambda.$$
 This shows the columns of $U$ are right eigenvectors of $A A^T$, with eigenvalues given by the corresponding diagonal element of $\Lambda$. From
 $\Lambda=\Sigma\Sigma^T$ we see this is just the square of the corresponding singular value of $A$.



