\documentclass{report}

\usepackage{amsmath, amssymb, amsfonts}
\usepackage{pdfsync}
\usepackage{graphicx}
\usepackage{gensymb, xspace}

\usepackage[usenames]{color}
\definecolor{myblue}{rgb}{.2, .2, .7}

\newcommand\RR{{\mathbb R}}
\newcommand\QQ{{\mathbb Q}}
\newcommand\NN{{\mathbb N}}
\newcommand\dsp{\displaystyle}
\newcommand\seq{{\left \{ x_n \right \}}_{n = 1}^\infty}
\newcommand\cotwo{$\operatorname{CO}_2$\xspace}

\newcommand{\vect}[1]{\overrightarrow{#1}}

\setlength{\oddsidemargin}{-.25in}
\setlength{\evensidemargin}{-.0in}
\setlength{\textwidth}{7in}
\setlength{\textheight}{9.2in}

\setlength{\topmargin}{0in}
\setlength{\headheight}{0in}
\setlength{\headsep}{0in}

\begin{document}

%\
%
%\vskip -.6in

\noindent MATH 202 \hfill Name :\underbar{\ \ \ \ \ \ \ \ \ \ \ \ \
\ \ \ \ \ \ \ \ \ \ \ \ \ \ \ \ \ \ \ \ \ \ \ \ \ \ \ \ \ \ \ \ \ \
\ }

\noindent Midterm 2 \hfill March 23, 2012

\bigskip

\noindent {\bf Instructions:}  Show all work for full credit.  Poor notation or sloppy work will
be penalized.

\begin{enumerate}

\item (15 pts.) 

\begin{enumerate}

\item (7 pts.) Prove that $\dsp \lim_{(x,y) \to (0,0)} \frac{3xy}{x^2+y^2}$ does not exist.

\vskip 7cm

\item (8 pts.) A particle moves along a curve in $3$-space given by 
$\mathbf r (t) = \langle 4 \sin(3t), 3t, 4 \cos(3t) \rangle$ meters, 
where $t$ is measured in seconds.
Give and evaluate a definite integral that computes the distance traveled by the particle between 
time $t = 0$ seconds and $t = 2\pi$ seconds. 

\end{enumerate}


\newpage


\item (8 pts.) Suppose that $w = xy^2 - x^2$ and that at time $t=0$, 
$(x,y) = (1,2), \; \frac{dx}{dt} \big\vert_{t=0} = -5,$ and $\frac{dy}{dt} = 3$.
Use the Multivariable Chain Rule to compute $\dsp \frac{dw}{dt}$ when $t = 0$.

\vskip 6cm

\item (12 pts. -- 6 pts. each)  It is well-known that mosquitoes are drawn to \cotwo.  
Below is a contour plot for the concentration $C(x,y)$ of \cotwo 
at positions $(x,y)$ in feet on the grid shown.  

\includegraphics[width=7cm]{fg-contour-mosquitoes-2.eps} \hskip 1.5cm \includegraphics[width=7cm]{fg-contour-mosquitoes-3.eps}

\centerline{(a) \hskip 7.8cm (b)}

\begin{enumerate}

\item In Figure (a) on the left, sketch the path
of a mosquito placed on the $X$ at $(-4,4)$ as it continuously moves
to maximize its exposure to \cotwo.  Briefly justify the reasoning for
drawing the path you did.

\vskip 2.5cm

\item In Figure (b) on the right, give the value of the directional derivative
of $C(x,y)$ in the direction indicated by the vector $\mathbf u$ at the point $P$
indicated with a dot.  Explain your answer briefly.

\vskip 2.5cm


\end{enumerate}

\newpage

\item (10 pts.)
Find the equation of the tangent plane to the ellipsoid $x^2 + 2y^2 + 9z^2 = 31$ at the point
$(2,3,1)$.

\vskip 7cm


\item (10 pts.) Electrical power $P$ in watts is given by
$$
P = \frac{V^2}{R},
$$
where $V$ is voltage and $R$ is resistance in ohms.

\begin{enumerate}

\item Give a formula for the total differential $dP$ for power.

\vskip 3.5cm


\item If $V = 120$ volts is applied to a $2000$-ohm resistor, compute
the total differential $dP$ for power.  

\vfill

\hfill $dP = \underbar{\hskip 12cm}$

\end{enumerate}


\newpage

\item (17 pts.) Suppose the elevation above sea level in tens of meters 
is given by the function $$h(x,y) = \frac{y^2}{5-x} \ \text{  tens of } m,$$ and a hiker 
is located at the position $(x,y) = (4,1)$.  

\begin{enumerate}

\item (6 pts.) In what direction from $(4,1)$ should the hiker move to increase
his/her elevation the most?

\vskip 5cm

\item (6 pts.) If the hiker moves in the direction indicated by the vector
$\mathbf{v} = \langle 1, \, -1 \rangle$, what is the rate of change of the hiker's elevation?

\vskip 5cm

\item (5 pts.) Using your answer to part (b), do you expect the hiker's elevation to rise or fall
as the hiker moves in the direction given by $\mathbf{v}$?

\vskip 2cm

\end{enumerate}

\newpage

\item (16 pts.) Consider the function $f(x,y) = -x^3 + 6xy - 3y^2 +1$.

\begin{enumerate}

\item (8 pts.) Find all critical points of $f(x,y)$.

\vskip 9cm

\item (8 pts.) Use the second derivative test to determine if the critical points are local maxima, local minima, saddle points or if there is not enough information to tell.

\end{enumerate}

\newpage

\item (12 pts.)  {\bf Use Lagrange multipliers} to find the maximum value of $\dsp f(x,y) = xy$ where $x>0$
and $y>0$, subject to the constraint $\dsp \frac{x^2}{8} + \frac{y^2}{2} = 1$.


\vfill

\hskip 1cm  The maximum value is ~\underbar{\hskip 2cm},~ and occurs at $(x,y) =$ ~\underbar{\hskip 5cm}.

\end{enumerate}

\end{document}
