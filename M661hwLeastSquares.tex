\documentclass[12pt]{report}
%prepared in AMSLaTeX, under LaTeX2e
\usepackage[margin=1.in]{geometry}
\usepackage{amsmath, amssymb,amsthm, verbatim}

\newtheorem*{thm}{Theorem}
\newtheorem*{defn}{Definition}
\newtheorem*{example}{Example}
\newtheorem*{problem}{Problem}
\newtheorem*{remark}{Remark}

\usepackage[final]{graphicx}
\newcommand{\mfigure}[1]{\includegraphics[height=2.5in,
width=3.5in]{#1.eps}}
\newcommand{\regfigure}[2]{\includegraphics[height=#2in,
keepaspectratio=true]{#1.eps}}
\newcommand{\widefigure}[3]{\includegraphics[height=#2in,
width=#3in]{#1.eps}}



\begin{document}
\thispagestyle{empty}

\noindent Math 661   \hfill  Homework

\noindent Optimization \hfill August 26, 2020

\ 

\centerline{Linear Least Squares}

\ 

For the problems below, I recommend using MATLAB for all calculations --- you do not need to show any work except what you are explicitly asked to write down.

\begin{enumerate}
\item Consider the 5 data points $(x,y)$:
(1, 2.3 ),\ 
(2, 1.1),\
(3, 0.8),\
(4, 1.0 ),\
(5, 1.4),\

\begin{enumerate}
\item Give the system of 5 equations in 2 unknowns you'd like to solve (but which has no solution) to find a straight line $y=b_0 +b_1x$ through these points. Write this as a matrix
equation $A\mathbf x=\mathbf b$.
\item Write the normal equations for this system as a $2\times 2$ matrix equation.
\item Solve the normal equations, and give the least-squares best fit line.

\end{enumerate}

\item Using the same data, and writing down answers to the analogs of the 3 parts of problem 1, find the least-squares best fit quadratic  $y=b_0 +b_1x+ b_2x^2$.

\item Using the same data, and writing down answers to the analogs of the 3 parts of problem 1, find the least-squares best fit equation of the form  $y=b_0 +b_1 e^x+ b_2e^{-x}$

\item Graph the 3 curves you found in the problems above on the same graph, and mark the  data points on it as well. 



\end{enumerate}

\ 

\noindent
MATLAB notes: 

1. To solve  $A\mathbf x=\mathbf b$ in MATLAB, enter a matrix {\tt A} and a column vector {\tt b}, and use {\tt A\textbackslash b}. If there is no solution, this give the least-squares solution (if there is a unique one), without you needing to construct the normal equations. While you can use this to check your answer to part (c), for this assignment you must still construct the normal equations.

2. For problem 2 you might find the following commands helpful:
\begin{verbatim}
z=1:5
A=vander(z)
A=A(:,3:5)
\end{verbatim}
Check the documentation for {\tt vander} if you need to.

3. For plotting, check the documentation of the {\tt plot} function. You have to generate closely spaced points to get a nice curve. For instance, to plot $y=x^2$
for $0\le x\le 1$, try 
\begin{verbatim}
x=0:.01:1
y=x.^2
plot(x,y)
\end{verbatim}

You may also find the {\tt hold on} command useful for plotting several curves together. 



\end{document}
