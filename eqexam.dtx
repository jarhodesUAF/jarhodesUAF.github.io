% \iffalse
% makeindex -s gglo.ist -o eqexam.gls eqexam.glo
%<*copyright>
%%%%%%%%%%%%%%%%%%%%%%%%%%%%%%%%%%%%%%%%%%%%%%%%%%%%%%%%%%
%% eqexam.sty package,             2011-08-17           %%
%% Copyright (C) 2005--2011  D. P. Story                %%
%%   dpstory@uakron.edu                                 %%
%%                                                      %%
%% This program can redistributed and/or modified under %%
%% the terms of the LaTeX Project Public License        %%
%% Distributed from CTAN archives in directory          %%
%% macros/latex/base/lppl.txt; either version 1 of the  %%
%% License, or (at your option) any later version.      %%
%%%%%%%%%%%%%%%%%%%%%%%%%%%%%%%%%%%%%%%%%%%%%%%%%%%%%%%%%%
%</copyright>
%<package>\NeedsTeXFormat{LaTeX2e}
%<package>\ProvidesPackage{eqexam}
%<package> [2012/05/16 v3.1 An Exam Construction Package (dps)]
%<*driver>
\documentclass{ltxdoc}
\usepackage[colorlinks,hyperindex]{hyperref}
%\pdfstringdefDisableCommands{\let\\\textbackslash}%
%\EnableCrossrefs
%\CodelineIndex
\RecordChanges
\begin{document}
    \def\switchCats{\catcode`\{=12\relax\catcode`\}=12\relax
    \catcode`\(=1\relax\catcode`\)=2\relax}
  \GetFileInfo{eqexam.sty}
  \title{\textsf{eqexam}: An Exam Construction Package}
    \author{D. P. Story\\
      Email: \texttt{dpstory@uakron.edu}}
    \date{Processed \today}
  \maketitle
  \tableofcontents
  \let\Email\texttt
  \renewenvironment{theglossary}{%
    \let\efill\relax
    \begin{itemize}}{\end{itemize}}
   \value{GlossaryColumns}=1
  \DocInput{eqexam.dtx}
\PrintIndex
\end{document}
%</driver>
% \fi
%
% \MakeShortVerb{|}
% \StopEventually{}
%
% \DoNotIndex{\def,\edef,\gdef,\xdef,\global,\long,\let}
% \DoNotIndex{\expandafter,\string,\the,\ifx,\else,\fi}
% \DoNotIndex{\csname,\endcsname,\relax,\begingroup,\endgroup}
% \DoNotIndex{\DeclareTextCommand,\DeclareTextCompositeCommand}
% \DoNotIndex{\space,\@empty,\special}
%
% \IfFileExists{\jobname.gls}{\section{Change History}\PrintChanges}{}
%
%    \section{Introduction}
%
%    This package is my attempt at writing a set of macros for creating exams.
%    The package can be used with or without \textsf{web} or \textsf{exerquiz}.
%    When used without, what is produced is a document suitable for printing.
%    When \textsf{exerquiz} is used, the same document is produced, but with
%    hypertext links to solutions. This may be useful for publishing solutions
%    on the web, or publishing pretests with/without solutions.
%
%    The package must be as flexible as possible vis-\`a-vis PDF. (1)
%    \textsf{web} only does not add much, it does input hyperrref and test info
%    such as \cs{title}, \cs{subject}, etc are placed in the Document Info
%    fields of the PDF. (2) If \textsf{exerquiz} is also input (prior to
%    \textsf{eqexam}), then hyperlinks will be created to the solutions to the
%    test, if solutions are included at the end of the document. (3) If
%    \textsf{web} and \textsf{exerquiz} are input, and the \texttt{online}
%    option is taken, then the checkboxes will be come interactive, the space
%    left to work the problems will be multi-line text boxes, fill-ins limited
%    to True/False and simple text fill-in the blank will also become text
%    boxes.
%
%    When in \texttt{online} mode, the student can take the test in a computer
%    lab, the completed exam can be printed and handed in, or perhaps submitted
%    server-side script.
%
%    \medskip\noindent (2011/05/13) The version of \textsf{eqexam} is a
%    departure from previous versions. Previously, the list of problems were
%    not in list, they were left-justified, with the problem number extending
%    out into a little area determined by \cs{oddsidemargin}. This makes it
%    hard to reformat a list of problems to fit into a custom book format.
%    This new version defines a new environment, \texttt{eqequestions}, that
%    makes each problem into a list. The list environment allows for an easy
%    redesign of the formatting of the problems. The purpose of this new
%    scheme, is to open up \textsf{eqexam} as a format package that can be used
%    by author for writing a textbook.
%
%    The \texttt{fortextbook} option supplies support for authors writing a
%    textbook. The \texttt{exam} environment is re-cast into the
%    \texttt{probset} environment, it can be used to write problem sets within
%    the text. there is a version for the instructor and the student. The
%    instructors version writes answers to the problem sets to the margins (or
%    inline). Two solutions are offered, short and long. The short solutions
%    appear in the back of the book (odd-numbered ones for the student edition.
%    The long solutions are used to build the stand-alone solutions manuals for
%    both the student and the instructor. More details are found in
%    \Nameref{fortextbook}, see also \Nameref{fortextbookstyle}.
%
% \newpage
%    \section{Package options and  Process Options}
%    Let us catalog the options of this package.
%
%    \subsection{Options New to \textsf{eqexam}}
%
%    Here are some options unique to this package.
%    \begin{macrocode}
%<*package>
%    \end{macrocode}
%    \begin{macrocode}
\usepackage{xkeyval}
%    \end{macrocode}
%
%    \begin{macro}{usecustomdesign}
% Use this option to avoid \textsf{eqexam} from setting up the ``standard'' page layout.
%    \begin{macrocode}
\DeclareOptionX{usecustomdesign}{\eqcustomdesigntrue}
\newif\ifeqcustomdesign \eqcustomdesignfalse
\DeclareOptionX{nocustomdesign}{\let\eqe@nocustomdesign=1}
\let\eqe@nocustomdesign=0
%    \end{macrocode}
%    \end{macro}
%    \begin{macro}{fortextbook}
%    An option to extend the application of \textsf{eqexam} to provide support
%    (exercises, providing solutions, short solutions, answers, and hints) for
%    authors writing textbooks. See \Nameref{fortextbook}.
%    \begin{macrocode}
\DeclareOptionX{fortextbook}{\eqfortextbooktrue}
\newif\ifeqfortextbook \eqfortextbookfalse
%    \end{macrocode}
%    \end{macro}
%    \begin{macro}{forinstr}
%    \begin{macro}{forstudent}
%    These two options simply set a switch to signal the intention of the
%    document author.
%    \begin{macrocode}
\DeclareOptionX{forinstr}{\eqforinstrtrue}
\DeclareOptionX{forstudent}{\eqforinstrfalse}
\newif\ifeqforinstr \eqforinstrfalse
%    \end{macrocode}
%    \end{macro}
%    \end{macro}
%    \begin{macro}{nomarginwrite}
%    The switch \cs{ifeqwritetomargins} is used by the \texttt{fortextbook}
%    option. It is normally \texttt{true}, but if set to \texttt{false}, the
%    \cs{AddToShipoutPicture} is not generated at the beginning of the
%    document. Here is the code taken from below:
%\begin{verbatim}
%   \ifeqfortextbook\ifeqwritetomargins
%   \AtBeginDocument{\chkmarginboxwidth
%       \AddToShipoutPicture{\eqe@tb@shipout}}
%   \fi\fi
%\end{verbatim}
%    Using this option, the check for the margin width is not done,
%    and writing to the margins is turned off.
%    (\cs{marginpar} still works)
%    \begin{macrocode}
\DeclareOptionX{nomarginwrite}{\eqwritetomarginsfalse}
\newif\ifeqwritetomargins\eqwritetomarginstrue
%    \end{macrocode}
%    \end{macro}
%
% \paragraph*{Configuration Files.}
% This section contains options for the configuration files.
%    \begin{macro}{cfg}
% The \texttt{cfg} option is used to specify a named configuration file, extension
% must be \texttt{.cfg}; usage \texttt{cfg=hwdoc}.
%\changes{v3.0u}{2012/09/03}{Added the \texttt{cfg} option for inputting
% a custom config file.}
%    \begin{macrocode}
\define@key{eqexam.sty}{cfg}[]{%
    \def\arg@i{#1}\ifx\arg@i\@empty
    \PackageWarning{eqexam}{No value for `cfg' specified}\else
    \AtEndOfPackage{\InputIfFileExists{#1.cfg}
    {\typeout{Inputting #1.cfg}}{\PackageWarning{eqexam}{%
        Cannot find configuration file #1.cfg}}}\fi
    }
%    \end{macrocode}
%    \end{macro}
%    \begin{macro}{myconfig}
%    \begin{macro}{myconfigi}
%    \begin{macro}{myconfigii}
%    \begin{macro}{myconfigiii}
%    \begin{macro}{myconfigiv}
%    \begin{macro}{myconfigv}
%    \begin{macro}{myconfigvi}
%    We offer seven sets of configuration files, that should be enough, especially
%    light of the new \texttt{cfg} option, defined above.
%    \changes{v3.0t}{2012/25/01}{Added four more CFG files are the request
%    of a user.}
%    \begin{macrocode}
\@for\eqe@tmp@i:={},i,ii,iii,iv,v,vi\do{%
    \edef\eqe@tmp@exp{%
    \noexpand\DeclareOptionX{myconfig\eqe@tmp@i}%
        {\noexpand\AtEndOfPackage{\expandafter\noexpand
            \csname eqemyconfig\eqe@tmp@i\endcsname}}%
    }\eqe@tmp@exp
}
%    \end{macrocode}
%    \end{macro}
%    \end{macro}
%    \end{macro}
%    \end{macro}
%    \end{macro}
%    \end{macro}
%    \end{macro}
%
%    \paragraph*{Point options.}
%    Options relating to points, points on left, right, both, no points,
%    totals on left and right.
%    \begin{macro}{pointsonleft}
%    \begin{macro}{pointsonright}
%    \begin{macro}{pointsonboth}
%    \begin{macro}{nopoints}
%    \begin{macro}{totalsonleft}
%    \begin{macro}{totalsonright}
%    We offer options for points and totals.
%    \begin{macrocode}
\DeclareOptionX{pointsonleft}{\AtEndOfPackage{\PointsOnLeft}}
\DeclareOptionX{pointsonright}{\AtEndOfPackage{\PointsOnRight}}
\DeclareOptionX{pointsonboth}{\AtEndOfPackage{\PointsOnBothSides}}
\DeclareOptionX{nopoints}{\AtEndOfPackage{\NoPoints}}
\DeclareOptionX{totalsonleft}{\AtEndOfPackage{\TotalsOnLeft}}
\DeclareOptionX{totalsonright}{\AtEndOfPackage{\TotalsOnRight}}
%    \end{macrocode}
%    \end{macro}
%    \end{macro}
%    \end{macro}
%    \end{macro}
%    \end{macro}
%    \end{macro}
%
%    \paragraph*{Totals options.}
%
%    \begin{macro}{nototals}
%    \begin{macro}{noparttotals}
%    \begin{macro}{parttotalsonright}
%    \begin{macro}{parttotalsonleft}
%    \begin{macro}{noseparationrule}
%    \begin{macro}{nosummarytotals}
%    Options relating to totals
%    \begin{macrocode}
\DeclareOptionX{nototals}{\AtEndOfPackage{\NoTotals}}
\DeclareOptionX{noparttotals}{\AtEndOfPackage{\let\eq@parttotals=n}}
\DeclareOptionX{parttotalsonright}{%
    \def\eqeomarginbox{\eqeomarginboxright}}
\DeclareOptionX{parttotalsonleft}{%
    \def\eqeomarginbox{\eqeomarginboxleft}}
\def\eqeomarginbox{\eqeomarginboxright}
\DeclareOptionX{noseparationrule}{%
    \AtEndOfPackage{\let\separationrule\relax}}
\DeclareOptionX{nosummarytotals}{\let\eq@nosummarytotals=y}
%    \end{macrocode}
%    \end{macro}
%    \end{macro}
%    \end{macro}
%    \end{macro}
%    \end{macro}
%    \end{macro}
%    \paragraph*{cover page options.} There are two such options,
%    \texttt{coverpage} and \texttt{cover\-page\-sumry}.
%    \begin{macro}{coverpage}
%    If this option is taken, a cover page is generate.
%    \begin{macrocode}
\DeclareOptionX{coverpage}{\def\eqex@coverpage{\eqexcoverpage}%
    \setcounter{page}{0}}
%    \end{macrocode}
%    \begin{macro}{coverpagesumry}
%    If this option is taken, an \textbf{Exam Record} is generated on the
%    cover page, provided the \texttt{coverpage} option is taken. Possible
%    values aer \texttt{byparts}, \texttt{bypages}, or \texttt{none}.
%    \begin{macrocode}
\define@choicekey+{eqexam.sty}{coverpagesumry}[\val\nr]%
    {byparts,bypages,none}{%
    \ifcase\nr\relax
        \def\sumryAnnots{\cpSumrybyparts}\or
        \def\sumryAnnots{\cpSumrybypages}\or
        \let\sumryAnnots\relax
    \fi
}{\PackageWarning{aeb}{Bad choice for coverpagesumry, permissible values
   are byparts, bypages, and none. Try again}}
\let\sumryAnnots\relax
%    \end{macrocode}
%    \end{macro}
%    \end{macro}
%
%    \paragraph*{Options related to how the document is built.}
%
%    \begin{macro}{nospacetowork}
%    The vertical space defined by the solution environment is removed.
%    \begin{macrocode}
\DeclareOptionX{nospacetowork}{%
    \AtEndOfPackage{\let\eq@insertverticalspace=n}}
%    \end{macrocode}
%    \begin{macro}{answerkey}
% Equivalent to solutionsafter and proofing.
%    \begin{macrocode}
\newif\ifanswerkey \answerkeyfalse
\DeclareOptionX{answerkey}{\answerkeytrue\eq@proofingtrue
    \eq@solutionsaftertrue}
%    \end{macrocode}
%    \begin{macro}{vspacewithsolns}
%    When \texttt{vspacewithsolns} is used, vertical space is created by
%    the solutions environment, and the solutions are written to the
%    end of the file.
%    \changes{v2.0d}{2011/03/04}{%
%       Added the \texttt{vspacewithsolns} option}
%    \begin{macro}{ftbsolns}
%    Added \texttt{ftbsolns} as an alias for \texttt{vspacewithsolns}
%    \changes{v3.0h}{2011/08/17}{% 2011/08/17 v3.0h Added the
%    \texttt{vspacewithsolns} option}
%    \begin{macrocode}
\newif\ifvspacewithsolns\vspacewithsolnsfalse
\DeclareOptionX{vspacewithsolns}{%
    \vspacewithsolnstrue\displayworkareatrue}
\DeclareOptionX{ftbsolns}{%
    \vspacewithsolnstrue\displayworkareatrue}
%    \end{macrocode}
%    \begin{macro}{useforms}
%    Use forms (if online option is taken); otherwise draw rectangles for
%    multiple choice/multiple selection questions.
%    \begin{macrocode}
\DeclareOptionX{useforms}{\AtEndOfPackage{\def\sqstar{*}}}
%    \end{macrocode}
%    \end{macro}
%    \end{macro}
%    \end{macro}
%    \end{macro}
%    \end{macro}
%
%    \paragraph*{PDF Options} The various options to go beyond paper!
%
%    \begin{macro}{online}
%    \begin{macro}{pdf}
%    \begin{macro}{links}
%    \begin{macro}{email}
%    Options related to the interactive capability of \textsf{eqexam}.
%    \begin{macrocode}
\DeclareOptionX{online}{\let\eq@online=y\ExecuteOptionsX{links}}
\DeclareOptionX{pdf}{\let\load@web=y}
\DeclareOptionX{links}{\let\load@web=y\let\load@exerquiz=y}
\DeclareOptionX{email}{\let\use@email=y\ExecuteOptionsX{online}}
%    \end{macrocode}
%    \end{macro}
%    \end{macro}
%    \end{macro}
%    \end{macro}
%    \begin{macro}{obeylocalversions}
%    This option is used for multiple versions of a document.
%    \begin{macrocode}
\newif\ifeqobeylocalversion \eqobeylocalversionfalse
\DeclareOptionX{obeylocalversions}{\eqobeylocalversiontrue}
%    \end{macrocode}
%    \end{macro}
%    \begin{macro}{usexkv}
%    Causes the \textsf{xkeyval} package to be input, this option extends the
%    option list of
% \cs{fillIn}.
%    \begin{macrocode}
\DeclareOptionX{usexkv}{\let\eq@usexkeys=y}
\let\eq@usexkeys=n
%    \end{macrocode}
%    \end{macro}
%\paragraph*{Renditions} Options relating to renditions.
%    \begin{macro}{max}
%    \begin{macro}{rendition}
%    The \texttt{max} and \texttt{rendition} option can be used instead of the
%    \verb!\numVersions{2}! and \verb!\forVersion{a}!, respectively. These options
%    allow you to set the version information though a package option. This allows us, for
%    example, to use a \textsf{cfg} file such as \texttt{rendition.cfg} to dynamically set the version.
%    This feature is used primarily by \textsf{AeB Exam Builder}.
%    \changes{v2.0}{2010/03/05}
%    {%
%        Switched over to xkeyval, added max and rendition to be consistent
%        with the renditions package, though we don't use the rendition package
%        itself. eqexam has a more extensive renditions system already.
%        Introduced this mostly for use AeB Exam Builder.
%    }
%    \begin{macrocode}
\let\eq@renditionOptions\@empty
\let\eq@max@selected\@empty \let\eq@ren@selected\@empty
\DeclareOptionX{max}{\def\eq@max@selected{#1}%
    \g@addto@macro\eq@renditionOptions{\numVersions{#1}}}
\DeclareOptionX{rendition}{\def\eq@ren@selected{#1}%
    \g@addto@macro\eq@renditionOptions{\forVersion{#1}}}
%    \end{macrocode}
%    \end{macro}
%    \end{macro}
%    \paragraph*{Randomization} Options relating to randomization.
%    \begin{macro}{allowrandomize}
%    Use this option to randomize the choices of a multiple choice question.
%    \begin{macrocode}
\DeclareOptionX{allowrandomize}{\AtEndOfPackage{\inputRandomizeChoices}}
\def\inputRandomizeChoices{\InputIfFileExists{aebrandom.def}
    {\typeout{inputting aebrandom.def}}{cannot find aebrandom.def}}
%    \end{macrocode}
%    \end{macro}
%    \paragraph*{Set Misc. Defaults and Helper Macros.}
%    We set some defaults, and define macros for use by the document author.
%    \begin{macrocode}
\let\eq@online=n
\let\use@email=n
\let\load@web=n
\let\load@exerquiz=n
\def\sqLinks{\def\sqstar{}}\sqLinks
\def\sqForms{\def\sqstar{*}}
\def\NoSpaceToWork{\let\eq@insertverticalspace=n}
\def\SpaceToWork{\let\eq@insertverticalspace=y}
\let\eq@nototals=n
\let\eq@nosummarytotals=n
\let\eq@parttotals=y
\let\eqx@separationrule=y
\let\eq@insertverticalspace=y
\let\eqex@coverpage\relax
\def\@reportpoints{0}
\let\marginpoints=\@empty
\let\totalsbox=\hfil
%    \end{macrocode}
%
%    \subsection{Options from \textsf{exerquiz}}
%
%    Options from \textsf{exerquiz} that are useful for this package.
%    \begin{macro}{forpaper}
%    \begin{macro}{forcolorpaper}
%    Here is the list of options of \textsf{exerquiz} we plan to recognize.
%    \begin{macrocode}
\DeclareOptionX{forpaper}{\eqforpapertrue
    \PassOptionsToPackage{monochrome}{\eq@ColorPackage}}
\DeclareOptionX{forcolorpaper}{\eqforpapertrue}  % for print
%    \end{macrocode}
%    \end{macro}
%    \end{macro}
%    \begin{macro}{preview}
%    Preview shows outlines for form fields.
%    \begin{macrocode}
\DeclareOptionX{preview}{\previewtrue}
%    \end{macrocode}
%    \end{macro}
%    \begin{macro}{nosolutions}
%    \begin{macro}{nohiddensolutions}
%    \begin{macro}{noHiddensolutions}
%    \begin{macro}{solutionsafter}
%    \changes{v1.7b}{2007/21/07}{Added a \texttt{solutionsonly} option}
%    \begin{macro}{solutionsonly}
%    Solutions related options
%    \begin{macrocode}
%    dps 03/04/11
\DeclareOptionX{nosolutions}{\eq@nolinktrue\eq@nosolutionstrue
    \displayworkareatrue}
\DeclareOptionX{nohiddensolutions}{\eq@globalshowsolutionstrue}
\DeclareOptionX{noHiddensolutions}%
    {\eq@globalshowsolutionstrue\AtBeginDocument{\def\Hidesymbol{h}}}
\DeclareOptionX{solutionsafter}{\eq@solutionsaftertrue
    \displayworkareafalse}
\DeclareOptionX{solutionsonly}{%
    \solutionsonlytrue\answerkeytrue\displayworkareafalse
    \AtEndOfPackage{\therearesolutionstrue\let\exerSolnsHeadnToc\relax}}
%    \end{macrocode}
%    \end{macro}
%    \end{macro}
%    \end{macro}
%    \end{macro}
%    \end{macro}
%    \begin{macro}{proofing}
%    The \texttt{proofing} option sets a switch that controls whether
%    the checkbox for multiple choice questions appears, and whether
%    the answer for the \cs{fillin} command appears. \cmd{\ifeq@proofing}
%    is set to true when the \texttt{answerkey} option is taken.
%    \begin{macrocode}
\DeclareOptionX{proofing}{\eq@proofingtrue}
%    \end{macrocode}
%    We provide two helper commands for turning on or off proofing. These are
%    \DescribeMacro{\showproofing}\cmd{\showproofing} for turning on proofing
%    and \DescribeMacro{\hideproofing}\cmd{\hideproofing} for turning off proofing.
%    There was some reason for defining these two, but can't remember now.
%    \begin{macrocode}
\newcommand{\showproofing}{\eq@proofingtrue}
\newcommand{\hideproofing}{\eq@proofingfalse}
%    \end{macrocode}
%    \end{macro}
%    \begin{macro}{showgrayletters}
%    \changes{v1.7c}{2008/08/21}
%    {
%       Added the \texttt{showgrayletters} option to eqexam
%       (ported from exerquiz)
%    }
%    When this option is in effect, capital letters in gray appear under
%    the multiple choice question boxes.
%    \begin{macrocode}
\newif\ifaebshowgrayletters\aebshowgraylettersfalse
\DeclareOptionX{showgrayletters}%
    {\AtEndOfPackage{\aebshowgrayletterstrue}}
%    \end{macrocode}
%    \end{macro}
%
%\changes{v2.0i}{2011/04/17 }
%{
%   Added the switch \cs{ifdisplayworkarea} to better control when the
%   work area is to be displayed.
%}
%    \begin{macrocode}
\newif\ifdisplayworkarea \displayworkareafalse
%    \end{macrocode}
%
%    \paragraph*{Color packages}
%    We set the color package, \texttt{xcolor} preferred.
%    \begin{macrocode}
\IfFileExists{xcolor.sty}%
{\def\eq@ColorPackage{xcolor}\PassOptionsToPackage{xcolor}{table}}
{\def\eq@ColorPackage{color}}
%    \end{macrocode}
%    \DescribeMacro{noxcolor}\texttt{noxcolor} forces the use of the color package.
%    \begin{macrocode}
\DeclareOptionX{noxcolor}{\def\eq@ColorPackage{color}}
%    \end{macrocode}
%
% \subsection{Drivers Recognized}
% These drivers are only relevant when
% a \textsf{PDF} option is taken (\texttt{pdf}, \texttt{links}, \texttt{online}, \texttt{email}).
% For ordinary paper documents, it is not necessary to specify the driver. If you
% put the assignment/homework/test (solns) on the web, suggested option is \texttt{pdf},
% this inputs hyperref, and the document info dialog is filled in.
%    \begin{macro}{dvipsone}
%    \begin{macro}{dvips}
%    \begin{macro}{pdftex}
%    \begin{macro}{dvipdfm}
%    \begin{macro}{textures}
% The list of recognized and supported drivers.
%    \begin{macrocode}
\DeclareOptionX{dvipsone}{%
    \def\eq@drivernum{0}\def\eqDriverName{dvipsone}
    \PassOptionsToPackage{\eq@ColorPackage}{dvipsone}}
\DeclareOptionX{dvips}{\def\eq@drivernum{0}\def\eqDriverName{dvips}
    \PassOptionsToPackage{\eq@ColorPackage}{dvips}}
\DeclareOptionX{pdftex}{\def\eq@drivernum{1}\def\eqDriverName{pdftex}
    \PassOptionsToPackage{\eq@ColorPackage}{pdftex}}
\DeclareOptionX{dvipdfm}{\def\eq@drivernum{2}\def\eqDriverName{dvipdfm}
    \PassOptionsToPackage{\eq@ColorPackage}{dvipdfm}}
\DeclareOptionX{textures}{%
    \def\eq@drivernum{3}\def\eqDriverName{textures}
    \PassOptionsToPackage{\eq@ColorPackage}{textures}}
\DeclareOptionX*{%
    \PassOptionsToPackage{\CurrentOption}{\eq@ColorPackage}}
\def\eq@drivernum{5}
\let\eqDriverName\@empty
%    \end{macrocode}
%    \end{macro}
%    \end{macro}
%    \end{macro}
%    \end{macro}
%    \end{macro}
% If \textsf{exerquiz} is not loaded, when we need to define some of the switches that
% were defined in \textsf{exerquiz}.
%
% The following switches are used in the options above, and are also defined
% in web, exerquiz, or eforms.
%    \begin{macrocode}
\newif\ifeq@solutionsafter \eq@solutionsafterfalse
\newif\ifsolutionsonly\solutionsonlyfalse
\newif\ifeq@hidesolution \eq@hidesolutionfalse
\newif\ifeq@globalshowsolutions \eq@globalshowsolutionsfalse
\newif\ifeq@nosolutions \eq@nosolutionsfalse
\newif\ifeq@proofing \eq@proofingfalse
\newif\ifeq@nolink \eq@nolinkfalse
\newif\ifpreview \previewfalse
\newif\ifeqforpaper \eqforpaperfalse
%    \end{macrocode}
% We define the commands for inputting the CFG files.
%    \begin{macrocode}
\@for\eqe@tmp@i:={},i,ii,iii,iv,v,vi\do{\expandafter
    \edef\csname eqemyconfig\eqe@tmp@i\endcsname
        {\noexpand\InputIfFileExists{eqexam\eqe@tmp@i.cfg}{}{}}%
    \eqe@tmp@exp
}
%    \end{macrocode}
%
%    \subsection{Bring in Config Files}
%
%    First read \texttt{web.cfg}, to possibly get the driver, then input
%    \texttt{eqecus.opt}, which is used to create convenient custom options.
%    \changes{v1.6e}{2006/05/07}{%
%    Added a custom option feature. Just before the options are processed, the
%    tex compiler looks for the file \texttt{eqecus.opt}. This file should
%    contain one or more custom options. }
%
% Here is an example of usage for defining your own custom options, must be based on
% current options, this code would be in the file \texttt{eqecus.opt}.
%\begin{verbatim}
%\DeclareOptionX{atbdbopts}
%{%
%    \ExecuteOptionsX{online}
%    \ExecuteOptionsX{forcolorpaper}
%    \ExecuteOptionsX{nosolutions}
%    \ExecuteOptionsX{nopoints}
%    \ExecuteOptionsX{nototals}
%    \ExecuteOptionsX{nospacetowork}
%    \ExecuteOptionsX{obeylocalversions}
%    \ExecuteOptionsX{myconfig}
%}
%\end{verbatim}
% The following config files are input prior to \cs{ProcessOptionsX}, and can,
% therefore, contain declaration of options. \texttt{web.cfg} usually only
% specifies the default driver. \texttt{eqecus.opt} is used by \textsf{@EASE},
% but can be used locally.
%    \begin{macrocode}
\InputIfFileExists{web.cfg}{}{}
\InputIfFileExists{eqecus.opt}{}{}
%    \end{macrocode}
%\changes{v2.0}{2010/03/05}
%{
% Added exambuilder.cfg for use by AeB Exam Builder, to pass the values of the options
% max and rendition to eqexam.
%}
% These two are used by the \textsf{rendition} package and the exam builder utility.
%    \begin{macrocode}
\InputIfFileExists{rendition.cfg}{}{}
\InputIfFileExists{exambuilder.cfg}{}{}
%    \end{macrocode}
%
% \subsection{Process Options}
%
% Now process the options.
%    \begin{macrocode}
\ProcessOptionsX
%    \end{macrocode}
% If \texttt{nocustomdesign} option is taken, we set the switch
% \cmd{\eqcustomdesignfalse}.
%    \begin{macrocode}
\if\eqe@nocustomdesign1\eqcustomdesignfalse\fi
%    \end{macrocode}
% Define a \cs{immediate}\cs{write} helper macro.
%    \begin{macrocode}
\long\def\eqe@IWO#1{\immediate\write#1}
%    \end{macrocode}
%    \paragraph*{Early definitions for the \texttt{fortextbook} option.}
%    \begin{macro}{\showAllAnsAtEnd}
%    If the user has chosen the \texttt{vspacewithsolns} option, we must turn
%    of all other solution options, namely \texttt{answerkey}. This command is
%    used internally.
%    \begin{macrocode}
\newcommand{\showAllAnsAtEnd}{%
    \makeAnsEnvForSolnsAtEnd
    \answerkeytrue\eq@proofingtrue
    \eq@solutionsaftertrue\vspacewithsolnstrue
    \displayworkareafalse\withsoldoctrue
}
%    \end{macrocode}
%    \begin{macro}{\makeAnsEnvForSolnsAtEnd}
%    One user wanted to be able to use the \texttt{answers} environment
%    in the solutions section at the end of the document (when the
%    \texttt{vspacewithsolns} is used). Here it is. This definition
%    is added to the definition of \cmd{\showAllAnsAtEnd}.
%    \begin{macrocode}
\newcommand{\makeAnsEnvForSolnsAtEnd}{%
  \proofingsymbol{\ding{52}}%
  \let\answers\answers@sq
  \let\endanswers\endanswers@sq
  \let\manswers\manswers@sq
  \let\endmanswers\endmanswers@sq
}
%    \end{macrocode}
% \cs{writeAllAnsAtEnd} writes the \cs{showAllAnsAtEnd} command to the solutions file.
%    \begin{macrocode}
\def\writeAllAnsAtEnd{%
    \ifsolutionsonly\else
    \let\quiz@solns\ex@solns
    \eqe@IWO\quiz@solns{\string\showAllAnsAtEnd}%
    \fi
}
%    \end{macrocode}
%    \end{macro}
%    \end{macro}
%    \begin{macro}{\setSolnMargins}
%    Sets the value of \cs{eqemargin} in the context of the solution file,
%    this command is redefined later.
%    \begin{macrocode}
\newcommand{\setSolnMargins}[1]{\setlength\eqemargin{#1}}
%    \end{macrocode}
%    \end{macro}
%    (2011/05/08) In the new version of \textsf{eqexam}, the one that makes
%    the problems within an \texttt{exam} environment, into a list, the
%    solutions file that appears at the end of the document also needs to be
%    put into a list. Here, we define the command that writes the beginning of
%    the \texttt{eqequestions} environment to the beginning of the
%    \cs{jobname.sol} file.
%    \begin{macrocode}
\def\writeBeginEqeQuestions{%
    \ifsolutionsonly\else
    \let\quiz@solns\ex@solns
    \eqe@IWO\quiz@solns{\string\setSolnMargins{\the\eqemargin}}%
    \eqe@IWO\quiz@solns{\string\begin{eqequestions}}%
    \fi
}
%    \end{macrocode}
% (2011/05/08) We define the command that writes the end of the \texttt{eqequestions}
% environment to the beginning of the \cs{jobname.sol} file.
%    \begin{macrocode}
\def\writeEndEqeQuestions{%
    \ifsolutionsonly\else
    \let\quiz@solns\ex@solns
    \eqe@IWO\quiz@solns{\string\end{eqequestions}}%
    \fi
}
%    \end{macrocode}
% If \cs{ifvspacewithsolns} we set the switches need to simulate
% \texttt{nosolutions}.
%    \begin{macrocode}
\ifvspacewithsolns
    \answerkeyfalse\eq@proofingfalse\eq@solutionsafterfalse
    \eq@nolinkfalse\eq@nosolutionsfalse\displayworkareatrue
\fi
%    \end{macrocode}
%
% \subsection{Save Switch Values}
%
% Now, save the current state of the switches defined above. When, and if,
% the packages web, \textsf{exerquiz} and \textsf{eforms} are loaded, they will overwrite the
% choices set by the author, so we save them.
%    \begin{macrocode}
\let\savedeq@online\eq@online
\let\savedifeq@solutionsafter\ifeq@solutionsafter
\let\savedifeq@hidesolution\ifeq@hidesolution
\let\savedifeq@globalshowsolutions\ifeq@globalshowsolutions
\let\savedifeq@nosolutions\ifeq@nosolutions
\let\savedifeq@proofing\ifeq@proofing
\let\savedifeq@nolink\ifeq@nolink
\let\savedifpreview\ifpreview
\let\savedifeqforpaper\ifeqforpaper
\let\ifnosolutions\ifeq@nosolutions
%    \end{macrocode}
% \section{Required Packages}
% The following are the required packages for \textsf{eqexam}.
%    \begin{macrocode}
\RequirePackage{amstext,amssymb}
%    \end{macrocode}
% Bring the \texttt{comment} package in early, before \texttt{verbatim}, these two
% clash a bit.
%    \begin{macrocode}
\RequirePackage{comment}
%    \end{macrocode}
% The macro \cs{includeexersolutions} is defined in \textsf{eqexam.def}. We execute
% the command \cs{include@solutions} before the web package is loaded. The \textsf{web} package
% has a \cs{AtEndDocument} as well, and inserts a new page that we don't want.
%    \begin{macrocode}
\AtEndDocument{\includeexersolutions}
%    \end{macrocode}
% If user has specified one of the pdf options (pdf, links, online, email), we bring in
% the web package.
%    \begin{macrocode}
\@ifpackageloaded{web}{\let\load@web=y}{%
    \ifx\load@web y \ifnum\eq@drivernum=5
            \PackageInfo{eqexam}{You have not selected a driver %
                for eqexam. Perhaps the \MessageBreak
                 driver is introduced through web.cfg%
            }\fi\RequirePackage[\eqDriverName]{web}%
        \edef\@pdfcreator{\@pdfcreator, The eqexam Package}\fi
}
%    \end{macrocode}
% If user has specified links, online or email, we bring in the exerquiz package.
%    \begin{macrocode}
\@ifpackageloaded{exerquiz}{\let\load@exerquiz=y}{%
    \ifx\load@exerquiz y
        \RequirePackage[nodljs]{exerquiz}[2011/08/30]
%    \end{macrocode}
% We input \textsf{exerquiz} with the \texttt{nodljs}, we don't need all the JavaScript
% to process interactive shortquizzes or quizzes, but we do want the option of
% adding in document JavaScript, so after we input \textsf{exerquiz}, we set the
% switches to allow these features.
%    \begin{macrocode}
        \let\importdljs=y
        \let\execjs=y
    \fi
}
%    \end{macrocode}
% Here is a fix to a problem I've been having previewing in \textsf{dviwindo}.  I've traced
% the problem down to \cs{@pdfviewparams}. Redefining \cs{@pdfviewparams} as follows.
%    \begin{macrocode}
\def\eqDvipsone{dvipsone}
\@ifpackageloaded{hyperref}
    {\ifx\eqDriverName\eqDvipsone
        \renewcommand\@pdfviewparams{ null null null}\fi
    }{\let\textorpdfstring\@firstoftwo}
%    \end{macrocode}
% Now that we have possibly input \textsf{web} or \textsf{exerquiz}, we need to restore the authors options.
%    \begin{macrocode}
\let\eq@online\savedeq@online
\let\ifeq@solutionsafter\savedifeq@solutionsafter
\def\ifsolutionsafter{\ifeq@solutionsafter} % user interface
\let\ifeq@hidesolution\savedifeq@hidesolution
\let\ifeq@globalshowsolutions\savedifeq@globalshowsolutions
\let\ifeq@nosolutions\savedifeq@nosolutions
\let\ifeq@proofing\savedifeq@proofing
\let\ifeq@nolink\savedifeq@nolink
\let\ifpreview\savedifpreview
\let\ifeqforpaper\savedifeqforpaper
%    \end{macrocode}
% Other packages of interest.
%    \begin{macrocode}
\RequirePackage{\eq@ColorPackage}
%    \end{macrocode}
% We require a minimal version for \textsf{xcolor}.
%    \begin{macrocode}
\@ifpackageloaded{xcolor}{\@ifpackagelater{xcolor}{2004/07/04}{}{%
    \PackageError{eqexan}{%
    *************************************************\MessageBreak
    * Your Version of `xcolor.sty' is too old!\MessageBreak
    * You need the version from 2004/07/04 or newer\MessageBreak
    * or use: \string\usepackage[noxcolor]{eqexam}\MessageBreak
    * or \string\documentclass[noxcolor]{article}\MessageBreak
    *************************************************}{}%
    }%
}{}
\RequirePackage{calc}
\RequirePackage{pifont}
%    \end{macrocode}
% Here, I input the \texttt{verbatim} package after the comment package.
%    \begin{macrocode}
\RequirePackage{verbatim}
%    \end{macrocode}
% When constructing paper tests, I often use a multi-column format for some of
% the questions, so let's require this package
%    \begin{macrocode}
\RequirePackage{multicol}
\setlength\columnseprule{.4pt}
\raggedcolumns\multicolsep=3pt
%    \end{macrocode}
% For the \texttt{fortextbook} option, we require \textsf{eso-pic}.
%    \begin{macrocode}
\edef\eqe@reqPack{\ifeqfortextbook\noexpand\RequirePackage{eso-pic}\else
\relax\fi}
\eqe@reqPack
%    \end{macrocode}
% \section{Page Layout}
% (2011/05/08) The revised version of \texttt{eqexam} allows the document author
% to more easily design the size of the page; the new version makes all content
% inside the \texttt{exam} environment into a list, this gives us better control over the
% margins and spacing.
%    \begin{macro}{\eqexammargin}
% (2011/05/08) Use this command to set the margin for the \texttt{exam} environment.
%    \begin{macrocode}
\newlength{\eqemargin}
\def\eqe@decPointPrb{.}\def\eqe@dpsepPrb{\ }
\def\eqe@prtsepPrb{\ }\def\eqe@hspannerPrb{\ }
\newcommand{\eqexammargin}[2][\normalsize\normalfont\bfseries]{%
    \settowidth{\eqemargin}{#1#2\eqe@decPointPrb\eqe@hspannerPrb}}
%    \end{macrocode}
% (2011/05/08) The default margin for the \textsf{eqexam} environment, two digits and a space.
%    \begin{macrocode}
\eqexammargin{00}
%    \end{macrocode}
%\changes{v3.0s}{2012/01/01}{%
% Moved a copy of \cs{eqe@spannerSoln} out of the \texttt{ftbsty} to the \texttt{package} section,
% its needed here as well.}
%    \begin{macrocode}
\def\eqe@hspannerSoln{\ }   % space after prob number
%    \end{macrocode}
%    \end{macro}
%    \begin{macro}{\eqeSetExamPageParams}
% (2011/05/08) The default spacing maximizes the amount of space on the page.
%    \begin{macrocode}
\newcommand{\eqeSetExamPageParams}{%
    \setlength{\headheight}{12pt}
    \setlength{\topmargin}{-.5in}
    \setlength{\headsep}{20pt}
    \setlength{\oddsidemargin}{0pt}
    \setlength{\evensidemargin}{0pt}
    \setlength{\marginparsep}{11pt}
    \setlength{\marginparwidth}{35pt}
    \setlength{\footskip}{11pt}
}
%    \end{macrocode}
%    \end{macro}
%    \begin{macro}{\eqExamPageLayout}
% Set the basic parameters of this exam page package
%    \begin{macrocode}
\newcommand{\eqExamPageLayout}{%
    \setlength\textwidth\paperwidth
    \addtolength{\textwidth}{-2in}
    \addtolength{\textwidth}{-\oddsidemargin}
    \setlength\textheight{\paperheight}
    \addtolength\textheight{-2in}
    \addtolength\textheight{-\headheight}
    \addtolength\textheight{-\headsep}
    \addtolength\textheight{-\topmargin}
    \addtolength\textheight{-\footskip}
}
%    \end{macrocode}
% (2011/05/08) If \texttt{usecustomdesign} is used it is expected that
% \cs{eqe\-Set\-Exam\-Page\-Params} and \cs{eqeSetExamPageParams} are redefined in he preamble,
% otherwise, we set up the standard parameters; otherwise
%    \begin{macrocode}
\ifeqcustomdesign\else
\eqeSetExamPageParams
\eqExamPageLayout
\fi
%    \end{macrocode}
%    \end{macro}
% A simple page layout scheme for this exam.
%    \begin{macrocode}
\newcommand{\ps@eqExamheadings}
{%
    \renewcommand{\@oddhead}{%
    {%
        \normalfont\normalsize
            \ifnum\value{page}<2 \hfil
            \else\eqExamRunHead\fi
    }%
}
\renewcommand{\@evenhead}{\@oddhead}
\renewcommand{\@oddfoot}{\settotalsbox\runExamFooter}
\renewcommand{\@evenfoot}{\@oddfoot}}
\raggedbottom
%    \end{macrocode}
% \section{Counters, Lengths and Tokens}
% Some counters to keep track of things. \DescribeEnv{eqpointsofar}
% \DescribeEnv{eqpointsthispage} The first two counters keep track, respectively,
% of the total points so far up the current page, and the number of points
% on the current page. \DescribeEnv{eq@numparts} The counter \texttt{eq@numparts}
% holds the number of parts of the multi-part question.
%    \begin{macrocode}
\newcounter{eqpointsofar}
\newcounter{eqpointsthispage}
\newcounter{eq@numparts}
\newcounter{eq@count}
\newtoks\partNames \partNames={}
\newlength{\eq@tmplengthA}
\newbox{\eq@pointbox}
\newlength{\eq@pointboxtotalheight}
%    \end{macrocode}
% Some scratch registers to do calc calculations.
%    \begin{macrocode}
\newlength{\eqetmplengtha}
\newlength{\eqetmplengthb}
%    \end{macrocode}
% \section{Some Macros to Support the Options}
% We make a few definitions to support various options.
%    \begin{macrocode}
\def\PointsOnLeft{\def\@reportpoints{1}\let\marginpoints\eqleftmargin}
\def\PointsOnRight{\def\@reportpoints{2}\relax
    \let\marginpoints\eqrightmarginbox}
\def\PointsOnBothSides{\def\@reportpoints{3}\relax
    \let\marginpoints\eqbothmargins}
\newif\ifeqe@nopoints \eqe@nopointsfalse
\def\NoPoints{\eqe@nopointstrue\def\@reporttotals{0}\let\totalsbox=\hfil
    \let\marginpoints\@empty\let\eq@nosummarytotals=y}
\def\TotalsOnLeft{\def\@reporttotals{1}\def\totalsbox{\totalsboxleft}}
\def\TotalsOnRight{\def\@reporttotals{2}\def\totalsbox{\totalsboxright}}
\def\NoTotals{\def\@reporttotals{0}\let\totalsbox=\hfil}
\def\eoeTotalOff{\let\eq@parttotals=n}
\def\eoeTotalOn{\let\eq@parttotals=y}
\def\separationruleOn{\let\eqx@separationrule=y}
\def\separationruleOff{\let\eqx@separationrule=n}
\def\AllowFitItIn{\global\let\eq@fititin\eqfititin}
\def\DoNotFitItIn{\global\let\eq@fititin\@gobble}
%    \end{macrocode}
%\changes{v3.0p}{2011/09/22}{Added \cs{NoSolutions} to be executed in
%preamble, needed with the fortextbook package.}
%    \begin{macrocode}
\def\NoSolutions{\eq@nolinktrue\eq@nosolutionstrue
    \displayworkareatrue}
\@onlypreamble\NoSolutions
%    \end{macrocode}
%    \begin{macro}{\vspacewithkeyOn}
%    \begin{macro}{\vspacewithkeyOff}
%\changes{v2.0k}{2011/04/29}{Added user interface to the switch
%\cs{ifkeepdeclaredvspacing}, which is defined in \texttt{eqexam.def}/\textsf{exerquiz}.}
% User interface to keeping the declare vspace, even when
% the \texttt{answerkey} (or \texttt{solutionsafter}) option is taken. The switch
% \cs{ifkeepdeclaredvspacing} is defined in \texttt{eqexam.def}/\textsf{exerquiz}.
%    \begin{macrocode}
\def\vspacewithkeyOn{\keepdeclaredvspacingtrue}
\def\vspacewithkeyOff{\keepdeclaredvspacingfalse}
%    \end{macrocode}
%    \end{macro}
%    \end{macro}
%    \begin{macro}{\encloseProblemsWith}
%\changes{v1.7b}{2007/21/07}
%{%
% Added \cs{encloseProblemsWith} to support the \texttt{solutionsonly} option
%}
% \cs{encloseProblemsWith} to support the \texttt{solutionsonly} option
%    \begin{macrocode}
\def\encloseProblemsWith#1{%
    \ifsolutionsonly\excludecomment{#1}\else
    \includecomment{#1}\fi
}
%    \end{macrocode}
%    \end{macro}
% \section{Colors}
%    \begin{macro}{\proofingsymbolColor}
%    \begin{macro}{\instructionsColor}
%    \begin{macro}{\eqCommentsColor}
%    \begin{macro}{\universityColor}
%    \begin{macro}{\titleColor}
%    \begin{macro}{\authorColor}
%    \begin{macro}{\subjectColor}
%    \begin{macro}{\linkcolor}
%    \begin{macro}{\nolinkcolor}
%    \begin{macro}{\fillinColor}
%    \begin{macro}{\forceNoColor}
%    \begin{macro}{\eqEndExamTotalColor}
%    Here we list commands for controlling colors. There are some other
%    colors defined in the stand alone code.
%    \changes{v1.6e}{2006/05/07}
%    {
%    Added easy user access to various colors, \cs{proofing\-symbol\-Color},
%    \cs{instructionsColor}, \cs{eqCommentsColor}, \cs{authorColor},
%    \cs{title\-Color}, \cs{universityColor} and \cs{subjectColor}
%    }
%    \begin{macrocode}
\providecommand{\proofingsymbolColor}[1]{\def\@proofingsymbolColor{#1}}
\proofingsymbolColor{red}
\providecommand{\instructionsColor}[1]{\def\@instructionsColor{#1}}
\instructionsColor{blue}
\providecommand{\eqCommentsColor}[1]{\def\@eqCommentsColor{#1}}
\eqCommentsColor{blue}
\providecommand{\eqCommentsColorBody}[1]{\def\@eqCommentsColorBody{#1}}
\eqCommentsColorBody{black}
\providecommand{\universityColor}[1]{\def\webuniversity@color{#1}}
\universityColor{blue}
\providecommand{\titleColor}[1]{\def\webtitle@color{#1}}
\titleColor{black}
\providecommand{\authorColor}[1]{\def\webauthor@color{#1}}
\authorColor{black}
\providecommand{\subjectColor}[1]{\def\websubject@color{#1}}
\subjectColor{blue}
\providecommand{\linkcolor}[1]{\def\@linkcolor{#1}}
\linkcolor{blue}
\providecommand{\nolinkcolor}[1]{\def\@nolinkcolor{#1}}
\nolinkcolor{black}
\providecommand{\eqEndExamTotalColor}[1]{\def\endexamtotal@color{#1}}
\eqEndExamTotalColor{black}
\def\fillinColor#1{\def\eq@fillinColor{#1}}\fillinColor{red}
\newcommand{\forceNoColor}{%
    \proofingsymbolColor{black}\instructionsColor{black}
    \eqCommentsColor{black}\universityColor{black}
    \titleColor{black}\authorColor{black}
    \subjectColor{black}\linkcolor{black}
    \nolinkcolor{black}\fillinColor{black}
    \if\load@web y\sectionColor{black}\fi
}
%    \end{macrocode}
%    \end{macro}
%    \end{macro}
%    \end{macro}
%    \end{macro}
%    \end{macro}
%    \end{macro}
%    \end{macro}
%    \end{macro}
%    \end{macro}
%    \end{macro}
%    \end{macro}
%    \end{macro}
% \section{Version Control}
% Here are some simple macros use to create two versions,
% version A and version B, of the same test.
%    \begin{macro}{\examNum}
% Convenience macro for holding the exam number. It sets the
% value of \cs{nExam}.
%    \begin{macrocode}
\def\examNum#1{\def\nExam{#1}}
\examNum{1}
%    \end{macrocode}
%    \end{macro}
%    \begin{macro}{\Exam}
%    \begin{macro}{\sExam}
% Convenience macros for titling the exam. Usage:
%\begin{verbatim}
%\VersionAtext{Test~\nExam--Version A}
%\VersionBtext{Test~\nExam--Version B}
%\shortVersionAtext{T\nExam A}
%\shortVersionBtext{T\nExam B}
%
%\examNum{1}
%\forVersion{c}
%\subject[C3]{Calculus III}
%\title[\sExam]{\Exam}
%\author{Dr.\ D. P. Story}
%\end{verbatim}
% These next two definitions are overwritten by the two
% commands \cs{longTitleText} and \cs{shortTitleText}.
%    \begin{macrocode}
\def\Exam{\ifAB{\eq@VersionAtext}{\eq@VersionBtext}}
\def\sExam{\ifAB{\eq@shortVersionAtext}{\eq@shortVersionBtext}}
%    \end{macrocode}
%    \end{macro}
%    \end{macro}
%    \begin{macro}{\VersionAtext}
%    \begin{macro}{\VersionBtext}
%    \begin{macro}{\shortVersionAtext}
%    \begin{macro}{\shortVersionBtext}
% Convenience macros for entering the text for the title, long and short
% for versions A and B.
%    \begin{macrocode}
\def\VersionAtext#1{\def\eq@VersionAtext{#1}}
\def\VersionBtext#1{\def\eq@VersionBtext{#1}}
\def\shortVersionAtext#1{\def\eq@shortVersionAtext{#1}}
\def\shortVersionBtext#1{\def\eq@shortVersionBtext{#1}}
\VersionAtext{Exam~\nExam--Version A}
\VersionBtext{Exam~\nExam--Version B}
\shortVersionAtext{Exam~\nExam A}
\shortVersionBtext{Exam~\nExam B}
%    \end{macrocode}
%    \end{macro}
%    \end{macro}
%    \end{macro}
%    \end{macro}
% In this section we introduce a new set of commands that supersedes
% the commands defined above. Those commands were limited to only
% two versions. The ones below can handle up to $26$ versions.
%    \begin{macrocode}
\newtoks\eqtemptokena
\newtoks\eqtemptokenb
%    \end{macrocode}
%    \begin{macro}{\numVersions}
% In the preamble, declare the number of versions for this document
% using \cs{numVersions}, e.g., |\numVersions{3}|. This sets the value
% of \cs{eq@nVersions}
%    \begin{macrocode}
\def\numVersions#1{\ifnum#1>26\def\eq@nVersions{26}%
    \PackageWarning{eqexam}{The value of \string\numVersions is too
    large. \MessageBreak Choose a natural number less than 27.}
    \else\def\eq@nVersions{#1}\fi}
%    \end{macrocode}
%    \end{macro}
%    \begin{macro}{\longTitleText}
%    \begin{macro}{\endlongTitleText}
%    \begin{macro}{\shortTitleText}
%    \begin{macro}{\endshortTitleText}
% Next we state the long and short titles for our document,
% one for each of our declare number of versions given earlier.
% For example, we can use the value \cs{nExam} in out titles. Usage:
%\begin{verbatim}
%    \longTitleText
%        {Test~\nExam--Version A}
%        {Test~\nExam--Version B}
%        {Test~\nExam--Make Up}
%    \endlongTitleText
%    \shortTitleText
%        {T\nExam A}
%        {T\nExam B}
%        {T\nExam MU}
%    \endshortTitleText
%\end{verbatim}
% I've added markers that delimit the end of the arguments. In this
% way, the end of the list of titles can be detected, even though
% the number of titles is not the same as what is declared by the
% \cs{numVersions}.
%
% If there are more titles than what is declared, the rest are absorbed (gobbled).
% If there are fewer titles than declared, a {\LaTeX} package error is generated,
% and substitute titles are generated.
% \changes{v1.9f}{2009/10/06}
% {
%    Modified \cs{longTitleText}, \cs{shortTitleText} to have an optional
%    argument (A--Z;a--z). You can select a particular title from a list
%    of titles. If no optional argument is passed, then the title determined
%    by \cs{forVersion} is used.
% }
% Modified \cs{longTitleText} and \cs{shortTitleText} to have an optional
% argument (A--Z;a--z). You can select a particular title from a list
% of titles. If no optional argument is passed, then the title determined
% by \cs{forVersion} is used.
%    \begin{macrocode}
\newcommand{\longTitleText}[1][]{%
    \ifeqglobalversion\let\eq@selectedVersion@save\eq@selectedVersion
    \else\let\eq@selectedVersion@save\relax\fi
    \uppercase{\def\eqe@localTextTitle{#1}}%
    \ifx\eqe@localTextTitle\@empty\else
    \expandafter\forVersion\expandafter{\eqe@localTextTitle}\fi
    \eqe@contTitleText{\Exam}{\endlongTitleText}%
}
\def\endlongTitleText{l}
\newcommand{\shortTitleText}[1][]{%
    \ifeqglobalversion\let\eq@selectedVersion@save\eq@selectedVersion
    \else\let\eq@selectedVersion@save\relax\fi
    \uppercase{\def\eqe@localTextTitle{#1}}%
    \ifx\eqe@localTextTitle\@empty\else
    \expandafter\forVersion\expandafter{\eqe@localTextTitle}\fi
    \eqe@contTitleText{\sExam}{\endshortTitleText}%
}
\def\endshortTitleText{s}
%    \end{macrocode}
% Both title commands, above, call this macro which sets the environment
% for \cs{@gatherTitleText}, which gathers the list of titles.
%    \begin{macrocode}
\def\eqe@contTitleText#1#2{%
    \setcounter{eq@count}{0}%
    \eqtemptokena={}\let\endtitleMarker#2
    \@gatherTitleText{#1}%
}
%    \end{macrocode}
%    \end{macro}
%    \end{macro}
%    \end{macro}
%    \end{macro}
% This command gathers each title and places it as the argument of a
% \cs{v<LETTTER>} command. These are accumulated in token registers
% then saved in \cs{Exam} and \cs{sExam}.
%    \begin{macrocode}
\def\@gatherTitleText#1#2{%
    \def\eqe@argii{#2}
    \if\endtitleMarker\eqe@argii
%    \end{macrocode}
% Encountered the end marker. See if we have collected the
% correct number of titles declared. If we have collected too
% few, we note an warning in the log, and create titles.
%    \begin{macrocode}
        \ifnum\value{eq@count}>\eq@nVersions\let\eqe@next\relax
        \else\def\eqe@next{\eq@shortTitlesFix{#1}}\fi
    \else
    \stepcounter{eq@count}
        \eqtemptokenb=\expandafter{#2}
        \xdef#1{\the\eqtemptokena\expandafter\noexpand
        \csname v\Alph{eq@count}\endcsname{\the\eqtemptokenb}}
        \xdef\sExam{\the\eqtemptokena\expandafter\noexpand
        \csname v\Alph{eq@count}\endcsname{\the\eqtemptokenb}}
        \eqtemptokena=\expandafter{#1}
        \ifnum\value{eq@count}<\eq@nVersions
            \def\eqe@next{\@gatherTitleText{#1}}%
        \else
            \def\eqe@next{%
                \if\endtitleMarker\endlongTitleText
                    \expandafter\eqe@absorbTokensLong
                \else
                    \expandafter\eqe@absorbTokensShort
                \fi
            }%
        \fi
    \fi
    \eqe@next
}
\long\def\eqe@absorbTokensLong#1\endlongTitleText{%
    \protected@xdef\Exam{\Exam}\ifx\eq@selectedVersion@save\relax
    \eqe@offVersion\else\expandafter\forVersion\expandafter
    {\eq@selectedVersion@save}\fi}
\long\def\eqe@absorbTokensShort#1\endshortTitleText{%
    \protected@xdef\sExam{\sExam}\ifx\eq@selectedVersion@save\relax
    \eqe@offVersion\else\expandafter\forVersion\expandafter
    {\eq@selectedVersion@save}\fi}
%    \end{macrocode}
% We have reached \cs{endtitleMarker}, but the count is still less than \cs{eq@nVersions},
%  so we'll warn the user, and create titles for user.
%    \begin{macrocode}
\def\eq@shortTitlesFix#1{%
    \PackageWarning{eqexam}{You have defined an insufficient number
    of titles for the number of versions declared in
    \string\numVersions. Please fix the problem}%
    \stepcounter{eq@count}%
    \if\endtitleMarker\endlongTitleText
        \edef\eqe@tmp{\noexpand\@gatherTitleText{\noexpand#1}
            {??---Title \# \the\value{eq@count}---??}%
            \noexpand\endlongTitleText}
    \else
        \edef\eqe@tmp{\noexpand\@gatherTitleText{\noexpand#1}
            {T\#\the\value{eq@count}??}\noexpand\endshortTitleText}
    \fi
    \addtocounter{eq@count}{-1}%
    \eqe@tmp
}
%    \end{macrocode}
% Here, we define \cs{ifAB} so that document under the old system
% still work properly, I hope. Usage of \cs{ifAB} at this point
% is discouraged.
%    \begin{macrocode}
\def\ifAB#1#2{\if\eq@selectedVersion A#1%
    \else\if\eq@selectedVersion B#2%
\fi\fi}
\def\eq@replaceToken#1{#1}
%    \end{macrocode}
%    \begin{macro}{\forVersion}
% Here is the command that does all the work. It creates alternate
% text macros for each of the versions declared using \cs{numVersions}.
%
% For example, assuming |\numVersions{3}| appeared earlier, the
% command |\forVersion{a}| (or |\forVersion{A}|) defines $3$ text commands \cs{vA}, \cs{vB} and
% \cs{vC}, each taking one argument, the text you want to display:
%\begin{verbatim}
% Name the \vA{place}\vB{date}\vC{year} of the signing of the Declaration
% of independence.
%\end{verbatim}
% Since we said |forVersion{a}| only the \cs{vA} text
% is displayed, the others are gobbled up, etc. But wait, the \cs{forVersion}
% does more than that! It also creates a series of comment environments
% |\begin{verA}/\end{verA}|, |\begin{verB}/\end{verB}|, |\begin{verC}/\end{verC}|, etc.,
% where only the version for which this compile applies will be typeset,
% the others are commented out.
%\begin{verbatim}
%\numVersions{3}
%\forVersion{b}
%...
%\begin{document}
%...
% Solve the equation for $\vA{x}\vB{y}\vC{z}$:
%\[
%\begin{verA}
%       2x + 4 = 7
%\end{verA}
%\begin{verB}
%       5y + 2 = 4
%\end{verB}
%\begin{verC}
%       3z - 2 = 2
%\end{verC}
%\]
%\end{verbatim}
% \changes{1.6b}{2006/04/03}
% {
%   Changed the alphabet environments \texttt{A}, \texttt{B}, etc.\ due to a
%   conflict with \cs{S}, the control sequence for the \textsf{amsmath} package
%   for a section: \cs{S} expands to \S.  Names changed to \texttt{verA}, \texttt{verB}, etc.
% }
% \changes{v1.6f}{2006/10/24}
% {
%   Fixed a bug in the \cs{eqe@initializeMultiVersions} command, made sure that any
%   already defined comment environments are set to relax.
% }
%    \begin{macrocode}
\newif\ifeqglobalversion \eqglobalversionfalse
\newif\ifeqlocalversion \eqlocalversionfalse
\newif\if@templocalversion \@templocalversionfalse
\def\eqe@initializeMultiVersions{%
    \let\save@message\message\let\message\@gobble
    \@tfor\eqe@tmp:=ABCDEFGHIJKLMNOPQRSTUVWXYZ\do{%
        \expandafter\let\csname v\eqe@tmp\endcsname\@gobble
        \expandafter\excludecomment\expandafter{ver\eqe@tmp}%
        \expandafter\let\csname Afterver\eqe@tmp Comment\endcsname\relax
    }\let\message\save@message
}
\AtBeginDocument{\let\eqe@initializeMultiVersions\relax}
%    \end{macrocode}
% (09/10/04) Trying to fix a bug in the case when the version selected is greater
% then the number of versions available for a given problem; that is, when modular
% arithmetic occurs (in \cs{selectVersion}).
%    \begin{macrocode}
\let\eqe@@onVersion\@empty
\def\eqe@onVersion{\g@addto@macro\eqe@@onVersion}
\let\eqe@@offVersion\empty
\def\eqe@offVersion{\g@addto@macro\eqe@@offVersion}
\let\eqe@@holdTemp\@empty
\def\eqe@holdTemp{\g@addto@macro\eqe@@holdTemp}
%    \end{macrocode}
% Two commands to turn on and off versions (the \cs{v<LETTER>} and the \texttt{ver<LETTER>}
% environment).
%
% Throughout the definitions below, we use \cs{csarg}, a command that is defined in the
% \textsf{comment} package.
%    \begin{macrocode}
\def\eqe@showArg#1{#1}
\def\eqe@turnOnComment#1{%
%    \csarg\let{v#1}\@empty
    \csarg\let{v#1}\eqe@showArg
    \edef\exp@temp{\noexpand\includecomment{ver#1}}\exp@temp
}
\def\eqe@turnOffComment#1{%
    \csarg\let{v#1}\@gobble
    \edef\exp@temp{\noexpand\excludecomment{ver#1}}\exp@temp
    \csarg\let{Afterver#1Comment}\relax
}
%    \end{macrocode}
% Finally, the \cs{forVersion} command.
%    \begin{macrocode}
\def\forVersion#1%
{%
    \eqe@initializeMultiVersions
    \let\eqe@@onVersion\@empty
    \let\eqe@@offVersion\@empty
    \global\eqglobalversiontrue
    \setcounter{eq@count}{0}%
    \uppercase{\edef\eq@selectedVersion{#1}}%
    \@ifundefined{eq@nVersions}{\PackageInfo{eqexam}{%
        \string\numVersions\space has not been declared, \MessageBreak
        taking the number of versions to be 2.}\def\eq@nVersions{2}}{}%
    \loop
        \stepcounter{eq@count}%
        \expandafter\if\Alph{eq@count}\eq@selectedVersion
            \xdef\eq@nSelectedVersion{\the\value{eq@count}}%
            \setcounter{eq@count}{27}\fi
        \ifnum\value{eq@count}<26\repeat
        \ifnum\eq@nSelectedVersion >\eq@nVersions
            \PackageError{eqexam}
            {The value of \string\forVersion
             \space(\eq@selectedVersion)\MessageBreak
             exceeds the value of \string\numVersions\space
             (\eq@nVersions)}%
            {Decrease the value of \string\forVersion.}%
        \fi
    \setcounter{eq@count}{0}%
    \let\save@message\message\let\message\@gobble
    \loop
        \stepcounter{eq@count}%
        \csarg\let{After\Alph{eq@count}Comment}\relax
        \lowercase
        {%
            \if#1\alph{eq@count}%
                \eqe@turnOnComment{\Alph{eq@count}}%
%    \end{macrocode}
% (09/10/04) Save the commands for turning  on the version with
% \cs{eqe@onVersion}, and for turning it off with
% \cs{eqe@offVeresion}.
%    \begin{macrocode}
                \edef\temp@exp{\noexpand
                    \eqe@turnOnComment{\Alph{eq@count}}}%
                \expandafter\eqe@onVersion\expandafter{\temp@exp}%
                \edef\temp@exp{\noexpand
                    \eqe@turnOffComment{\Alph{eq@count}}}%
                \expandafter\eqe@offVersion\expandafter{\temp@exp}%
            \else
                \eqe@turnOffComment{\Alph{eq@count}}%
            \fi
        }%
        \ifnum\value{eq@count}<\eq@nVersions\repeat
        \let\message\save@message
}
%    \end{macrocode}
% Let us assume version A initially, user with reset this in document.
%\changes{1.6c}{2006/05/02}
%{%
% At the end of the package, set the initial value of \cs{select\-Version} to
%    \string\verb!*+\string\selectVersion{26}+ and set the initial value
% of \cs{forVersion} to \string\verb!*+\string\forVersion{A}+.
%}
%    \begin{macrocode}
\AtEndOfPackage{\numVersions{26}\forVersion{A}%
    \eq@renditionOptions}
%    \end{macrocode}
%    \end{macro}
%    \begin{macro}{\selectVersion}
% When an exam has questions in which the number of variations are not all the same,
% then you can locally change the version between problems.
% If the first argument is empty, the first variation is chosen.
% The syntax is
%\begin{verbatim}
%\selectVersion{2}{3}
%\end{verbatim}
%This command says that the next problem has 3 variations, and here we select the second one.
%\changes{1.6a}{2006/01/22}
% {%
%   added \cs{selectVersion} command, also the command was fixed \cs{forVersion} so that it can be changed
%   within the document, added switches to control new selection.
%   Added option \texttt{ignorelocalversions}
% }
%    \begin{macrocode}
\def\selectVersion#1#2{% #1 \le #2
    \xdef\nLocalSelection{#1}%
    \xdef\nLocalVersions{#2}%
%    \end{macrocode}
% When the solutions appear at the end of the document, the version may not match
% the version for the question. We need to use a private hook defined in
% \textsf{exerquiz} (and \texttt{eqexam.def}) to reproduce the same settings going into
% each solution at the end. So, we write the \cs{selectVersion} to the solution
% file.
%    \begin{macrocode}
    \edef\exer@solnheadhook{%
    \string\selectVersion{#1}{#2}}%
%    \end{macrocode}
% Turn off messaging.
%    \begin{macrocode}
    \let\save@message\message\let\message\@gobble
%    \end{macrocode}
% Reset the selected version, the one selected in the preamble. A previous problem
% may have changed the version due to modular arithmetic.
%    \begin{macrocode}
    \eqe@@onVersion
    \ifx\eqe@@holdTemp\@empty\else
%    \end{macrocode}
% If \cs{eqe@holdTemp} is non-empty, this means that modular
% arithmetic was performed on the previous problem. We need to turn
% on the original choice, and turn off the temporary choice, then
% clear the command \cs{eqe@@holdTemp}.
%    \begin{macrocode}
        \eqe@@onVersion\eqe@@holdTemp
        \let\eqe@@holdTemp\@empty
    \fi
%    \end{macrocode}
% If \cs{eqglobalversion} is \texttt{true}, then a \cs{forVersion}
% has been executed. If the number of versions declared by
% \cs{numVersions} is greater than the number of local versions for
% this problem, then we perform modular arithmetic to get an
% appropriate alternative. It may be necessarily to temporarily put
% \cs{eqobeylocalversion} to true to accomplish, but we use change it
% back at the end.
%    \begin{macrocode}
    \ifeqglobalversion\ifnum\eq@nSelectedVersion>\nLocalVersions
%    \end{macrocode}
% If we perform modular arithmetic, turn off original choice.
%    \begin{macrocode}
        \eqe@@offVersion
%    \end{macrocode}
% Now perform mod arithmetic
%    \begin{macrocode}
        {\count0=\eq@nSelectedVersion \count2=\count0
         \advance\count0by-1 \divide\count0by\nLocalVersions
         \multiply\count0by\nLocalVersions \count2=\eq@nSelectedVersion
         \advance\count2by-\count0
         \xdef\nLocalSelection{\the\count2 }%
         \ifeqobeylocalversion\else
            \global\@templocalversiontrue
            \global\eqobeylocalversiontrue\fi}%
    \fi\fi
    \ifeqglobalversion\ifnum\eq@nSelectedVersion>\nLocalVersions
        \global\let\eqe@@holdTemp\@empty
        {\count0=\eq@nSelectedVersion \count2=\count0
         \advance\count0by-1 \divide\count0by\nLocalVersions
         \multiply\count0by\nLocalVersions \count2=\eq@nSelectedVersion
         \advance\count2by-\count0
         \xdef\nLocalSelection{\the\count2 }%
         \ifeqobeylocalversion\else
            \global\@templocalversiontrue
            \global\eqobeylocalversiontrue\fi}%
    \fi\fi
    \ifeqobeylocalversion
        \global\eqlocalversiontrue
        \setcounter{eq@count}{0}%
        \ifx\nLocalSelection\@empty\def\nLocalSelection{1}\fi
        \ifnum\nLocalSelection>\nLocalVersions\PackageWarning{eqexam}
        {Selected local version greater than the number\MessageBreak
        of local versions. I'm selecting version 1\MessageBreak
        for you}\def\nLocalSelection{1}\fi
        \let\save@message\message\let\message\@gobble
        \loop
            \stepcounter{eq@count}%
            \csarg\let{Afterver\Alph{eq@count}Comment}\relax
            \lowercase
            {%
                \ifnum\value{eq@count}=\nLocalSelection
                   \eqe@turnOnComment{\Alph{eq@count}}%
                    \edef\temp@exp{\noexpand
                        \eqe@turnOffComment{\Alph{eq@count}}}%
                    \expandafter\eqe@holdTemp\expandafter{\temp@exp}%
                \else
                    \eqe@turnOffComment{\Alph{eq@count}}%
                \fi
            }%
            \ifnum\value{eq@count}<\nLocalVersions\repeat
            \let\message\save@message
    \fi
    \if@templocalversion\global\eqobeylocalversionfalse\fi
%    \end{macrocode}
% added 09/10/03 reset back to default
%    \begin{macrocode}
    \@templocalversionfalse
    \let\message\save@message
}
%    \end{macrocode}
%    \end{macro}
% \section{Title Definitions from Web}
%    \begin{macro}{\title}
%    \begin{macro}{\subject}
%    \begin{macro}{\author}
%    \begin{macro}{\email}
%    \begin{macro}{\keywords}
%    \begin{macro}{\university}
% Make Title Definitions taken from the \texttt{Web} package. This is
% to maintain compatibility with \texttt{Web}.
%    \begin{macrocode}
\@ifpackageloaded{web}{}
{
    \let\web@save@title\title
    \def\title{\@ifnextchar[{\@web@title}{\@web@title[]}}
    \def\@web@title[#1]#2{\gdef\webtitle{#2}%
        \@ifundefined{hypersetup}{}{\hypersetup{pdftitle={#2}}}%
        \def\webArg{#1}\ifx\webArg\@empty\gdef\shortwebtitle{#2}\else
        \gdef\shortwebtitle{#1}\fi\web@save@title{#2}}
    \let\web@saved@author\author
    \def\author#1{\gdef\webauthor{#1}%
        \@ifundefined{hypersetup}{}{\hypersetup{pdfauthor={#1}}}%
        \web@saved@author{#1}}
    \def\subject{\@ifnextchar[{\@subject}{\@subject[]}}
    \def\@subject[#1]#2{\def\webArg{#1}%
        \ifx\webArg\@empty\gdef\shortwebsubject{#2}\else
        \gdef\shortwebsubject{#1}\fi\gdef\websubject{#2}%
        \@ifundefined{hypersetup}{}{\hypersetup{pdfsubject={#2}}}}
    \def\email#1{\gdef\webemail{#1}}
    \def\keywords#1{\gdef\webkeywords{#1}%
        \@ifundefined{hypersetup}{}{\hypersetup{pdfkeywords={#1}}}}
    \def\university#1{\gdef\webuniversity{#1}}
    \def\web@versionlabel{Version}
    \def\web@toc{Table of Contents}
    \def\web@continued{cont.}
% set  some defaults
    \title{}\author{}\email{}\subject{}\keywords{}\university{}
    \def\optionalpagematter{}
}
%    \end{macrocode}
%    \end{macro}
%    \end{macro}
%    \end{macro}
%    \end{macro}
%    \end{macro}
%    \end{macro}
%    \begin{macro}{\date}
% {\LaTeX} (\TeX) defines a \cs{date} command that is also used by \textsf{eqexam}.
%    \begin{macrocode}
\def\duedate#1{\def\theduedate{#1}}
\duedate{}
%    \end{macrocode}
%    \end{macro}
%    \begin{macro}{\duedate}
% In addition to these, we also define a \cs{duedate} macro, may be useful for
% writing assignments with a due date.
%    \begin{macrocode}
\def\duedate#1{\def\theduedate{#1}}
\duedate{}
%    \end{macrocode}
%    \end{macro}
%    \begin{macro}{\thisterm}
% The command \cs{thisterm} can be used in the \cs{date} field to indicate the term
% of this test, for example, \verb!\date{\thisterm, \the\year}!  This command may
% be redefined to conform to your own academic terms.
%    \begin{macrocode}
\newcommand\thisterm
{%
    \ifnum\the\month<6
        Spring%
    \else
        \ifnum\the\month<8
            Summer%
        \else
            % August or later
            \ifnum\the\month>8
            % September or later
                Fall%
            \else
            % month of August
                \ifnum\the\day>25
                    Fall%
                \else
                    Summer%
                \fi
            \fi
        \fi
    \fi
}
%    \end{macrocode}
%    \end{macro}
% \section{Identification Information}
% We define a series of commands in support of building an exam: Lines to identify
% the student and his/her student id (SID), the instructors email address, the name of the
% test and the course.
%\par\medskip
%\noindent\DescribeMacro{\eqExamName}\DescribeMacro{\examNameLabel} provides a line for the student to enter his/her name into the
% exam. The command \cs{examNameLabel} can be used to define the name label, the default
% is \texttt{Name:}
%
% Will insert a text box as well if the \texttt{option} is taken in addition
% to \texttt{nosolutions} and with \texttt{solutionsafter} not taken.
% This macro defines \cs{eq@ExamName}, which actually contains the code.
% The first (optional) parameter is passed to \cs{insTxtFieldIdInfo}, and can be
% used to change the appearance of the text field created; the second required parameter
% is the width of the field.
%    \begin{macrocode}
\newcommand{\examAnsKeyLabel}[1]{%
    \def\@examAnsKeyLabel{\ifanswerkey\space #1\fi}}
\examAnsKeyLabel{Answer Key}
\newcommand\examNameLabel[1]{\gdef\@examNameLabel{#1\@examAnsKeyLabel}}
\examNameLabel{Name:}
\newcommand\eqExamName[2][]{\gdef\eq@ExamName{%
    \bgroup\settowidth\eq@tmplengthA{\@examNameLabel\ }%
         \@tempdima=#2 \advance\@tempdima by-\eq@tmplengthA
         \underbar{\makebox[#2][l]{\@examNameLabel}}%
            \insTxtFieldIdInfo[#1]{\@tempdima}{IdInfo.Name}%
    \egroup}%
}
%    \end{macrocode}
% Here we set the field to be a required field with width of $2.25$ inches
%    \begin{macrocode}
\eqExamName[\Ff\FfRequired]{2.25in}
%    \end{macrocode}
% \DescribeMacro{\eqSID} provides a line for the student to enter his/her ID number (SID).
%
%  Will insert a text box as well if the \texttt{option} is taken in addition
% to \texttt{nosolutions} and with \texttt{solutionsafter} not taken.
% The first (optional) parameter is passed to \cs{insTxtFieldIdInfo}, and can be
% used to change the appearance of the text field created; the second required parameter
% is the width of the field.
%    \begin{macrocode}
\newcommand\examSIDLabel[1]{\gdef\@examSIDLabel{#1}}
\examSIDLabel{SID:}
\newcommand\eqSID[2][]{\gdef\eq@SID{%
    \bgroup\settowidth\eq@tmplengthA{\@examSIDLabel\ }%
        \@tempdima=#2 \advance\@tempdima by-\eq@tmplengthA
        \underbar{\makebox[#2][l]{\@examSIDLabel}}%
            \insTxtFieldIdInfo[#1]{\@tempdima}{IdInfo.SID}%
    \egroup}%
}
%    \end{macrocode}
% Here we set the field to be a required field with width of $2.25$ inches
%    \begin{macrocode}
\eqSID[\Ff\FfRequired]{2.25in}
%    \end{macrocode}
% \DescribeMacro{\eqEmail} provides a line for the student
% to enter his/her email address. Useful for documents submitted by email, the
% instructor can reply.
%
%  Will insert a text box as well if the \texttt{option} is taken in addition
% to \texttt{nosolutions} and with \texttt{solutionsafter} not taken.
% The first (optional) parameter is passed to \cs{insTxtFieldIdInfo}, and can be
% used to change the appearance of the text field created; the second required parameter
% is the width of the field.
%    \begin{macrocode}
\newcommand\examEmailLabel[1]{\gdef\@examEmailLabel{#1}}
\examEmailLabel{Email:}
\newcommand\eqEmail[2][]{\gdef\eq@Email{%
    \bgroup\settowidth\eq@tmplengthA{\@examEmailLabel\ }%
        \@tempdima=#2 \advance\@tempdima by-\eq@tmplengthA
        \underbar{\makebox[#2][l]{\@examEmailLabel}}%
            \insTxtFieldIdInfo[#1]{\@tempdima}{IdInfo.email}%
    \egroup}%
}
%    \end{macrocode}
% Here we set the field to be a field with width of $2.25$ inches. (Not set to be a required field.)
%    \begin{macrocode}
\eqEmail{2.25in}
%    \end{macrocode}
% \DescribeMacro{\insTxtFieldIdInfo} The above macros (\cs{eqExamName}, \cs{eqSID} and \cs{eqEmail})
% all call this macro, which inserts a Acroforms text field if the \texttt{option} is taken in addition
% to \texttt{nosolutions} and with \texttt{solutionsafter} not taken.
%
% The first (optional) parameter is used to change the appearance of the text field.
% The second parameter is the width of the field, and the third is the field name.
%    \begin{macrocode}
\def\insTxtFieldIdInfo[#1]#2#3{%
    \@ifundefined{@quiz}{}{\ifx\eq@online y\relax
        \ifeq@nosolutions\ifeq@solutionsafter\else
            \makebox[0pt][r]{\textField[\BC{}#1]{#3}{#2}{11bp}}%
        \fi\fi\fi
    }%
}
%    \end{macrocode}
% \DescribeMacro{\SubmitInfo} is required when the \texttt{email} option is taken,
% and should appear in the preamble. The first argument is the URL to the \textsf{eqAttach.asp}
% code on the server, and the second is the email of the instructor is to receive the results.
% (Multiple recipients can be specified by separating each with a comma.)
%    \begin{macrocode}
\def\SubmitInfo#1#2{%
    \def\EqExam@SubmitURL{#1}\def\@EmailInstr{#2}%
}
%    \end{macrocode}
% \DescribeMacro{\EmailCourseName} is used to specify the course name of the course. The default
% value for this is \cs{websubject}, obtained from the \cs{subject} macro used in the preamble; however,
% if you want a different name in the email, perhaps with more information included, you can redefine
% the value using this macro.
%    \begin{macrocode}
\def\EmailCourseName#1{\def\@EmailCourseName{#1}}
%    \end{macrocode}
% Here's the default value.
%    \begin{macrocode}
\EmailCourseName{\websubject}
%    \end{macrocode}
% \DescribeMacro{\EmailExamName} is used to specify the exam name of the course. The default
% value for this is \cs{webtitle}, obtained from the \cs{title} macro used in the preamble; however,
% if you want a different name in the email, perhaps with more information included, you can redefine
% the value using this macro.
% (Multiple recipients can be specified by separating each with a comma.)
%    \begin{macrocode}
\def\EmailExamName#1{\def\@EmailExamName{#1}}
%    \end{macrocode}
% Here's the default value.
%    \begin{macrocode}
\EmailExamName{\webtitle}
%    \end{macrocode}
% \DescribeMacro{\EmailSubject} The document author mail want a custom subject in the email, instead
% of the standard one. By using this macro, he can design his own email subject.
%    \begin{macrocode}
\def\EmailSubject#1{\def\@EmailSubject{#1}}
%    \end{macrocode}
% Here's the default value, which generates no custom subject line.
%    \begin{macrocode}
\EmailSubject{}
%    \end{macrocode}
% In this case \textsf{eqAttach.asp} inserts the standard one.
%\begin{verbatim}
%Exam Results: \webtitle of \websubject
%\end{verbatim}
% The email would read like ``\texttt{Exam Results:~Test 1 of Calculus I}'', for example.
%
% \DescribeMacro{\ServerRetnMsg} Unless submitted in \texttt{silent} mode, the \textsf{eqAttach.asp}
% returns a message acknowledging the receipt of the data. \cs{ServerRetnMsg} is used to customize
% this message.
%    \begin{macrocode}
\def\ServerRetnMsg#1{\def\@ServerRetnMsg{#1}}
%    \end{macrocode}
% Here's the default value, which generates no custom return message
%    \begin{macrocode}
\ServerRetnMsg{}
%    \end{macrocode}
% \noindent\DescribeMacro{\SubmitButtonLabel} is the label that appears on the submit button.
%    \begin{macrocode}
\def\SubmitButtonLabel#1{\def\@SubmitButtonLabel{#1}}
%    \end{macrocode}
% Here's the default value.
%    \begin{macrocode}
\SubmitButtonLabel{Submit}
%    \end{macrocode}
% \DescribeMacro{\SubmitButton} is the macro that provides the submit button
% when the \texttt{email} option is taken. It appears automatically at the top of
% the first page of the exam, and appears only if \texttt{nosolutions} has has been
% taken, and \texttt{solutionsafter} has  not been taken.
%    \begin{macrocode}
\let\priorSubmitJS\@gobble
\let\postSubmitJS\@empty
\def\SubmitButton
{%
    \ifx\use@email y\ifeq@nosolutions\ifeq@solutionsafter\else
        \makebox[0pt][l]{\pushButton
            [\CA{\@SubmitButtonLabel}\A{\JS{%
                var _eqEok2Submit = true;\r
                var aSubmitFields = new Array("eqexam", "IdInfo");\r
                \priorSubmitJS\r
                if(_eqEok2Submit) this.submitForm("\EqExam@SubmitURL",
                    true, false, aSubmitFields);\r
                \postSubmitJS
                }}]{Submit}{1.5in}{16bp}}%
        \makebox[0pt][l]{\textField[\F\FHidden\DV{\@EmailInstr}
            \V{\@EmailInstr}]{IdInfo.mailTo}{11bp}{11bp}}%
        \makebox[0pt][l]{\textField[\F\FHidden\DV{\@EmailCourseName}
            \V{\@EmailCourseName}]{IdInfo.courseName}{11bp}{11bp}}%
        \makebox[0pt][l]{\textField[\F\FHidden\DV{\@EmailExamName}
            \V{\@EmailExamName}]{IdInfo.examName}{11bp}{11bp}}%
        \makebox[0pt][l]{\textField[\F\FHidden\DV{\@EmailSubject}
            \V{\@EmailSubject}]{IdInfo.subject}{11bp}{11bp}}%
        \makebox[0pt][l]{\textField[\F\FHidden\DV{\@ServerRetnMsg}
            \V{\@ServerRetnMsg}]{IdInfo.retnmsg}{11bp}{11bp}}%
    \fi\fi\fi
}
%    \end{macrocode}
%    \begin{macrocode}
\def\thequizno{\if\probstar*\Alph{quizno}\else\alph{quizno}\fi}%
\def\linkContentFormat{%
    \if\probstar*\Alph{quizno}\else\alph{quizno}\fi}
\def\linkContentWrapper{(\hfil\linkContentFormat\hfil)}%
\def\Ans@r@l@Defaults
{%
    \BC{}\S{S}\W{1}\Ff{\FfNoToggleToOff}
    \textSize{12}\textColor{0 g}
}
\def\eqExam@Ans@sq@l{%
    \leavevmode\stepcounter{quizno}\PBS\raggedright
    \settowidth{\eq@tmplength}{\eq@lw@l}\sbox{\eq@tmpbox}{\eq@l@l}%
    \eq@tmpdima=\wd\eq@tmpbox
    \def\link@@Content{\linkContentWrapper}%
    \hangindent=\eq@tmplength\hangafter=1\relax
    \edef\fieldName{%
        \if\probstar*eqexam.\curr@quiz.\theeqquestionnoi.part\thepartno
        \else
            eqexam.\curr@quiz.\theeqquestionnoi
        \fi
    }%
    \ifx\eq@online y\relax
    \makebox[0pt][l]{\radio@@Button{}{\fieldName}%
        {\eq@tmpdima}{\RadioFieldSize}{\Ans@choice\alph{quizno}}%
        {\eq@protect\A}{\eq@setWidgetProps\eq@l@check@driver}%
        {\Ans@r@l@Defaults\every@RadioButton\every@qRadioButton}}%
    \else
        \edef\@linkcolor{\@nolinkcolor}%
    \fi
    \ifeq@nosolutions\edef\@linkcolor{\@nolinkcolor}\fi
    \textcolor{\@linkcolor}{\makebox[\eq@tmpdima]{\link@@Content}}%
    \Ans@proofing{\eq@tmpdima}%
}
\def\eqExam@Ans@sq@f{%
    \stepcounter{quizno}\PBS\raggedright
    \settowidth{\eq@tmplength}{\eq@lw@f}%
    \eq@tmpdima=\wd\eq@tmpbox%
    \hangindent=\eq@tmplength\hangafter=1\relax
    \ifx\eq@online n\previewtrue
        \Ans@sq@f@driver
    \else
        \edef\fieldName{%
            \if\probstar*eqexam.\curr@quiz.\theeqquestionnoi.%
                part\thepartno
            \else
                eqexam.\curr@quiz.\theeqquestionnoi
            \fi
        }\insertGrayLetters
        \radio@@Button{}{\fieldName}{\RadioFieldSize}%
            {\RadioFieldSize}{\Ans@choice\alph{quizno}}{\eq@protect\A}%
            {\eq@setWidgetProps\eq@RadioCheck@driver}%
            {\@@Ans@sq@f@Defaults\Ans@sq@f@Actions\every@RadioButton
            \every@sqRadioButton}%
    \fi
    \Ans@proofing{\RadioFieldSize}%
}
\def\eqExam@Ans@ck@sq@l{%
    \leavevmode\stepcounter{quizno}\PBS\raggedright
    \settowidth{\eq@tmplength}{\eq@lw@l}\sbox{\eq@tmpbox}{\eq@l@l}%
    \eq@tmpdima=\wd\eq@tmpbox
    \def\link@@Content{\linkContentWrapper}%
    \hangindent=\eq@tmplength\hangafter=1\relax
    \edef\fieldName{%
        \if\probstar*eqexam.\curr@quiz.\theeqquestionnoi.%
            part\thepartno.\alph{quizno}
        \else
            eqexam.\curr@quiz.\theeqquestionnoi.\alph{quizno}
        \fi
    }%
    \ifx\eq@online y\relax
    \makebox[0pt][l]{\check@@Box{}{\fieldName}%
        {\eq@tmpdima}{\RadioFieldSize}{\Ans@choice\alph{quizno}}%
        {\eq@protect\A}{\eq@setWidgetProps\eq@l@check@driver}%
        {\Ans@r@l@Defaults\every@RadioButton\every@qRadioButton}}%
    \else
        \edef\@linkcolor{\@nolinkcolor}%
    \fi
    \ifeq@nosolutions\def\@linkcolor{\@nolinkcolor}\fi
    \textcolor{\@linkcolor}{\makebox[\eq@tmpdima]{\link@@Content}}%
    \Ans@proofing{\eq@tmpdima}%
}
\def\eqExam@Ans@ck@sq@f{%
    \stepcounter{quizno}\PBS\raggedright
    \settowidth{\eq@tmplength}{\eq@lw@f}%
    \eq@tmpdima=\wd\eq@tmpbox%
    \hangindent=\eq@tmplength\hangafter=1\relax
    \ifx\eq@online n\previewtrue
        \Ans@sq@f@driver
    \else
        \edef\fieldName{%
            \if\probstar*eqexam.\curr@quiz.\theeqquestionnoi.%
                part\thepartno.\alph{quizno}
            \else
                eqexam.\curr@quiz.\theeqquestionnoi.\alph{quizno}
            \fi
        }%
        \check@@Box{}{\fieldName}{\RadioFieldSize}%
            {\RadioFieldSize}{\Ans@choice\alph{quizno}}{\eq@protect\A}%
            {\eq@setWidgetProps\eq@RadioCheck@driver}%
            {\@@Ans@sq@f@Defaults\Ans@sq@f@Actions\every@RadioButton
            \every@sqRadioButton}%
    \fi
    \Ans@proofing{\RadioFieldSize}%
}
\def\eqExamPriorVspace#1{%
    \edef\fieldName{%
        \if\probstar*eqexam.\curr@quiz.%
            \theeqquestionnoi.part\thepartno.solution
        \else
            eqexam.\curr@quiz.\theeqquestionnoi.solution
        \fi
    }%
    \nobreak\textField[\Ff\FfMultiline\BC{}]
        {\fieldName}{\linewidth}{#1}%
    \@gobble
}
%    \end{macrocode}
%
% Test to see if \texttt{exerquiz} is loaded. If not, we input
% the `stand alone', \texttt{eqalone.def}, followed by \texttt{eqexam.def}. The latter
% definition file is maintained in \texttt{exerquiz.dtx} under the \texttt{eqexam} option.
%
%    \begin{macrocode}
\@ifpackageloaded{exerquiz}{%
    \let\Ans@sq@l\eqExam@Ans@sq@l
    \let\Ans@sq@f\eqExam@Ans@sq@f
    \let\Ans@ck@sq@l\eqExam@Ans@ck@sq@l
    \let\Ans@ck@sq@f\eqExam@Ans@ck@sq@f
    \def\eqexheader@wrapper{\makebox[0pt][r]{%
        \hypertarget{qex.\the@exno}{\eqexheader}}}
    \ifx\eq@online y\relax
        \newcounter{@cntfillin}
        \let\eqPriorVspace\eqExamPriorVspace
    \fi
}%
{%
    %%
%% This is file `eqalone.def',
%% generated with the docstrip utility.
%%
%% The original source files were:
%%
%% eqexam.dtx  (with options: `copyright,standalone')
%% 
%%%%%%%%%%%%%%%%%%%%%%%%%%%%%%%%%%%%%%%%%%%%%%%%%%%%%%%%%%
%% eqexam.sty package,             2011-08-17           %%
%% Copyright (C) 2005--2011  D. P. Story                %%
%%   dpstory@uakron.edu                                 %%
%%                                                      %%
%% This program can redistributed and/or modified under %%
%% the terms of the LaTeX Project Public License        %%
%% Distributed from CTAN archives in directory          %%
%% macros/latex/base/lppl.txt; either version 1 of the  %%
%% License, or (at your option) any later version.      %%
%%%%%%%%%%%%%%%%%%%%%%%%%%%%%%%%%%%%%%%%%%%%%%%%%%%%%%%%%%
\@ifundefined{eq@tmpbox}{\newsavebox{\eq@tmpbox}}{}% defined in eforms
\@ifundefined{eq@tmpdima}{\newdimen\eq@tmpdima}{}  % defined in eforms
\def\RadioFieldSize{11bp}
\newdimen\eqcenterWidget
\def\centerWidget
#1{%
    \eqcenterWidget=#1
    \eqcenterWidget=.5\eqcenterWidget
    \advance\eqcenterWidget by -4bp
}
\def\Bbox#1#2{\vbox{\hrule width #1
    \hbox to#1{\vrule height#2\hfill\vrule height#2}\vfill\hrule}}
\def\Rect#1{\textcolor{\@linkcolor}{#1}}
\def\ReturnTo#1#2{\eq@fititin{#2}}
\newcommand{\proofingsymbol}[1]{%
    \def\@proofingsymbol{\textcolor{\@proofingsymbolColor}{#1}}}
\proofingsymbol{$\bullet$}
\endinput
%%
%% End of file `eqalone.def'.

    %%
%% This is file `eqexam.def',
%% generated with the docstrip utility.
%%
%% The original source files were:
%%
%% exerquiz.dtx  (with options: `eqexam')
%% 
\ProvidesFile{eqexam.def}
 [2012/04/13 v6.4r %
  Exerquiz support file for eqexam (dps)]
\newif\ifeq@noforms \eq@noformsfalse
\newif\ifeq@noquizsolutions \eq@noquizsolutionsfalse % dps new
\newif\ifnocorrections \nocorrectionsfalse
\newif\ifkeepdeclaredvspacing \keepdeclaredvspacingfalse
\newif\ifeq@nolink \eq@nolinkfalse
\def\eq@ckglobalhide{\ifeq@globalshowsolutions\eq@hidesolutionfalse\fi}
\def\hidesymbol{h}\def\Hidesymbol{H}
\newif\iftherearesolutions \therearesolutionsfalse
\def\SolutionsAfter{\eq@solutionsaftertrue\eq@nolinktrue}
\def\SolutionsAtEnd{\eq@solutionsafterfalse\eq@nolinkfalse}
\def\NoSpaceToWork{\let\eq@insertverticalspace=n}
\def\SpaceToWork{\let\eq@insertverticalspace=y}
\SpaceToWork
\def\ifNoSolutions#1#2{%
    \ifeq@nosolutions\expandafter#1\else
    \expandafter#2\fi
}
\newif\ifeq@randomizeChoices \eq@randomizeChoicesfalse
\newif\ifeq@randomizeallChoices \eq@randomizeallChoicesfalse
\def\turnOnRandomize{\eq@randomizeallChoicestrue}
\def\obeyLocalRandomize{\eq@randomizeallChoicesfalse}
\let\saveRandomSeed\relax
\let\inputRandomSeed\relax
\newcommand{\exsolafter}{\eq@exsolafterDefault}
\newcommand{\resetSolnAfterToDefault}{%
    \def\exsolafter{\eq@exsolafterDefault}}
\newcommand{\exsolafterDefault}[1]{\def\eq@exsolafterDefault{#1}%
    \resetSolnAfterToDefault}
\exsolafterDefault{\textit{Solution}:}
\newcommand{\renameSolnAfterTo}[1]{\def\exsolafter{#1}}
\newcommand\eq@defaultQuizLableName{Quiz}
\ifsolutionsonly
\InputIfFileExists{\jobname_xdefs.cut}{%
    \typeout{^^JExerquiz|Eqexam: Reading \jobname_xdefs.cut^^J}}
    {\PackageWarningNoLine{exerquiz|eqexam}{%
    \jobname_xdefs.cut not found.\MessageBreak
     Recompile file under the vspacewithsolns\MessageBreak
     option, then compile with the solutionsonly\MessageBreak
     option}}
\else
\newwrite\ex@solns \immediate\openout \ex@solns \jobname.sol
\newwrite\eq@xrefdefns \immediate\openout\eq@xrefdefns\jobname_xdefs.cut
\fi
\newwrite\quiz@solns \immediate\openout \quiz@solns \jobname.qsl
\def\eq@IWAuxOut#1{\immediate\write\@auxout{#1}}
\def\eq@IWDefs#1{\immediate\write\eq@xrefdefns{#1}}
\newcounter{eqexno} \setcounter{eqexno}{0}
\newcounter{@exno} \setcounter{@exno}{0}  % running exno
\newcounter{quizno} \setcounter{quizno}{0}
\renewcommand\thequizno{\alph{quizno}}
\newcounter{eqpointvalue} \setcounter{eqpointvalue}{0}
\newlength\eq@tmplength
\newcounter{eqquestionnoi}
\newcounter{eqquestionnoii}
\newcounter{eqquestionnoiii}
\newcount\@eqquestiondepth \@eqquestiondepth=0
\newcounter{questionno}
\newcounter{partno}
\renewcommand\thepartno{\alph{partno}}
\def\verbatimwrite{\@bsphack
  \let\do\@makeother\dospecials
  \catcode`\^^M\active \catcode`\^^I=12
  \def\verbatim@processline{%
    \immediate\write\verbatim@out
      {\the\verbatim@line}}%
  \verbatim@start}
\def\endverbatimwrite{\@esphack}
\providecommand\PBS[1]{\let\temp=\\#1\let\\=\temp}
\newcommand{\eqexerskip}[1]{\def\eq@exerskip{\vskip#1}}
\eqexerskip{\medskipamount}
\newcommand{\priorexskip}[1]{\def\eq@priorexskip{\vspace{#1}}}
\priorexskip{\medskipamount}
\let\eq@postexerciseHook\relax
\newcommand\eqafterexersolnskip{\string\medskip}
\def\marginparpriorhook{} % used to material before the exercise
\def\marginparafterhook{} % used to material after the exercise
\let\afterlabelhskip\space
\let\exersolnheadhook\@empty
\let\exer@solnheadhook\@empty
\newcommand{\eqexheader}
{%
        \hbox{\color{\@nolinkcolor}\if\exerstar*\exlabelformatwp\else
        \exlabelformat\fi}%
}
\def\eqexheader@wrapper{\hypertarget{qex.\the@exno}{\eqexheader}}
\newcommand{\partsformat}[1]{\def\eq@partsformat{#1}}
\partsformat{(\hfil\alph{partno}\hfil)}
\newcommand{\defaultpartsformat}{%
    \partsformat{(\hfil\alph{partno}\hfil)}}
\newcommand{\eqexlisttabheader}
{%
      \color{\@nolinkcolor}\eq@partsformat
}
\newcommand{\eq@CommonCmd}[1]{\def\eq@@CommonCmd{#1}#1}
\let\eq@@CommonCmd\@empty
\newenvironment{exercise}
{%
    \let\eqCommonCmd\eq@CommonCmd
    \par\eq@priorexskip\noindent%\begingroup % set defaults
    \def\eq@argi{eqexno}%  % use eqexno counter
    \if\eq@exerstarEnv*\def\exerstar{*}\else\def\exerstar{x}\fi
    \def\currhideopt{x}%
    \@ifnextchar[{\exercise@}{%
    \if\exerstar*%
        \def\eq@next{\@exercise}%
    \else
        \def\eq@next{\exercise@@}%
    \fi\eq@next}%
}{\eq@postexerciseHook\par\global\eq@exerciseheadingtrue\eq@exerskip}
\def\exercise@[#1]{\edef\eq@arg{#1}%
    \if\eq@arg h%
        \def\currhideopt{h}%
        \eq@hidesolutiontrue\eq@nolinktrue%
        \ifeq@globalshowsolutions
            \eq@hidesolutionfalse\eq@nolinkfalse
        \fi
        \def\eq@next{\@exercise}%   h, no *, no counter
    \else
        \if\eq@arg H%
            \edef\currhideopt{\Hidesymbol}%
            \eq@hidesolutiontrue\eq@nolinktrue%
            \ifeq@globalshowsolutions
                \eq@hidesolutionfalse\eq@nolinkfalse
            \fi
            \def\eq@next{\@exercise}%   h, no *, no counter
        \else
            \def\currhideopt{x}%
            \if\exerstar*%
                \def\eq@next{\def\eq@argi{#1}\@exercise}%
            \else
                \def\eq@next{\def\eq@argi{#1}\exercise@@}%
            \fi
        \fi
    \fi
\eq@next}
\def\exercise@@{\@ifstar{\def\exerstar{*}\@exercise}% was \exercise@@@
   {\exercise@@@}}%
\def\exercise@@@{%
\@ifnextchar[{\exercise@@@@}{\@exercise}}
\def\exercise@@@@[#1]{\def\eq@arg{#1}%
    \if\eq@arg h%
        \def\currhideopt{h}%
        \eq@hidesolutiontrue\eq@nolinktrue
        \ifeq@globalshowsolutions
            \eq@hidesolutionfalse\eq@nolinkfalse
        \fi
        \def\eq@next{\@exercise}%   h, no *, no counter
    \else
        \if\eq@arg H%
            \edef\currhideopt{\Hidesymbol}%
            \eq@hidesolutiontrue\eq@nolinktrue%
            \def\eq@next{\@exercise}%   h, no *, no counter
        \else
            \def\currhideopt{x}%
            \typeout{The option #1 is not recognized}
            \let\eq@next=\relax
        \fi
    \fi
\eq@next}
\newif\ifeq@exerciseheading \eq@exerciseheadingtrue
\newcommand\exerSolnHeader[3]
{%
    \ifeqforpaper\else\webnewpage\fi\markright{#1}\par\noindent%
    #3%
    \solnhspace
}
\let\eqEXt\@gobble
\let\endeqEXt\relax
\let\eq@writeexheader\@empty
\let\eqExtArg\@empty
\def\@exercise
{%
    \let\verbatim@out=\ex@solns
    \if\eq@argi0\else\refstepcounter{\eq@argi}\fi\stepcounter{@exno}%
    \if\exerstar*% if exercise with parts
        \eq@nolinktrue
    \else
        \if\currhideopt H%
        \else
            \ifeq@solutionsafter
                \eq@nolinktrue  % no link to solution
            \else
                \eq@ckglobalhide %
                \ifeq@hidesolution\else
                   \gdef\eq@writeexheader{\set@display@protect
                     \immediate\write\verbatim@out{%
                     \protect\eqEXt{\eqExtArg}\protect
                     \exerSolnHeader{\exsecrunhead}{ex.\the@exno}%
                     {\exsllabelformat}\exer@solnheadhook
                     \exersolnheadhook\protect\relax\space}%
                     \set@typeset@protect}%
                \fi
            \fi
        \fi
    \fi
    \ifvmode\ifdim\lastskip>\z@
        \vskip-\lastskip
    \fi\fi
    \if\exerstar*%
        \let\solution=\solnexer@woparts
        \let\endsolution=\endsolnexer@woparts
        \let\parts=\exercise@parts
        \let\endparts\endexercise@parts
    \else
        \let\solution=\solnexer@woparts
        \let\endsolution=\endsolnexer@woparts
        \let\parts=\relax
        \let\endparts=\relax
        \if\Hidesymbol h\eq@nolinkfalse\ifeq@solutionsafter
            \eq@nolinktrue\fi\fi
    \fi
    \ifeq@exerciseheading
    \prior@questionsHook\marginparpriorhook\noindent\eqexheader@wrapper
    \afterlabelhskip\marginparafterhook\ignorespaces\fi
}
\long\def\setsolnspace#1{%
    \def\newsolnspace{#1}%
    \let\solnhspace\newsolnspace
}
\let\solnhspace\space
\def\solnexer@woparts{\def\bLeaveVspace{x}% = no vertical space added
    \def\eq@next{\@ifnextchar[{\solnexer@@woparts}%
        {\solnexer@@woparts[\null]}}%
    \ifx\endparts\endexercise@parts@tabular
        \def\eq@next{\solnexer@@@woparts}%
    \fi
    \eq@next}
\let\eqPriorVspace\@gobble
\newcommand{\vspaceFiller}[1]{\vfill}
\newcommand{\vspaceFillerDefault}[1]{\vfill}
\newcommand{\vspaceFmt}[1]{%
    \ifdim\sameVspace>0pt
        \let\bLeaveVspace\@empty
        \def\leavevspace{%
            \ifx\eq@insertverticalspace y\par\eqPriorVspace{#1}%
                {\nobreak\noindent
                 \parbox[b][#1][t]{\linewidth}{\vspaceFiller{#1}}}%
            \fi
        }%
    \fi
}
\def\solnexer@@woparts[#1]{%
    \ifx#1\null\gdef\sameVspace{0pt}%
    \else
        \xdef\sameVspace{#1}%
        \ifx\sameVspace\@empty\gdef\sameVspace{0pt}\fi
        \ifvspacewithsolns\vspaceFmt{#1}\fi
        \ifeq@nosolutions\ifeq@solutionsafter\else
            \vspaceFmt{#1}\fi\fi
    \fi
    \solnexer@@@woparts
}
\def\solutionsafterSkip{\vskip\smallskipamount}
\def\eqe@debugVertSkip#1{}
\def\eqe@showEndHere#1{#1}
\def\eqe@setStartSolns{%
    \xdef\eq@startSoln{\the\pagetotal}%
    \eqe@debugVertSkip{\marginpar{\smash{b[\sameVspace]}}}%
}
\let\eqSavedComment\comment
\let\eqSavedEndCommet\endcomment
\def\solnexer@@@woparts
{%
    \def\exerwparts@cols{x}%
    \let\verbatim@out\ex@solns
    \if\currhideopt H%
        \let\procsoln\eqSavedComment
        \let\endprocsoln\eqSavedEndCommet
        \def\eq@next{\procsoln}%
    \else
        \eq@ckglobalhide
        \ifeq@hidesolution
            \let\procsoln\eqSavedComment
            \let\endprocsoln\eqSavedEndCommet
        \else
            \ifx\bLeaveVspace\@empty\leavevspace\fi
\set@display@protect
\immediate\write\verbatim@out{\eq@@CommonCmd}%
\set@typeset@protect
            \let\procsoln\verbatimwrite
            \let\endprocsoln\endverbatimwrite
        \fi
        \def\eq@next{%
            \ifeq@solutionsafter%\par
                \ifx\exsolafter\@empty
                    \ifdim\sameVspace=0pt\eqe@setStartSolns
                    \else
                        \solutionsafterSkip
                        \eqe@setStartSolns
                    \fi
                \else\par\kern0pt
                    \solutionsafterSkip
                    \noindent\strut\eqe@setStartSolns
                        \exsolafter\space\ignorespaces
                \fi
            \else
                \eq@writeexheader
                \global\let\eq@writeexheader\@empty
                \global\therearesolutionstrue\expandafter\procsoln
            \fi
        }%
    \fi
\eq@next}
\def\endsolnexerhook{}
\def\endsolnexerhookaux{}
\def\eq@fititin#1{\noindent\unskip\nobreak\hfill\penalty50
    \hskip2em\hbox{}\nobreak\hfill#1}
\let\eqfititin\eq@fititin
\newcommand\eqExerSolnTrailer
{%
    \string\ReturnTo{page.\the\c@page}%
    {\hbox{\if\exerstar*\exrtnlabelformatwp\else\exrtnlabelformat\fi}}%
    \string\endeqEXt\ifeqforpaper\string\par\eqafterexersolnskip\fi^^J
}
\def\endsolnexer@woparts
{%
    \if\currhideopt H%
        \ifkeepdeclaredvspacing\vskip\sameVspace\fi
        \csname endprocsoln\endcsname
    \else
        \ifeq@solutionsafter
            \eq@fititin{\hbox{\exrtnlabelformat}}\par\kern0pt
            \ifkeepdeclaredvspacing
                \@tempdima\pagetotal \advance\@tempdima-\eq@startSoln
                \@tempdimb\sameVspace
                \ifdim\@tempdimb>0pt
                    \advance\@tempdimb-\@tempdima
                    \ifdim\@tempdimb>0pt\vglue\@tempdimb\kern0pt
                        \let\eqe@showEndHere\@gobble
                        \edef\tmp@exp{\noexpand\marginpar{%
                        \noexpand\smash{e: adj \the\@tempdimb}}}%
                        \eqe@debugVertSkip{\tmp@exp}%
                    \fi
                \fi
            \fi
            \eqe@showEndHere{\eqe@debugVertSkip{\marginpar{\smash{e}}}}%
        \else
            \endprocsoln
            \eq@ckglobalhide
            \ifeq@hidesolution\else
                \endsolnexerhookaux
                \set@display@protect
                \immediate\write\verbatim@out{\eqExerSolnTrailer}%
                \set@typeset@protect
            \fi
        \fi
    \fi
\endsolnexerhook}
\let\eq@exerstarEnv\relax
\newenvironment{exercise*}{\def\eq@exerstarEnv{*}\exercise}
{\endexercise}
\def\exercise@parts{\@@par\ifdim\parskip>\z@\vskip-\parskip\fi
  \def\exerwparts@cols{x}\@ifnextchar[%
  {\let\endparts=\endexercise@parts@tabular\exercise@parts@tabular@}%
  {\let\endparts=\endexercise@parts@list\exercise@parts@list}}
\def\eq@extralabelsep{0pt}
\newcommand{\setPartsWidth}[1]{\def\parts@indent{\normalfont#1}}
\setPartsWidth{(d)}
\def\eqe@prtsepPrb{\ }
\let\prior@parts@hook\@empty
\let\post@parts@hook\@empty
\let\eqp@rtc@lcm@rk\relax
\let\eq@insertContAnnot\relax
\def\eq@item@common{\eq@insertContAnnot\eqp@rtc@lcm@rk
    \def\currhideopt{x}\eq@hidesolutionfalse\eq@nolinkfalse
    \@ifnextchar[\@ckhide\eq@item}
\newcommand{\eqpartsitemsep}[1]{\def\eqparts@itemsep{#1}}
\eqpartsitemsep{0pt}
\newenvironment{exercise@parts@list}
{\settowidth{\eq@tmplength}{\parts@indent}%
\eq@initializeContAnnot\eq@nolinkfalse\prior@parts@hook
\list{\normalfont
    \if\Hidesymbol h\eq@nolinkfalse\ifeq@solutionsafter
        \eq@nolinktrue\fi\fi
    \if\currhideopt H%
    \else
        \ifeq@solutionsafter
            \eq@nolinktrue  % no link to solution
        \else
            \ifeq@nosolutions
                \eq@nolinktrue  % no link to solution
            \else
                \eq@ckglobalhide
                \ifeq@hidesolution\eq@nolinktrue\else
                    \gdef\eq@writeexheader{\set@display@protect
                    \immediate\write\verbatim@out{%
                    \protect\eqEXt{\eqExtArg}\protect
                    \exerSolnHeader{\exsecrunhead}%
                    {ex.\the@exno\alph{partno}}{\exsllabelformatwp}%
                    \exer@solnheadhook\exersolnheadhook\relax}%
                    \set@typeset@protect}%
                \fi
            \fi
        \fi
    \fi
    \makebox[\eq@tmplength]{\eqexlisttabheader}%
}{%
    \usecounter{partno}%
    \setlength{\topsep}{3pt}%
    \setlength{\partopsep}{0pt plus 1pt minus 1pt}%
    \setlength{\parsep}{0pt}\setlength{\itemindent}{0pt}%
    \setlength{\listparindent}{\parindent}%
    \setlength{\itemsep}{\eqparts@itemsep}
    \settowidth{\labelsep}{\normalfont\eqe@prtsepPrb}%
    \addtolength{\labelsep}{\eq@extralabelsep}%
    \settowidth{\labelwidth}{\parts@indent}%
    \setlength{\leftmargin}{\labelwidth}%
    \addtolength{\leftmargin}{\labelsep}%
    \let\eq@item\item
    \def\eqthisenv{parts}%
    \def\item{\ifx\@currenvir\eqthisenv
        \def\eq@next{\eq@item@common}\else
        \def\eq@next{\eq@item}\fi
    \eq@next}%
}}{\endlist\post@parts@hook}
\def\eq@initializeContAnnot{\@ifundefined{eqequestions}{%
    \global\let\eqeCurrProb\relax}{%
    \xdef\eqeCurrProb{\theeqquestionnoi}}%
    \xdef\eq@currProbStartPage{\arabic{page}}%
}
\def\exercise@parts@tabular@[#1]{%
  \def\exerwparts@cols{#1}\exercise@parts@tabular}
\newenvironment{exercise@parts@tabular}
{%
    \setcounter{partno}{0}%
    \settowidth{\eq@tmplength}{\parts@indent\eqe@prtsepPrb}%
    \sbox{\eq@tmpbox}{\parts@indent}%
    \let\eq@item@saved\item
    \let\eq@item\item@part@tabular
    \def\eqthisenv{parts}%
    \def\item{\ifx\@currenvir\eqthisenv
        \def\eq@next{\eq@item@common}\else
        \def\eq@next{\eq@item@saved}\fi
    \eq@next}\eq@nolinkfalse
    \eq@tmpdima=\linewidth \divide\eq@tmpdima by\exerwparts@cols
    \vskip\partopsep\noindent\normalbaselines\tabcolsep=0pt
    \tabular{*{\exerwparts@cols}{p{\eq@tmpdima}}}%
}{\endtabular}
\def\item@part@tabular{\leavevmode\refstepcounter{partno}%
  \eq@solutionsafterfalse % no solutionsafter are allowed
  \ifeq@solutionsafter
    \eq@nolinktrue  % no link to solution
  \else\ifeq@nosolutions
    \eq@nolinktrue  % no link to solution
  \else
      \eq@ckglobalhide
      \ifeq@hidesolution\eq@nolinktrue\else
      \gdef\eq@writeexheader{%
        \set@display@protect
        \immediate\write\verbatim@out{%
        \protect\eqEXt{\eqExtArg}\string\exerSolnHeader{\exsecrunhead}%
            {ex.\the@exno\alph{partno}}{\exsllabelformatwp}\relax}%
        \set@typeset@protect
      }%
  \fi\fi\fi
  \PBS\raggedright
  \settowidth{\eq@tmplength}{\parts@indent\eqe@prtsepPrb}%
  \sbox{\eq@tmpbox}{\parts@indent}%
  \eq@tmpdima=\wd\eq@tmpbox
  \addtolength\eq@tmplength{\eq@extralabelsep}%
  \hangindent=\eq@tmplength\hangafter=1\relax
  \makebox[\eq@tmpdima]{\eqexlisttabheader}\eqe@prtsepPrb
  \ignorespaces
}
\def\@ckhide[#1]{\edef\eq@arg{#1}%
    \def\currhideopt{x}%
    \ifx\eq@arg\@empty\else
    \if\eq@arg H%
        \eq@hidesolutiontrue\eq@nolinktrue%
        \edef\currhideopt{\Hidesymbol}%
    \else
        \ifeq@globalshowsolutions\else
            \if\eq@arg h%
                \eq@hidesolutiontrue\eq@nolinktrue%
                \def\currhideopt{h}%
            \fi
        \fi
    \fi\fi
    \eq@item
}
\def\includeexersolutions{%
       \include@solutions\global\let\include@solutions\relax
}
\let\eqsolutionshook\@empty
\let\priorexsectitle\@empty
\let\priorexslinput\@empty
\def\exerSolnsHeadnToc{\section*{\exsectitle}%
    \addcontentsline{toc}{section}{%
    \@ifundefined{web@latextoc}{}{%
    \ifx\web@latextoc y\else
    \protect\numberline{}\fi}\exsectitle}}
\@ifpackageloaded{web}{\def\eq@normallheader{\lheader{\rightmark}}}
    {\let\eq@normallheader\relax}
\@ifpackageloaded{web}{\def\eq@defaultlheader{\lheader{\aeb@setmarks}}}
    {\let\eq@defaultlheader\relax}
\newcommand{\exerSolnInput}
{%
    \let\webnewpage\relax
    \ifsolutionsonly\else\immediate\closeout\ex@solns\fi
    \ifeq@nosolutions\else
        \iftherearesolutions
            \ifsolutionsonly\eqsolutionshook
        \else
            \newpage\eqsolutionshook\markright{}\eq@normallheader
        \fi
        \markright{\exsectitle}%
            \ifx\webnewpage\relax
                \def\webnewpage{\let\webnewpage\newpage}%
            \fi
            \priorexsectitle\exerSolnsHeadnToc\priorexslinput
            \InputIfFileExists{\jobname.sol}{}{\PackageWarning{exerquiz}
            {!!! Solutions to exercises not found}}%
            \newpage\eq@defaultlheader
        \fi
    \fi
}
\def\include@solutions{%
        \exerSolnInput
}
\newcommand{\prior@questionsHook}{}
\let\qMark@Hook\@empty
\let\aeb@titleQuiz\@empty
\newcommand{\post@questionsHook}{}
\renewcommand{\theeqquestionnoi}{\arabic{eqquestionnoi}}
\newcommand{\labeleqquestionnoi}{%
    \color{blue}\bfseries\theeqquestionnoi.}
\renewcommand{\theeqquestionnoii}{\alph{eqquestionnoii}}
\newcommand{\labeleqquestionnoii}{%
    \color{blue}\bfseries(\theeqquestionnoii)}
\renewcommand{\theeqquestionnoiii}{\roman{eqquestionnoiii}}
\newcommand{\labeleqquestionnoiii}{%
    \color{blue}\bfseries(\theeqquestionnoiii)}
\newenvironment{questions}
{%
    \ifnum\@eqquestiondepth>\tw@\@toodeep\else
    \advance\@eqquestiondepth\@ne\fi
    \def\@quesctr{eqquestionno\romannumeral\the\@eqquestiondepth}%
    \list{\qMark@Hook\prior@questionsHook\gdef\eqPTs{1}%
    \global\let\eqQT\eq@na%
    {\@tempcnta=0 \let\@thispr@b\@empty
     \@whilenum\@tempcnta<\@eqquestiondepth\do{\advance\@tempcnta\@ne
     \ifx\@thispr@b\@empty\edef\@thispr@b{%
     \csname theeqquestionno\romannumeral\the\@tempcnta\endcsname}\else
     \edef\@thispr@b{\@thispr@b%
     (\csname theeqquestionno\romannumeral\the\@tempcnta\endcsname)}\fi
    }\xdef\@currentQues{\@thispr@b}}%
    \makebox[\labelwidth][r]{\normalfont\bfseries
    \csname label\@quesctr\endcsname}\xdef\eq@pageThisQ{\the\c@page}%
    \post@questionsHook}{\usecounter{\@quesctr}%
    \settowidth{\labelwidth}{\normalfont\bfseries00.\ }%
    \setlength{\topsep}{3pt}\setlength{\parsep}{0pt}%
    \setlength{\itemindent}{0pt}\setlength{\itemsep}{3pt}%
    \setlength{\leftmargin}{\labelwidth}%
    \settowidth{\labelsep}{\ }}%
}{\endlist}
\def\pushquestions{%
    \expandafter\xdef\csname save\@quesctr\endcsname{%
        \expandafter\the\csname c@\@quesctr\endcsname}%
    \end{questions}}
\def\popquestions{%
    \begin{questions}
    \setcounter{\@quesctr}{\csname save\@quesctr\endcsname}}
\def\sq@priorhook{\medskip\noindent}
\def\sq@afterhook{}
\def\@shortquizCnt{0}
\def\@sqGenBaseName{eqSqBn\@shortquizCnt}
\newenvironment{shortquiz}
{%
    \xdef\eq@pageThisQ{\the\c@page}%
    \let\@currentQues\@empty
    {\count0=\@shortquizCnt \advance\count0by1\relax
        \xdef\@shortquizCnt{\the\count0}}%
    \goodbreak\@ifstar{\sqForms\@shortquiz}%
        {\sqLinks\@shortquiz}%
}{\aeb@endshortquiz}
\newenvironment{shortquiz*}
{%
    \xdef\eq@pageThisQ{\the\c@page}%
    {\count0=\@shortquizCnt\advance\count0by1\relax
        \xdef\@shortquizCnt{\the\count0 }}%
    \sqForms\@shortquiz
}{\aeb@endshortquiz}
\def\@shortquiz{\@ifnextchar[%
    {\@@shortquiz}{\@@shortquiz[\@sqGenBaseName]}}
\def\@@shortquiz[#1]{%\begingroup
  \gdef\oField{#1}\gdef\curr@quiz{#1}\gdef\currQuiz{#1}%
    \global\let\eqQzQuesList\@empty
    \let\eq@AddProbToQzQuesList\relax
    \edef\@currentlabel{\@shortquizCnt}%
    \edef\@currentHref{shortquiz.\@shortquizCnt}%
    \ifx\aeb@@titleQuiz\@empty
        \def\@currentlabelname{\eq@defaultQuizLableName}\else
        \protected@edef\@currentlabelname{\@currentlabelname}\fi
  \global\let\eqQuizType=s\let\@qzsolndest\@empty
  \if\sqstar*\relax
    \let\@Ans\Ans@sq@f
    \ifx\oField\@empty
        \typeout{^^JExerquiz: Base field name required when using
                shortquiz with '*' option}
        \typeout{Exerquiz: Assuming link style^^J}
        \let\@Ans\Ans@sq@l
    \fi
  \else
     \let\@Ans\Ans@sq@l
  \fi
  \setcounter{questionno}{0}%
  \let\answers\answers@sq
  \let\endanswers\endanswers@sq
  \let\manswers\manswers@sq
  \let\endmanswers\endmanswers@sq
  \let\solution\solution@sq
  \let\endsolution\endsolution@sq
}
\def\aftershortquizskip{\medskip}
\def\aeb@endshortquiz{\setcounter{quizno}{0}%
    \sq@afterhook
    \global\let\sqlabel\eq@sqlabel
    \global\let\sqslrtnlabel\eq@sqslrtnlabel
    \global\let\sqsllabel\eq@sqsllabel
    \par\aftershortquizskip
}
\def\sqLinks{\def\sqstar{}}\sqLinks
\def\sqForms{\def\sqstar{*}}
\let\eq@tq@star\relax
\def\sqPriorSolutionAfterSkip{\smallskip}
\def\solution@sq{\let\eq@next\relax
  \ifx\@qzsolndest\@empty
    \typeout{exerquiz: * Solutions unexpected here *}%
    \typeout{exerquiz: * Will assume 'solutionsafter' option *}%
    \eq@solutionsaftertrue
  \fi
  \ifeq@solutionsafter
    \par\sqPriorSolutionAfterSkip\noindent
    \sqsolafter
  \else
    \global\therearequizsolutionstrue\let\verbatim@out=\quiz@solns
    \set@display@protect
    \immediate\write\verbatim@out{%
      \ifx\eqQuizType q\string\eqQt\else\string\eqSQt\fi%
      \string\quizSolnHeader\ifx\eqQuizType q\ifx\allow@peek n%
        [{\curr@quiz}{\currQuizStartPage}]\fi\fi%
        {\@qzsolndest}{\sqsllabel}\relax}%
    \set@typeset@protect
    \expandafter\verbatimwrite\fi
}
\def\sqSolutionsAfterSkip{\par\bigskip}
\def\endsolution@sq
{%
    \ifeq@solutionsafter
        \eq@fititin{\hbox{\sqslrtnlabel}}\sqSolutionsAfterSkip
    \else
        \endverbatimwrite
        \ifx\@qzsolndest\@empty\else\set@display@protect
            \immediate\write\verbatim@out{\eqSqSolnTrailer}%
         \set@typeset@protect
        \fi
    \fi
    \global\let\@qzsolndest\@empty
}
\def\fpAfterSolutionsSkip{\par\medskip}
\newcommand\eqSqSolnTrailer{%
    \ifx\eqQuizType q%
        \string\ReturnTo{page.\eq@pageThisQ}%
            {\hbox{\sqslrtnlabel}}\string\endeqQt
    \else
        \string\ReturnTo{page.\eq@pageThisQ}%
            {\hbox{\sqslrtnlabel}}\string\endeqSQt
    \fi
    \ifeqforpaper\string\fpAfterSolutionsSkip\fi^^J%
}
\let\eqSQt\relax
\let\endeqSQt\relax
\let\eqQt\relax
\let\endeqQt\relax
\newcommand\quizSolnHeader[3][]{%
    \ifeqforpaper\else\webnewpage\fi\noindent
    #2%
    \solnhspace
}
\newif\iftherearequizsolutions \therearequizsolutionsfalse
\let\aeb@FLOverride\relax
\def\includequizsolutions
{%
        \include@quizsolutions
        \let\include@quizsolutions\relax
}
\let\aeb@titleQuiz\@empty
\let\aeb@@titleQuiz\@empty
\def\answers@sq{\stepcounter{questionno}%
    \if\sqstar*\relax
        \if\aeb@FLOverride l
            \let\@Ans\Ans@sq@l\else
            \let\@Ans\Ans@sq@f\fi
    \else
        \if\aeb@FLOverride f
            \let\@Ans\Ans@sq@f\else
            \let\@Ans\Ans@sq@l\fi
    \fi
    \def\aeb@answerType{r}\@ifnextchar[{\answers@@sq}%
    {\@ifstar{\answers@@sq[\curr@quiz.\thequestionno]}{\answers@@sq[]}}}
\def\manswers@sq{\stepcounter{questionno}%
    \if\sqstar*\relax
        \if\aeb@FLOverride l\let\@An\Ans@ck@sq@f\fi
    \else
        \if\aeb@FLOverride f
            \let\@An\Ans@ck@sq@f\else
            \let\@Ans\Ans@ck@sq@l\fi
    \fi
    \def\aeb@answerType{c}\@ifnextchar[{\answers@@sq}%
    {\@ifstar{\answers@@sq[\curr@quiz.\thequestionno]}{\answers@@sq[]}}}
\let\sq@hwdest\@empty % hard-wired destination
\def\answers@@sq[#1]#2{%
    \gdef\aeb@numCols{#2}%
    \ifx\sq@hwdest\@empty
        \xdef\@qzsolndest{#1}\else
        \gdef\@qzsolndest{\sq@hwdest}\fi
    \ifx#21\gdef\eq@listType{1}\expandafter\answers@sq@list\else
    \gdef\eq@listType{0}\expandafter\answers@@sq@tabular\fi{#2}%
}
\def\eq@hspanner{\ }
\def\eq@hspanner@default{\ }
\def\eq@lw@l{\eq@l@l\eq@hspanner}
\def\setMClabelsep#1{\def\eq@hspanner{#1}}
\def\resetMClabelsep{\let\eq@hspanner\eq@hspanner@default}
\def\eq@l@l{\normalsize\normalfont(d)}
\def\eq@lw@f{\kern\RadioFieldSize\eq@hspanner}
\newcommand\Ans@list[2][0]{\gdef\eq@pPTs{#1}%
    \xdef\Ans@choice{#2}\item\relax\if\eq@listType1%
    \addtocounter{quizno}{-1}\refstepcounter{quizno}\fi}
\newenvironment{answers@sq@list}[1]
{%
    \if\aeb@answerType r
        \let\endanswers\endanswers@sq@list\else
        \let\endmanswers\endanswers@sq@list\fi
    \ifinner\else\vskip\aboveanswersSkip\fi
    \list{\strut\@Ans\eq@hspanner}%
    {%
        \if\sqstar*\relax
            \settowidth{\labelwidth}{\eq@lw@f\eq@hspanner}\else
            \settowidth{\labelwidth}{\eq@lw@l\eq@hspanner}\fi
        \setlength{\parsep}{0pt}\setlength{\itemindent}{0pt}%
        \setlength{\topsep}{0pt}\setlength{\partopsep}{0pt}%
        \setlength{\listparindent}{\parindent}%
        \setlength{\leftmargin}{\labelwidth}%
        \setlength{\labelsep}{0pt}%
        \def\Ans{\Ans@list}%
    }%
}%
{\endlist\setcounter{quizno}{0}}
\newcommand\Ans@tabular[2][0]{\gdef\eq@pPTs{#1}%
    \xdef\Ans@choice{#2}\leavevmode\@Ans
}
\def\answers@@sq@tabular#1{%
    \vskip\aboveanswersSkip\noindent\tabcolsep=0pt
    \eq@tmpdima=\linewidth \divide \eq@tmpdima by#1 %
    \def\Ans{\Ans@tabular}%
    \begin{tabular}{*{#1}{p{\eq@tmpdima}}}}%
\def\endanswers@sq{\end{tabular}\setcounter{quizno}{0}}%
\def\endmanswers@sq{\end{tabular}\setcounter{quizno}{0}}%
\def\popiiictm{\special{CTM: pop pop pop}}
\def\linkContentFormat{%
    \if\probstar*\Alph{quizno}\else\alph{quizno}\fi}
\def\linkContentWrapper{(\hfil\linkContentFormat\hfil)}%
\def\Ans@sq@l{%
    \leavevmode\if\eq@listType1\stepcounter{quizno}%
    \else\refstepcounter{quizno}\fi\PBS\raggedright
    \settowidth{\eq@tmplength}{\eq@lw@l}\sbox{\eq@tmpbox}{\eq@l@l}%
    \eq@tmpdima=\wd\eq@tmpbox
    \def\link@@Content{\linkContentWrapper}%
    \hangindent=\eq@tmplength\hangafter=1\relax
    \Ans@sq@l@driver
\eq@hspanner\ignorespaces}
\let\Ans@ck@sq@l=\Ans@sq@l
\def\Ans@sq@f{%
    \if\eq@listType1\stepcounter{quizno}%
    \else\refstepcounter{quizno}\fi\PBS\raggedright
    \settowidth{\eq@tmplength}{\eq@lw@f}%
    \eq@tmpdima=\wd\eq@tmpbox%
    \hangindent=\eq@tmplength\hangafter=1\relax
    \insertGrayLetters % 6.3d
    \Ans@sq@f@driver
\eq@hspanner\ignorespaces}
\let\Ans@ck@sq@f=\Ans@sq@f
\newskip\aboveanswersSkip
\setlength\aboveanswersSkip{3pt}
\def\insertGrayLetters{\ifaebshowgrayletters
    \rlap{\makebox[\RadioFieldSize]%
        {\textcolor{gray}{\Alph{quizno}}}}\else\relax\fi}
\newcount\eq@tabColCnt
\define@key{bchoice}{nCols}{\def\bChoiceNumCols{#1}}
\@for\eqi:=1,2,3,4,5,6,7,8,9,10 \do{\edef\temp@expand@def{%
    \noexpand\define@key{bchoice}{\eqi}[\eqi]{%
        \noexpand\def\noexpand\bChoiceNumCols{\eqi}}%
    }\temp@expand@def
}
\def\bChoiceNumCols{\aeb@numCols}
\define@key{bchoice}{random}[true]{%
    \csname if#1\endcsname\eq@randomizeChoicestrue
        \else\eq@randomizeChoicesfalse\fi
}
\define@key{bchoice}{label}[]{%
    \xdef\bChoiceLabel{#1}%
}
\def\bChoices{\@ifnextchar[{\@ansChoices}{\@ansChoices[\aeb@numCols]}}
\def\@ansChoices[#1]%
{%
    \global\let\@tempholdSaveAns\@empty %4/5
    \global\let\@tempholdSaveChoice\@empty %4/5
    \global\let\bChoiceLabel\@empty %4/5
    \setkeys{bchoice}{#1}%
    \global\eq@tabColCnt=0
    \ifnum\aeb@numCols=1 % list mode
        \def\eq@next{\@layoutListAns}%
    \else % tabular mode
         \def\eq@next{\@layoutTabularAns{\bChoiceNumCols}}%
    \fi
    \eq@next
}
\let\eChoices\relax
\def\@layoutListAns{\@ifnextchar\Ans{\@getListAns}%
    {\@lookforendansChoices{\@layoutListAns}}%
}
\newcommand{\eq@saveAns}[2][]{%
    \let\eq@next\@@@SaveAnsGobbleAns
    \ifx\bChoiceLabel\@empty\else
    \def\eq@savedAnsOpt{#1}\def\eq@savedAnsZO{#2}%
    \if\eq@savedAnsZO1 \let\eq@next\@@@SaveAns\fi
    \fi\eq@next
}
\long\def\@@@SaveAns#1\eAns{% 4/5
    \g@addto@macro\@tempholdSaveAns{\\{\ignorespaces#1}}%
    \addtocounter{quizno}{1}%
    \edef\temp@expand{\noexpand\g@addto@macro\noexpand
    \@tempholdSaveChoice{%
        \noexpand\\{\ifx\sqstar\@empty\linkContentFormat\else
        \ifaebshowgrayletters\Alph{quizno}\else
        \thequizno\fi\fi}}}\temp@expand
    \addtocounter{quizno}{-1}%
}
\def\eq@insertComma{\ifx\eq@comma\@empty\def\eq@comma{,}\else
    \eq@comma\space\fi}
\long\def\eq@displayAns#1{\eq@insertComma#1} % 4/5
\def\eq@displayAlts#1{\eq@insertComma(#1)} % 4/5
\newcommand{\useSavedAns}[2][]{% 4/5
    \bgroup
        \def\eq@argi{#1}\ifx\eq@argi\@empty
            \let\eq@comma\@empty\let\label\@gobble
            \let\\\eq@displayAns\csname SavedAns#2\endcsname\else
            \csname SavedAns#2-Idx#1\endcsname\fi
    \egroup
} % 4/5
\newcommand{\useSavedAlts}[2][]{%
    \bgroup
        \def\eq@argi{#1}\ifx\eq@argi\@empty
            \let\eq@comma\@empty
            \let\\\eq@displayAlts\csname SavedAlts#2\endcsname\else
            (\csname SavedAlts#2-Idx#1\endcsname)\fi
    \egroup
} % 4/5
\newcommand{\useSavedAltsAns}[2][]{%
    \bgroup
        \def\eq@argi{#1}\ifx\eq@argi\@empty
            \csname SavedAltsAns#2\endcsname\else
            (\csname SavedAlts#2-Idx#1\endcsname)
            \csname SavedAns#2-Idx#1\endcsname\fi
    \egroup
} %4/5
\newcommand{\useSavedNumAns}[1]{\csname NumAns#1\endcsname}
\long\def\@@@SaveAnsGobbleAns#1\eAns{} % 4/5
\long\def\@getListAns\Ans#1\eAns{%
    \eq@saveAns#1\eAns % 4/5
    \Ans#1\vspace{\@rowskip}%
    \@layoutListAns
}
\long\def\@lookforendansChoices#1{%
    \@ifnextchar\eChoices{\rowsep{\rowsep@default}%
        \expandafter\@findendans\@gobble}{\expandafter#1\@gobble}%
}
\def\@layoutTabularAns#1{%
    \let\eq@tabSep=\@empty
    \xdef\numShortCols{#1}%
    \ifnum#1>\aeb@numCols \xdef\numShortCols{\aeb@numCols}\fi
    \@@layoutTabularAns
}
\def\@@layoutTabularAns{%
    \@ifnextchar\Ans{\@getTabAns}%
        {\@lookforendansChoices{\@@layoutTabularAns}}%
}
\def\rowsep#1{\gdef\@rowsep{[#1]}\gdef\@rowskip{#1}}%
    \rowsep{\rowsep@default}
\def\rowsepDefault#1{\def\rowsep@default{#1}}
\def\rowsep@default{0pt}
\long\def\@getTabAns\Ans#1\eAns{%
    \eq@saveAns#1\eAns
    \global\advance\eq@tabColCnt1
    \let\@save@tabSep=\eq@tabSep
    \ifnum\eq@tabColCnt=\numShortCols
        \global\eq@tabColCnt=0
            \xdef\eq@tabSep{\noexpand\\\noalign{\kern\@rowskip\relax}}%
    \else
        \gdef\eq@tabSep{&}%
    \fi
    \@ifnextchar\eChoices{\@save@tabSep\Ans#1\rowsep{\rowsep@default}
        \expandafter\@findendans\@gobble}%
        {\@save@tabSep\Ans#1\@@layoutTabularAns}%
}
\def\@findendans{\@ifnextchar\end{% 4/5
    \ifx\bChoiceLabel\@empty\else
    \processLabeledAns\fi
    }{\expandafter\@findendans\@gobble}}%
\def\defineEachAns#1{\advance\count0by1\relax
    \@temptokena={#1}\expandafter\xdef
    \csname SavedAns\bChoiceLabel-Idx\the\count0\endcsname{#1}%
    \ifsolutionsonly\else{\let\\\relax\eq@IWDefs{%
    \string\expandafter\string\gdef\string\csname\space
    SavedAns\bChoiceLabel-Idx\the\count0%
    \string\endcsname{\the\@temptokena}}}\fi
}
\def\defineEachChoice#1{\advance\count0by1\relax
    \@temptokena={#1}\expandafter\xdef
    \csname SavedAlts\bChoiceLabel-Idx\the\count0\endcsname{#1}%
    \ifsolutionsonly\else{\let\\\relax\eq@IWDefs{%
    \string\expandafter\string\gdef\string\csname\space
    SavedAlts\bChoiceLabel-Idx\the\count0%
    \string\endcsname{\the\@temptokena}}}\fi
}
\def\processLabeledAns{% 4/5 \@tempholdSaveAns \@tempholdSaveChoice
    \bgroup
        \let\label\@gobble
        \toks@=\expandafter{\@tempholdSaveAns}\expandafter
        \xdef\csname SavedAns\bChoiceLabel\endcsname{\the\toks@}%
        \ifsolutionsonly\else{\let\\\relax\eq@IWDefs{%
        \string\expandafter\string\gdef\string\csname\space
        SavedAns\bChoiceLabel\string\endcsname{\the\toks@}}}\fi
        \count0=0\relax\let\\\defineEachAns\the\toks@
        \xdef\@currNCntAns{\the\count0 }\expandafter
        \xdef\csname NumAns\bChoiceLabel\endcsname{\@currNCntAns}%
        \ifsolutionsonly\else{\eq@IWDefs{%
        \string\expandafter\string\gdef\string\csname\space
        NumAns\bChoiceLabel\string\endcsname{\the\count0 }}}\fi
        \toks@=\expandafter{\@tempholdSaveChoice}\expandafter
        \xdef\csname SavedAlts\bChoiceLabel\endcsname{\the\toks@}%
        \ifsolutionsonly\else{\let\\\relax\eq@IWDefs{%
        \string\expandafter\string\gdef\string\csname\space
        SavedAlts\bChoiceLabel\string\endcsname{\the\toks@}}}\fi
        \count0=0\relax\let\\\defineEachChoice\the\toks@
        \count0=1\relax\toks@={\ignorespaces\@gobble}%
        \loop
            \edef\temp@exp{\the\toks@, %
            \noexpand\useSavedAlts[\the\count0]{\bChoiceLabel}
            \noexpand\useSavedAns[\the\count0]{\bChoiceLabel}}%
            \toks@=\expandafter{\temp@exp}%
            \ifnum\count0<\@currNCntAns
            \advance\count0by1
        \repeat
        \expandafter
        \xdef\csname SavedAltsAns\bChoiceLabel\endcsname{\the\toks@}%
        \ifsolutionsonly\else{\let\\\relax\eq@IWDefs{%
        \string\expandafter\string\gdef\string\csname\space
        SavedAltsAns\bChoiceLabel\string\endcsname{\the\toks@}}}\fi
    \egroup
}
\def\graylettersOn{\aebshowgrayletterstrue}
\def\graylettersOff{\aebshowgraylettersfalse}
\def\aeb@exiii{\expandafter\expandafter\expandafter}
\def\REF{\@ifstar{\let\isREFstar=1\aeb@REFstar}
    {\let\isREFstar=0\aeb@REF}}
\def\aeb@REFstar#1{\@ifundefined{r@#1}{\hbox{\reset@font\bfseries ??}}
    {\ifaebshowgrayletters\aeb@buildUpperCaseRef{#1}%
     \else\ref*{#1}\fi}%
}
\def\aeb@REF#1{\@ifundefined{r@#1}{\hbox{\reset@font\bfseries ??}}
    {\ifaebshowgrayletters\aeb@buildUpperCaseRef{#1}%
     \else\ref{#1}\fi}%
}
\def\aeb@buildUpperCaseRef#1{%
    \xdef\tmp@expand{\aeb@exiii\@firstoftwo\csname r@#1\endcsname}%
    \xdef\tmp@expand{\uppercase{\tmp@expand}}\tmp@expand
}
\endinput
%%
%% End of file `eqexam.def'.

    \def\eqexheader@wrapper{\makebox[0pt][r]{\eqexheader}}
}
%    \end{macrocode}
% We wrote \verb!\begin{eqequestions}! to the top of the solutions file (\cs{jobname.sol}.
%    \begin{macrocode}
%\writeBeginEqeQuestions
%    \end{macrocode}
% If the \texttt{vspacewithsolns} is in effect, we write solutions to the end of the document.
%    \begin{macrocode}
\ifvspacewithsolns\writeAllAnsAtEnd\else
\ifeqfortextbook\writeAllAnsAtEnd\fi\fi
%    \end{macrocode}
% We execute \cs{vspacewithkeyOff}, which sets \cs{ifkeepdeclaredvspacing} to false,
% the default behavior of \textsf{eqexam} before the new feature.
%    \begin{macrocode}
\vspacewithkeyOff
%    \end{macrocode}
%    \begin{macrocode}
%</package>
%    \end{macrocode}
% \section{Stand alone Code}
%
%    \begin{macrocode}
%<*standalone>
%    \end{macrocode}
% Now we begin the listing of the stand alone code. This code is necessary if
% \textsf{exerquiz} has not been loaded, which is the case if there is no \textsf{PDF} options
% or if the \texttt{pdf} option is taken.
%
% Many of the following definitions are given in \texttt{eqforms}, which was recently separated
% from \texttt{exerquiz} and is now maintained as a separate package.
%    \begin{macrocode}
\ProvidesFile{eqalone.def}
 [2012/25/01 v3.0t Minimal code used by eqexam (dps)]
\@ifundefined{eq@tmpbox}{\newsavebox{\eq@tmpbox}}{}% defined in eforms
\@ifundefined{eq@tmpdima}{\newdimen\eq@tmpdima}{}  % defined in eforms
\def\RadioFieldSize{11bp}
%    \end{macrocode}
%    \begin{macrocode}
\newdimen\eqcenterWidget
%    \end{macrocode}
% This macro is used to vertically center the response box on the line. Seems to
% work well.
%    \begin{macrocode}
\def\centerWidget
#1{%
    \eqcenterWidget=#1
    \eqcenterWidget=.5\eqcenterWidget
    \advance\eqcenterWidget by-4bp
}
%    \end{macrocode}
% \DescribeMacro{\eqe@Bbox}When the \texttt{preview} option has been used, draw a frame box
% around the bounding rectangle.
%    \begin{macrocode}
\def\eqe@Bbox#1#2{\vbox{\hrule width #1
    \hbox to#1{\vrule height#2\hfill\vrule height#2}\vfill\hrule}}
\let\Bbox\eqe@Bbox
%    \end{macrocode}
% \DescribeMacro{\Rect} is used internally to color a link.
%    \begin{macrocode}
\def\Rect#1{\textcolor{\@linkcolor}{#1}}
%    \end{macrocode}
% The auxiliary file \texttt{eqexam.def}, created by \textsf{exerquiz}, writes
% \DescribeMacro{\ReturnTo}\cs{ReturnTo} to the \textsf{SOL} file
% in the form \verb!\ReturnTo{page.1}{\hbox {}}!. We want to remove the
% \cs{hbox} because it causes, at times, more vertical space that is wanted
% in an exam document.
%    \begin{macrocode}
\def\eqe@striphbox\hbox#1{#1}
%\newcommand{\ReturnTo}[2]{\eq@fititin{#2}}
\newcommand{\ReturnTo}[2]{\eq@fititin{\eqe@striphbox#2}}
%    \end{macrocode}
% \DescribeMacro{\proofingsymbol} The definition of the proofing symbol, this
% symbol marks the correct answer of a multiple choice question when the
% \texttt{proofing} option is used.
%    \begin{macrocode}
\newcommand{\proofingsymbol}[1]{%
    \def\@proofingsymbol{\textcolor{\@proofingsymbolColor}{#1}}}
\proofingsymbol{$\bullet$}
%    \end{macrocode}
% This is the answers macro for the link-style and is called from the \texttt{eqexam.def} file.
%    \begin{macrocode}
%</standalone>
%    \end{macrocode}
%    \begin{macrocode}
%<*package>
%    \end{macrocode}
% \section{The Main Code}
% We now continue with the main package. Mostly, we define macros specific to the
% \texttt{eqexam} package: define the \texttt{problem} and \texttt{problem*} environments,
% macros for calculating totals per page, etc.
%
%    \begin{macrocode}
\def\Ans@sq@l@driver{%
    \Rect{\makebox[\eq@tmpdima]{\linkContentWrapper}}%
    \Ans@proofing{\eq@tmpdima}%
}
%    \end{macrocode}
% This is the answers macro for the form-style and is called from the \texttt{eqexam.def} file.
%    \begin{macrocode}
\def\Ans@sq@f@driver{%
    \centerWidget\RadioFieldSize
    \leavevmode\lower\eqcenterWidget\eqe@Bbox
        {\RadioFieldSize}{\RadioFieldSize}%
    \Ans@proofing{\RadioFieldSize}%
}
%    \end{macrocode}
% Write quiz solutions to the exercise solutions file
%    \begin{macrocode}
\def\eq@sqsllabel{\string\textbf{Solution to Quiz:}}
\def\sqsllabel{\eq@sqsllabel}
%    \end{macrocode}
%    \begin{macro}{\writeToSolnFile}
% General purpose command for writing to the solution file.
%    \begin{macro}{\preExamSolnHead}
% Executed just before a user friendly name
%    \begin{macro}{\examSolnHeadFmt}
% Format for the user friendly name
%    \begin{macro}{\postExamSolnHead}
% Executed just after a user friendly name
%    \begin{macrocode}
\let\quiz@solns\ex@solns
\newcommand{\preExamSolnHead}{\goodbreak\noindent}
\newcommand{\examSolnHeadFmt}[1]{\textbf{#1}}
\newcommand{\postExamSolnHead}{\par\medskip}
\newcommand{\writeToSolnFile}[1]{%
    \set@display@protect
    \eqe@IWO\quiz@solns{#1}%
    \set@typeset@protect
}
%    \end{macrocode}
% We will write all solutions to the \texttt{.sol} auxiliary file.
%    \begin{macrocode}
\def\eqe@writetoSolns#1{%
    \set@display@protect
    \eqe@IWO\quiz@solns{\string\preExamSolnHead
        \string\examSolnHeadFmt{#1}\string\postExamSolnHead}%
    \set@typeset@protect
}
\def\eqe@writetoAux#1{%
    \set@display@protect
    \eqe@IWO\@auxout{#1}%
    \set@typeset@protect
}
%    \end{macrocode}
%    \end{macro}
%    \end{macro}
%    \end{macro}
%    \end{macro}
% Turn off interactivity of short quiz.
\def\Ans@sq@l@Actions{}
\def\Ans@sq@f@Actions{}
%    \begin{macrocode}
%    \end{macrocode}
% This macro is defined in \texttt{exerquiz}, but has a little different definition for \textsf{eqexam}.
%    \begin{macrocode}
\def\Ans@proofing
#1{%
    \ifeq@proofing\if\Ans@choice1\relax
        \llap{\rlap{\,\@proofingsymbol}\hskip#1\relax}%
    \fi\fi
}
%    \end{macrocode}
% This macro gets the page number of the last page of the exam. It is read in through
% a macro definition made and written to the \texttt{.aux} file.
%    \begin{macrocode}
\def\eq@ExamLastPage{\csname eqExamLastPage\endcsname}
\newcommand{\nPagesOnExam}{\csname eqExamLastPage\endcsname}
%    \end{macrocode}
%    \begin{macro}{\lastPageOfExam}
%\changes{v2.0l}{2011/05/05}{%
% Returns the page number of the end of the exam with a given name.
%}
% Returns the page number of the end of the exam with a name of \texttt{\#1}.
%    \begin{macro}{\firstPageOfExam}
%\changes{v2.0l}{2011/05/05}{%
% Returns the page number of the beginning of the exam with a given name.
%}
% Returns the page number of the beginning of the exam with a name of \texttt{\#1}.
%    \begin{macrocode}
\newcommand{\lastPageOfExam}[1]{\pageref{#1PageEnd}}
\newcommand{\firstPageOfExam}[1]{\pageref{#1PageBegin}}
%    \end{macrocode}
%    \end{macro}
%    \end{macro}
%    The last two commands are meant to produce typeset numbers; however, there
%    is a need to convert these to numbers that tex's registers can manipulate.
%    Here goes. \DescribeMacro{\eqe@defNumRefii}\cmd{\eqe@defNumRefii} takes
%    its argument and strips away the other arguments of \cs{pageref}; it picks
%    off the second of two or five, depending if \textsf{hyperref} is loaded.
%    It defines a macro |\csname nRefii@#1\endcsname| whose value is a page
%    number of the referenced object.
%    \begin{macrocode}
\newcommand{\eqe@defNumRefii}[1]{%
    \@ifundefined{hyperref}{\let\@getsecondOf\@secondoftwo}
    {\let\@getsecondOf\@secondoffive}%
    \@ifundefined{r@#1}{\expandafter
%    \end{macrocode}
%    If the reference \texttt{r@\#1} is undefined, define the value to be 0
%    \begin{macrocode}
    \gdef\csname nRefii@#1\endcsname{0}}{%
%    \end{macrocode}
%    If the reference \texttt{r@\#1} is defined, define the value to be
%    the second argument of \cs{r@\#1} expanded
%    \begin{macrocode}
    \expandafter\xdef\csname nRefii@#1\endcsname
        {\expandafter\expandafter\expandafter
    \@getsecondOf\csname r@#1\endcsname}%
    }%
}
%    \end{macrocode}
%    \DescribeMacro{\eqe@numRefii} takes one argument, the control
%    name. Its value is zero or \cs{nRefii@\#1}. This expands to a number
%    in all cases. It can be used in tex comparisons.
%    \begin{macrocode}
\def\eqe@numRefii#1{%
    \expandafter\ifx\csname nRefii@#1\endcsname\relax 0\else
    \csname nRefii@#1\endcsname\fi}
%    \end{macrocode}
%    \begin{macro}{\numLastPageOfExam}
%    \begin{macro}{\numFirstPageOfExam}
%    This is the user-interface to acquiring the first and last page
%    numbers of the exam with name \texttt{\#1}. These can be used
%    in comparisons, e.g.
%\begin{verbatim}
%   \rfooteqe{\ifnum\value{page}<\numLastPageOfExam{<myTest>}%
%       \textbf{Test Continues}\fi}
%\end{verbatim}
%    \begin{macrocode}
\newcommand{\numLastPageOfExam}[1]{\eqe@numRefii{#1PageEnd}}
\newcommand{\numFirstPageOfExam}[1]{\eqe@numRefii{#1PageBegin}}
%    \end{macrocode}
%    \end{macro}
%    \end{macro}
%    \begin{macrocode}
\newcommand{\makeRefsNums}{%
    \@ifundefined{thePartNames}{}{\begingroup
        \def\\##1{\typeout{processing ##1}%
            \eqe@defNumRefii
            {##1PageEnd}\eqe@defNumRefii{##1PageBegin}}%
        \thePartNames
    \endgroup}%
}
%    \end{macrocode}
%    \begin{macrocode}
\AtBeginDocument{\makeRefsNums}
%    \end{macrocode}
% \subsection{Running Heads and Feet}
% We develop a series of macros for creating running headers and footers for the exam.
%    \begin{macro}{\lheadeqe}
%    \begin{macro}{\cheadeqe}
%    \begin{macro}{\cheadeqe}
%\changes{2.0a}{2010/05/06}{
% Changed the definitions of \cs{lhead}, \cs{chead}, and \cs{rhead} so they don't clash
% with the \textsf{fancyhdr} package. If \textsf{fancyhdr} is not loaded at the time \textsf{eqexam} is loaded, we
% \cs{let} the old names to the new names. Therefore, when \textsf{fancyhdr} is loaded first use
% the new definitions.
%}
% Set the left, center, and right running headers.
%    \begin{macrocode}
\newcommand{\lheadeqe}[1]{\def\eq@lhead{#1}}
\lheadeqe{\shortwebsubject/\shortwebtitle}
\newcommand{\cheadeqe}[1]{\def\eq@chead{#1}}
\cheadeqe{-- Page \arabic{page}\space of \eq@ExamLastPage\space--}
\newcommand{\rheadeqe}[1]{\def\eq@rhead{#1}}
%    \end{macrocode}
%    The default is \cs{eq@ExamName}, which is defined by \cs{eqExamName}, the default displays the word \texttt{"Name"}
%    and an underlined horizontal space for the student to enter his/her name.
%    \begin{macrocode}
\rheadeqe{\eq@ExamName}
%    \end{macrocode}
%    \begin{macro}{\lhead}
%    \begin{macro}{\chead}
%    \begin{macro}{\rhead}
% These are the original names for the headers, we'll keep them if
% \textsf{fancyhdr} is not already loaded to maintain compatibility
% with previous versions of \textsf{eqexam}. The use of these commands
% is \emph{discouraged}.
%    \begin{macrocode}
\@ifpackageloaded{fancyhdr}{}{%
    \let\lhead\lheadeqe
    \let\chead\cheadeqe
    \let\rhead\rheadeqe
}
%    \end{macrocode}
%    \end{macro}
%    \end{macro}
%    \end{macro}
%    \end{macro}
%    \begin{macro}{\runExamHeader}
%    \begin{macro}{\eqExamRunHead}
% The running header of the exam, may be redefined.
%    \begin{macrocode}
\newcommand{\runExamHeader}{\eq@lhead\hfill\eq@chead\hfill\eq@rhead}
\newcommand\eqExamRunHead{%
    \addtolength\textwidth{\oddsidemargin}%
    \noindent\hspace*{-\oddsidemargin}\makebox[\textwidth]
    {\runExamHeader}%
}
%    \end{macrocode}
%    \end{macro}
%    \end{macro}
%    \paragraph*{Running footers.}
%    One or two users wanted to use running footers, so
%    here they are.
%    \begin{macro}{\lfooteqe}
%    \begin{macro}{\cfooteqe}
%    \begin{macro}{\rfooteqe}
%    There has been some demand for running footers. You  have to be
%    a little careful, \textsf{eqexam} uses the footer for the
%    command \cmd{\settotalsbox}, which puts in the totals for the pages
%    either on the left (\texttt{totalsonleft}) or right (\texttt{totalsonright}) side, depending on the option.
%    \begin{macrocode}
\newcommand{\lfooteqe}[1]{\def\eq@lfoot{#1}}
\lfooteqe{}
\newcommand{\cfooteqe}[1]{\def\eq@cfoot{#1}}
\cfooteqe{}
\newcommand{\rfooteqe}[1]{\def\eq@rfoot{#1}}
\rfooteqe{}
%    \end{macrocode}
%    \DescribeMacro{\runExamFooter}Considering the defaults for the left, center, and right footer elements,
%    the default footer contributes nothing, except inserting
%    \cmd{\settotalsbox} (see the definition of \cs{@oddfoot})
%    \begin{macrocode}
\newcommand{\runExamFooter}{\eq@lfoot\hfill\eq@cfoot\hfill\eq@rfoot}
%    \end{macrocode}
%    \end{macro}
%    \end{macro}
%    \end{macro}
%    \end{macro}
%    \end{macro}
%    \paragraph*{Running headers for solutions.}
%    We provide a special set of headers for the solution pages.
%    The document author needs to manage running footers for the solution
%    pages.
%    \begin{macro}{\lheadSol}
%    \begin{macro}{\cheadSol}
%    \begin{macro}{\rheadSol}
%    \begin{macro}{\runExamHeaderSol}
%    \begin{macro}{\eqsolutionshook}
% The running header of the exam, when solutions are included at the end of
% the document, perhaps for posting the solutions to the exam, or
% publication of a ``pretest''. Note that \cs{eqsolutionshook} is defined
% in \texttt{exerquiz/eqexam.def}. May be redefined.
%    \begin{macrocode}
\newcommand{\lheadSol}[1]{\def\eq@lheadSol{#1}}
\lheadSol{\shortwebsubject/\shortwebtitle}
\newcommand{\cheadSol}[1]{\def\eq@cheadSol{#1}}
\cheadSol{-- Page \arabic{page}\space of \eq@ExamLastPage\space--}
\newcommand{\rheadSol}[1]{\def\eq@rheadSol{#1}}
\rheadSol{SOLUTIONS}
%    \end{macrocode}
%    The \cs{runExamHeaderSol} distributes the three solution headers across
%    the page.
%    \begin{macrocode}
\newcommand{\runExamHeaderSol}
    {\eq@lheadSol\hfill\eq@cheadSol\hfill\eq@rheadSol}
%    \end{macrocode}
%    The solutions headers are inserted using the \cs{eqsolutionshook}, see
%    the definition of \cmd{\exerSolnInput}.
%    \begin{macrocode}
\def\eqsolutionshook
{%
    \gdef\eqExamRunHead{\addtolength\textwidth{\oddsidemargin}%
    \noindent\hspace*{-\oddsidemargin}\makebox[\textwidth]
    {\runExamHeaderSol}}%
}
%    \end{macrocode}
%    \end{macro}
%    \end{macro}
%    \end{macro}
%    \end{macro}
%    \end{macro}
% \subsection{\texorpdfstring{\protect\cs{maketitle}}{\textbackslash{maketitle}} definitions}
%    \begin{macro}{\maketitle}
%    \begin{macro}{\maketitledesign}
%    \begin{macro}{\altTitle}
% Standard {\LaTeX} macro, but this time it is used to create the header at the top of the first
% page of the exam. Typically, consisting of two rows of info. (1) first row has course name, exam name, and
% a place for the student to put his/her name. (2) second row has date and instructor. May be redefined.
%
% Modify the title by redefining \cs{maketitledesign}, the \cs{maketitle} command
% itself has {\LaTeX} commands in it that should not be changed.
% \changes{v1.6i}{2007/09/18}
%{
% Added the command \cs{altTitle} as an alternate title for the exam
% document.  This alternate title appear centered under the title
% of the document.
%}
%    \begin{macrocode}
\newcommand\maketitledesign
{%
    \makebox[\textwidth]{\normalsize
        \shortstack[l]{\strut\websubject\\\@date}\hfill
        \shortstack[c]{\webtitle\\\strut\@altTitle}\hfill
        \shortstack[l]{\strut\eq@ExamName\\\webauthor}}%
}
\def\altTitle#1{\def\@altTitle{#1}}
\let\@altTitle\@empty
\def\eqemaketitle
{%
%    \end{macrocode}
% \cs{EQEcalculateAllTotals}: We don't actually calculate all totals, just
% some of them. We do calculate the grade total of all the \texttt{exam}
% environments in the document, we also calculate the percentage that each
% exam contributes to to the total. If \cs{maketitle} is not used, for
% whatever reason, this command should be calculated explicitly just after
% |\begin{document}|.
% \changes{v3.0y}{2012/04/20}{Moved \cs{EQEcalculateAllTotals} from the bottom
% to the top of \cs{eqemaketitle}. In case the author wants the grand total
% of the exam in the title, we need to make all calculations before
% \cs{maketitledesign.}}
%    \begin{macrocode}
    \EQEcalculateAllTotals
    \begingroup
    \addtolength\textwidth{\oddsidemargin}%
    \noindent\hspace*{-\oddsidemargin}%
    \raisebox{.7in}[0pt][0pt]{\SubmitButton}%
    \maketitledesign
    \endgroup
}
%    \end{macrocode}
% If the \texttt{fortextbook} option is not taken, we set up the usual \cs{maketitle} definition.
%    \begin{macrocode}
\ifeqfortextbook\else\let\maketitle\eqemaketitle\fi
%    \end{macrocode}
%    \end{macro}
%    \end{macro}
%    \end{macro}
% \subsection{The cover page definitions}
%    \begin{macro}{\eqexcoverpage}
% The \texttt{eqexam} package allows for the possibility of a cover page, if the \texttt{coverpage}
% option is taken.
%    \begin{macro}{\placeCoverPageLogo}
% A simple command to insert a logo on the cover page. The logo can be used to cover the
% score in the next page, if the instructor places the score under the logo. Example of usage
%\begin{verbatim}
% \placeCoverPageLogo{5in}{-1.5in}{\includegraphics{nwfsc_logo}}
%\end{verbatim}
% Working from the upper left corner, the first parameter is the amount to move to logo
% to the right, the second parameter is the amount to move the logo vertically. The
% Third parameter is the content; perhaps an \cs{includegraphics} command.
%\changes{v1.9b}{2009/09/29}{Added \cs{placeCoverPageLogo} to insert a logo on the cover page.}
%    \begin{macrocode}
\newcommand\placeCoverPageLogo[3]{%
    \def\eqe@insertLogo{\hbox to0pt{%
        \hspace*{#1}\smash{\raisebox{#2}{#3}}\hss}}}
\let\eqe@insertLogo\relax
%    \end{macrocode}
% Define \cs{eqexcoverpage}, this command places \cs{eqe@insertLogo} and
% \cs{eqex\-cover\-page\-design} in a group. It is this command that gets executed when
% the user calls for the \texttt{coverpage} option.
%    \begin{macrocode}
\def\eqexcoverpage{%
    \begingroup
    \pagenumbering{roman}
    \eqe@insertLogo
    \eqexcoverpagedesign
    \endgroup\newpage
    \pagenumbering{arabic}
}
%    \end{macrocode}
%    \end{macro}
%    \end{macro}
%    \begin{macro}{\eqexcoverpagedesign}
% The \texttt{eqexam} package allows for the possibility of a cover page, if the \texttt{coverpage}
% option is taken. This macro can and should be redefined to fit your needs. The definition below
% is just a representative example.
%    \begin{macro}{\coverpagesubject}
%    \begin{macro}{\coverpageUniversityFmt}
%    \begin{macro}{\coverpageSubjectFmt}
%    \begin{macro}{\coverpageTitleFmt}
% The following four commands are used with the cover page.
% The \cs{cover\-page\-sub\-ject} is used to provide a special subject
% for the cover page, different from \cs{websubject}. The others
% are used for formatting.
%\changes{v1.9a}{2009/28/09}{Added these various Fmt commands for coverpage.}
%    \begin{macrocode}
\newcommand{\coverpagesubject}[1]{\def\coverpage@subject{#1}}
\let\coverpage@subject\@empty
\def\eqexamsubject{\ifx\coverpage@subject\@empty\websubject
    \else\coverpage@subject\fi}
\newcommand{\coverpageUniversityFmt}[1]{%
    \def\eqex@coverpageUniversityFmt{#1}}
\coverpageUniversityFmt{\bfseries\large}
\newcommand{\coverpageSubjectFmt}[1]{%
    \def\eqex@coverpageSubjectFmt{#1}}
\coverpageSubjectFmt{\bfseries\large}
\newcommand{\coverpageTitleFmt}[1]{%
    \def\eqex@coverpageTitleFmt{#1}}
\coverpageTitleFmt{\bfseries\large}
%    \end{macrocode}
%    \end{macro}
%    \end{macro}
%    \end{macro}
%    \end{macro}
%    When \texttt{coverpage} and \texttt{coverpagesumry} are used, an \textbf{Exam Record}
%    is generated, just a summary may appear beneath the name and ID section of the cover
%    page (\DescribeMacro{\qeSumryVert}\cmd{\qeSumryVert}) or to the right
%    (\DescribeMacro{\eqeSumryHoriz}\cmd{\eqeSumryHoriz}).
%    \begin{macrocode}
\def\eqeSumryVert{\def\eqe@SumryVert{\par\vfill}%
    \let\eqe@SumryHoriz\relax
    \def\eqe@@SumryVert{\vspace{\stretch{-1}}\bigskip}}
\def\eqeSumryHoriz{\def\eqe@SumryHoriz{\hfill}\let\eqe@SumryVert\relax
    \let\eqe@@SumryVert\relax}
%    \end{macrocode}
%    The default is a horizontal orientation.
%    \begin{macrocode}
\eqeSumryHoriz
%    \end{macrocode}
%    We can name the components of the \textbf{Exam Record} by page or by parts. For the case
%    they are named by parts, there are three options: You can used the exam name (the default);
%    you can use the friendly name of the exam (the optional argument), executing
%    \DescribeMacro{\useUIPartNames}\cmd{\useUIPartNames} invokes this option; you can use
%    custom names (useful if the  friendly names are too long), execute the command
%    \DescribeMacro{\useCustomPartNames}\cmd{\useCustomPartNames} for this option.
%    \begin{macrocode}
\def\eqe@coverPageNaming{0}
\newcommand{\useUIPartNames}{\def\eqe@coverPageNaming{1}}
\newcommand{\useCustomPartNames}{\def\eqe@coverPageNaming{2}}
%    \end{macrocode}
%    When there is custom naming (\cmd{\customNaming}), we need to provide the user with a
%    way of defining these custom names. \DescribeMacro{\customNaming}\cmd{\customNaming}
%    provides that mechanism. The command takes two arguments, the first is
%    \meta{name}, the exam name, the second is the associated \meta{text} the text that is to appear
%    in the \textbf{Exam Record}
%    \begin{macrocode}
\newcommand{\customNaming}[2]{\@namedef{userCustom#1}{#2}}
%    \end{macrocode}
%    Now we present the definition of \cs{eqexcoverpagedesign}.
%    \begin{macrocode}
\newcommand{\eqexcoverpagedesign}
{%
    \thispagestyle{empty}
    \addtolength\textwidth{\oddsidemargin}
    \vspace*{.1\textheight}
    \noindent\hspace*{-\oddsidemargin}%
    \makebox[\linewidth]{\parbox{\linewidth}%
        {\eqex@coverpageUniversityFmt
        \color{\webuniversity@color}%
        \centering\webuniversity}}
    \par\vspace{.1\textheight}
    \noindent\hspace*{-\oddsidemargin}%
    \makebox[\linewidth]{\parbox{\linewidth}%
        {\eqex@coverpageSubjectFmt
        \color{\websubject@color}%
        \centering\eqexamsubject}}
    \par\vspace{\bigskipamount}
    \noindent\hspace*{-\oddsidemargin}%
    \makebox[\linewidth]{\parbox{\linewidth}%
        {\eqex@coverpageTitleFmt
         \color{\webtitle@color}%
        \centering\webtitle}}
    \par\vspace{\stretch{1}}
    \optionalpagematter
    \par\vspace{\stretch{1}}\parbox[b]{.45\linewidth}{\parindent0pt
    \eq@ExamName\\[2ex]\eq@SID\\[2ex]
    \ifx\use@email y\eq@Email\\[2ex]\fi
    \textcolor{\webauthor@color}{\webauthor}, \@date}%
    \eqe@SumryHoriz\eqe@SumryVert
    \sumryAnnots
}
%    \end{macrocode}
%    \end{macro}
%    The following are various local strings used in the \textbf{Exam Record}, the
%    default is to use English words.
%    \begin{macro}{\cpSumryHeader}
%    The header that appears at the top of the box
%    \begin{macro}{\cpSumryPts}
%    The formatting for the number of points
%    \begin{macro}{\cpSumryPage}
%    The word for ``Page''
%    \begin{macro}{\cpSumryTotal}
%    The word for ``Total''
%    \begin{macro}{\cpSumryGrade}
%    The word for ``Grade''
%    \begin{macro}{\cpSetSumryWidth}
%    The width of the summary box
%    \begin{macro}{\cpUsefbox}
%    Enclose in an \cs{fbox}? The default is yes
%    \begin{macro}{\cpNofbox}
%    Do not enclose in an \cs{fbox}
%    \begin{macrocode}
\newcommand{\cpSumryHeader}{\textbf{Exam Record}}
\newcommand{\cpSumryPts}{\,\text{pts}}
\newcommand{\cpSumryPage}{Page}
\newcommand{\cpSumryTotal}{Total:}
\newcommand{\cpSumryGrade}{Grade:}
\newcommand{\cpSetSumryWidth}[1]{\def\cp@SetSumryWidth{#1}}
\cpSetSumryWidth{.5\linewidth}
\newcommand{\cpUsefbox}{\let\cp@Usefbox\fbox}
\cpUsefbox
\newcommand{\cpNofbox}{\let\cp@Usefbox\mbox}
%    \end{macrocode}
%    \end{macro}
%    \end{macro}
%    \end{macro}
%    \end{macro}
%    \end{macro}
%    \end{macro}
%    \end{macro}
%    \end{macro}
%    A helper command used in \cs{cpSumybyparts}
%    \begin{macrocode}
\def\cp@IsertNaming#1{%
    \ifcase\eqe@coverPageNaming
    #1\or\@nameuse{userFriendly#1}\or
    \@nameuse{userCustom#1}\else#1\fi
}
%    \end{macrocode}
%    \DescribeMacro{\cpSumrybyparts}
%    The routine for building the summary box, where we list the statistics
%    for each part.
%    \begin{macrocode}
\newcommand{\cpSumrybyparts}{%
    \eqe@@SumryVert\cp@Usefbox{%
    \begin{minipage}[b]{\cp@SetSumryWidth}\kern0pt
        \begin{flushleft}
            \expandafter\ifx\csname NumberOfParts\endcsname\relax
            \else
                \count2=0
                \medskip
                \cpSumryHeader\par\parskip\bigskipamount
                \def\\##1{\advance\count2by\csname ##1total\endcsname
                \underbar{\makebox[0pt][l]{%
                \cp@IsertNaming{##1}}\hspace*{1.5in}}%
                / $\csname ##1total\endcsname\cpSumryPts$\par}
                \thePartNames
            \fi
            \underbar{\makebox[0pt][l]{\cpSumryTotal}\hspace{1.5in}}/
                $\the\count2\relax\cpSumryPts$ \par
            \underbar{\makebox[0pt][l]{\cpSumryGrade}\hspace{1.5in}}%
            \ifx\cp@Usefbox\fbox\medskip\fi
        \end{flushleft}\kern0pt
    \end{minipage}}\par
}
%    \end{macrocode}
%    \DescribeMacro{\cpSumrybypages}
%    The routine for building the summary box, where we list the statistics
%    for each page.
%    \begin{macrocode}
\newcommand{\cpSumrybypages}{%
    \eqe@@SumryVert\cp@Usefbox{%
    \begin{minipage}[b]{\cp@SetSumryWidth}
        \begin{flushleft}
            \expandafter\ifx\csname NumberOfParts\endcsname\relax
            \else\medskip\count0= 0\relax\count2=0\relax
                \cpSumryHeader\par\parskip\bigskipamount
                \@whilenum\count0<\csname eqExamLastPage\endcsname\do{%
                    \advance\count0by1\relax
                    \count4=\@nameuse{Page\the\count0total}\relax
                    \expandafter\ifx\csname%
                        Page\the\count0spilltotal\endcsname\relax
                    \else\advance\count4by\csname%
                        Page\the\count0spilltotal\endcsname\relax\fi
                    \advance\count2by\count4\relax\underbar{%
                    \makebox[0pt][l]{\cpSumryPage~\the\count0}%
                    \hspace*{1.5in}}/ $\the\count4\relax\cpSumryPts$%
                        \expandafter\ifx\csname%
                            Page\the\count0spilltotal\endcsname\relax
                        \else\ $(\@nameuse{Page\the\count0spilltotal}%
                            \cpSumryPts+\@nameuse{Page\the\count0total}%
                            \cpSumryPts)$\fi\par
                }%
            \fi
            \underbar{\makebox[0pt][l]{\cpSumryTotal}\hspace{1.5in}}/
                $\the\count2\relax\cpSumryPts$\par
            \underbar{\makebox[0pt][l]{\cpSumryGrade}\hspace{1.5in}}%
            \ifx\cp@Usefbox\fbox\medskip\fi
        \end{flushleft}
    \end{minipage}}\par
}
%    \end{macrocode}
% If the author takes the \texttt{coverpage} option, \cs{eqex@coverpage} is set equal
% to \cs{eqexcoverpagedesign}, otherwise, it is set equal to \cs{relax}.
%    \begin{macrocode}
\AtBeginDocument{\eqex@coverpage}
%    \end{macrocode}
% \subsection{Insert Points in Margins and Compute Page Totals}
% Here we attempt to place point values of a problem in the margins and
% to compute the page totals.
%
%\begin{macro}{\probvalue}
% This is a fundamental macro for keeping track of the points of the problem.
% It increments the counter \texttt{eqpointvalue}, which keeps a running total
% of the points of the current part of the exam, puts the value in the margins,
% and sets a \texttt{mark}, so that at the end of the page, we can compute the
% number of points on the current page. This macro is used in several situations,
% for example, in the \texttt{problem} environment, \cs{manualcalcparts} and in
% \cs{autocaleparts}. \textbf{Should not be redefined}.
%\begin{flushleft}
%Parameters
%\begin{verbatim}
%#1 = total points for this problem
%#2 = 0 if total points, otherwise, #2 is the number of points each
%     problem.
%\end{verbatim}
%\end{flushleft}
%    \begin{macrocode}
\def\@marktotalvalue{%
    \mark{\theeqpointvalue\csname eqExam\endcsname\theeq@numparts}%
}
\def\probvalue#1#2{%
    \addtocounter{eqpointvalue}{#1}%
    \marginpoints{#1}{#2}\@marktotalvalue
}
%    \end{macrocode}
%    \end{macro}
% \DescribeMacro{\widthtpboxes} is the width of the box in the margins that
% contains the points or totals. The design of the box keys off this width.
%    \begin{macrocode}
\newcommand{\widthtpboxes}{35pt}
%    \end{macrocode}
% \DescribeMacro{\marginboxdesign} is the basic box that encloses the points
% on the right, and the totals. This box may be redefined as desired, in which case,
% \cs{measurePtBoxHt} should be re-executed.
%    \begin{macrocode}
\newcommand{\marginboxdesign}[2][]{%
    \parbox{\widthtpboxes}{\tabcolsep=0pt\relax
        \begin{tabular}{|c|}\hline
            \vrule height15pt width0pt#1\\\hline
            \makebox[\widthtpboxes-2\fboxrule]{#2}\\\hline
        \end{tabular}%
    }%
}
%    \end{macrocode}
%    \begin{macro}{\eqleftmarginbox}
% This macro places the problem value in the left margin, can be redefined, if you dare.
%\begin{flushleft}
% Parameters
%\begin{verbatim}
%#1 = total points for this problem
%#2 = 0 if total points, otherwise, #2 is the number of points each
%     problem.
%\end{verbatim}
%\end{flushleft}
% Currently, this macro is not used so I'll make it into verbatim text.
%\begin{verbatim}
\newcommand{\eqleftmarginbox}[2]{\makebox[0pt][r]{%
    \setlength\tabcolsep{0pt}%
    \raisebox{-.5\height}[0pt][0pt]{%
        \marginboxdesign{\marginpointsboxtext{#1}{#2}}%
    }\hspace*{\marginparsep}\hspace*{\eqemargin}}%
}
%\end{verbatim}
%    \end{macro}
% \paragraph*{Formatting the points}
%    \begin{macro}{\ptsLabel}
%    \begin{macro}{\eachLabel}
%    \begin{macro}{\pointsLabel}
%  The following three convenience commands can be used to localize some of
%  the strings to other languages.
%    \begin{macrocode}
\providecommand{\ptsLabel}[1]{\def\eqptsLabel{#1}}\ptsLabel{pts}
\providecommand{\ptLabel}[1]{\def\eqptLabel{#1}}\ptLabel{pt}
\newcommand{\eachLabel}[1]{\def\eq@eachLabel{#1}}\eachLabel{ea.}
\newcommand{\pointsLabel}[1]{%
    \def\eq@pointsLabel{#1}}\pointsLabel{points}
\newcommand{\pointLabel}[1]{%
    \def\eq@pointLabel{#1}}\pointLabel{point}
%    \end{macrocode}
%    \end{macro}
%    \end{macro}
%    \end{macro}
% \paragraph*{pointsonleft or pointsonboth}
%    \begin{macro}{\marginpointtext}
% Creates the text for \cs{eqleftmargin} to use. From the macro definition,
% if \texttt{\#2} is 0, then we write the points for the problem, else, we write
% the points each for the problem. \DescribeMacro{\leftmarginPtsTxt}\cmd{\leftmarginPtsTxt}
% is the formatting for the total points for the problem; while
% \DescribeMacro{\leftmarginPtsEaTxt}\cmd{\leftmarginPtsEaTxt} is the formatting
% for the \verb!{problem*}{<num>ea}!-type problems.
%    \begin{macrocode}
\newcommand{\marginpointtext}[2]{\ifnum#2=0\leftmarginPtsTxt{#1}\else
    \leftmarginPtsEaTxt{#2}\fi
}
%    \end{macrocode}
% \texttt{\#1} is the number of points for this problem.
%    \begin{macrocode}
\newcommand{\leftmarginPtsTxt}[1]{(\small$#1^{\text{%
    \ifnum#1=1\relax\eqptLabel\else\eqptsLabel\fi}}$)}
%    \end{macrocode}
% \texttt{\#1} is the number of points for each part of this problem.
%    \begin{macrocode}
\newcommand{\leftmarginPtsEaTxt}[1]{(\small$#1_{\text{%
    \eq@eachLabel}}^{\text{\ifnum#1=1\relax\eqptLabel\else
    \eqptsLabel\fi}}$)}
%    \end{macrocode}
%    \end{macro}
%    \begin{macro}{\eqleftmargin}
% Places the number of points (or points each) in the left margin. Can be redefined as
% desired. This macro is used when author chooses the \texttt{pointsonleft}
% or \texttt{pointsonbothsides} option. The text for the points is defined above,
% \cmd{\marginpointtext}.
%    \begin{macrocode}
\newcommand{\eqleftmargin}[2]{\makebox[0pt][r]{\marginpointtext{#1}{#2}%
    \setlength{\@tempdima}{\marginparsep+\eqemargin}%
    \hspace*{\@tempdima}}%
}
%    \end{macrocode}
%    \end{macro}
% \paragraph*{pointsonright or pointsonboth}
%    \begin{macro}{\marginpointsboxtext}
% When the author selects \texttt{pointsonright} or \texttt{pointsonbothsides}, a box
% appears in the right margin containing problem totals, this is the text for the box.
%    \begin{macrocode}
\newcommand{\marginpointsboxtext}[2]{\small$#1\,\text{%
    \ifnum#1=1\relax\eqptLabel\else\eqptsLabel\fi}$}
%    \end{macrocode}
%    \end{macro}
%    \begin{macro}{\eqrightmarginbox}
% When the author selects \texttt{pointsonright} or \texttt{pointsonbothsides}, a box
% appears in the right margin containing problem totals, this is the box that appears.
%    \begin{macro}{\insertPointsBoxPDF}
%    \begin{macro}{\insertTotalsBoxPDF}
% These two commands are \cmd{\let}ing them be either a form field (that the
% document author can fill in during online grading, or as \cs{relax}. It depends
% on whether the \texttt{email} option is taken or not.
%    \begin{macrocode}
\def\@insertPointsBoxPDF
{
    \def\fieldName{pointsgiven.\curr@quiz.page\thepage.%
        \theeqquestionnoi}%
    \calcTextField[\F\FHidden\BC{}\Q1\textColor{1 0 0 rg}]
        {\fieldName}{\widthtpboxes}{15pt}%
}
\def\@insertTotalsBoxPDF
{
    \def\fieldName{pagetotals.\curr@quiz.page\thepage}%
    \calcTextField[\F\FHidden\BC{}\Q1\textColor{1 0 0 rg}
        \AA{\AACalculate{AFSimple_Calculate("SUM",
            new Array("pointsgiven.\curr@quiz.page\thepage"));}}]
    {\fieldName}{\widthtpboxes}{15pt}%
}
\ifx\use@email y
    \let\insertPointsBoxPDF\@insertPointsBoxPDF
    \let\insertTotalsBoxPDF\@insertTotalsBoxPDF
\else
    \let\insertPointsBoxPDF\@empty
    \let\insertTotalsBoxPDF\@empty
\fi
%    \end{macrocode}
%    \end{macro}
%    \end{macro}
% The points box that appears in the left margin, may be re-defined, if you dare.
% At the bottom of the box goes the points for the problem, if the \texttt{email} option
% is used, in the top, a text field is inserted.
%    \begin{macrocode}
\newcommand{\eqrightmarginbox}[2]{%
    \makebox[0pt][l]{%
        \setlength\tabcolsep{0pt}%
%    \end{macrocode}
% (2011/05/08) new calculation for the right point box.
%    \begin{macrocode}
        \setlength{\@tempdima}{\textwidth-\eqemargin+\marginparsep}%
        \hspace*{\@tempdima}%
        \raisebox{-.5\height}[0pt][0pt]{%
            \marginboxdesign[\insertPointsBoxPDF]%
                {\marginpointsboxtext{#1}{#2}}%
        }\hfil
    }%
}
%    \end{macrocode}
% At the end of each exam \verb!\end{exam}! the part totals can be optionally
% displayed on the right. This is the box for doing that.
%    \begin{macrocode}
\newcommand{\eqeomarginboxright}[2]{%
    \makebox[0pt][l]{%
        \setlength{\@tempdima}{\textwidth+\marginparsep-\parindent}%
        \hspace*{\@tempdima}%
        \raisebox{-.5\height}[0pt][0pt]{%
            \marginboxdesign[\insertTotalsBoxPDF]%
                {\marginpointsboxtext{#1}{#2}}%
        }\hfil
    }%
}
%    \end{macrocode}
% At the end of each exam \verb!\end{exam}! the part totals can be optionally
% displayed on the left. This is the box for doing that.
%    \begin{macrocode}
\newcommand{\eqeomarginboxleft}[2]{\makebox[0pt][r]{%
    \raisebox{-.5\height}[0pt][0pt]{%
        \marginboxdesign[\insertTotalsBoxPDF]%
            {\marginpointsboxtext{#1}{#2}}%
    }{\settowidth{\@tempdimb}{\eqe@hspannerPrb}%
    \setlength{\@tempdima}{\marginparsep+\eqemargin-\@tempdimb}%
    \hspace*{\@tempdima}}}\hfil
}
%    \end{macrocode}
% \DescribeMacro{\measurePtBoxHt} We measure the height of our point/total boxes and store it
% in the length \cs{eq@pointboxtotalheight}, which is used
% to separate the boxes so they don't overlap.
%    \begin{macrocode}
\newcommand{\measurePtBoxHt}{%
    \setbox\eq@pointbox=\hbox{%
        \marginboxdesign{\marginpointsboxtext{00}{0}}%
    }\setlength\eq@pointboxtotalheight{\dp\eq@pointbox+\ht\eq@pointbox}%
}
\measurePtBoxHt
%    \end{macrocode}
%    \end{macro}
% A helper command to set both margin boxes.
%    \begin{macrocode}
\newcommand{\eqbothmargins}[2]{\eqleftmargin{#1}{#2}%
    \eqrightmarginbox{#1}{#2}}
%    \end{macrocode}
% The macro \cs{probvalue}, defined above, says
%\begin{verbatim}
%\mark{\theeqpointvalue\csname eqExam\endcsname\theeq@numparts}
%\end{verbatim}
% so the \cs{botmark} contains the point value of this page and the part number for
% this page.
%    \begin{macrocode}
\def\lastparttotaled{0}
%    \end{macrocode}
% The \texttt{botmark} looks like \texttt{18\string\eqExam0}, where $18$ will be the total number of
% points accumulated for this exam part. We need to retrieve these number, I don't remember why
% I save them this way.
%
% \DescribeMacro{\parsetotals}
% The \cs{parsetotals} macro is called in \cs{settotalsbox} to retrieve the ongoing point values
% from \cs{botmark}, and returns two parameters, contained in \cs{argi} and \cs{argiii}.
% \DescribeMacro{\stripeqExam}
% Whereas, \cs{stripeqExam} just retrieves the first parameter only.
%    \begin{macrocode}
\def\parsetotals#1\eqExam#2\end{\def\argi{#1}\def\argii{#2}}
\def\stripeqExam#1\eqExam{\def\argii{#1}}
%    \end{macrocode}
%    \begin{macro}{\settotalsbox}
% This is the box containing the page total, it may appear on the left or right side bottom
% corner.
%    \begin{macrocode}
\def\settotalsbox{%
    \expandafter\parsetotals\botmark\eqExam\end
    \ifx\argi\@empty\hfil
    \else\ifx\argii\@empty\hfil
        \else
            \expandafter\stripeqExam\argii
%    \end{macrocode}
% Set \texttt{eqpointsthispage} equal to \cs{argi}, which should be the accumulated total
% for this part of the test so far.
%    \begin{macrocode}
            \setcounter{eqpointsthispage}{\argi}%
%    \end{macrocode}
% This subtract \texttt{eqpointsofar}, which should be the total for this test part through the
% previous page. The difference is the number of points for this page.
%    \begin{macrocode}
            \addtocounter{eqpointsthispage}{-\value{eqpointsofar}}%
%    \end{macrocode}
% Now we set \texttt{eqpointsofar} to the new accumulated total \cs{argi}.
%    \begin{macrocode}
            \setcounter{eqpointsofar}{\argi}%
%    \end{macrocode}
% And save to the auxiliary file for later usage.
%    \begin{macrocode}
            \xdef\lastparttotaled{\argii}%
            \eqe@IWO\@auxout{\string\expandafter\string\gdef
            \string\csname\space Page\thepage total\string\endcsname
                {\theeqpointsthispage}}%
%    \end{macrocode}
% Finally, place the totals box.
%    \begin{macrocode}
            \totalsbox
        \fi
    \fi
}
%    \end{macrocode}
%    \end{macro}
% \DescribeMacro{\totalsboxtext} This is the text that appears in the totals boxes
%    \begin{macrocode}
\newcommand\totalsboxtext{\small$\theeqpointsthispage\,\text{%
    \ifnum\theeqpointsthispage=1\relax\eqptLabel\else
    \eqptsLabel\fi}$}
%    \end{macrocode}
% \DescribeMacro{\eqevtranstotbox}\cmd{\eqevtranstotbox} is a length
% that can be used to raise or lower the position of the total boxes
% in the left/right bottom corner. The default is \texttt{0pt}.
%    \begin{macrocode}
\newlength\eqevtranstotbox
\setlength{\eqevtranstotbox}{0pt}
%    \end{macrocode}
% \DescribeMacro{\totalsboxleft}\DescribeMacro{\totalsboxright} There are two totals boxes, one for the left and one
% for the right side. These can be redefined as desired.
%    \begin{macrocode}
\newcommand{\totalsboxleft}{%
    \makebox[0pt][r]{\setlength\tabcolsep{0pt}%
        \raisebox{-\height+\eqevtranstotbox}[0pt][0pt]{%
            \marginboxdesign[\insertTotalsBoxPDF]{\totalsboxtext}%
        }\hspace*{\marginparsep}%
    }\hfil
}
\newcommand{\totalsboxright}{%
    \makebox[0pt][l]{\setlength\tabcolsep{0pt}%
        \hspace*{\textwidth}\hspace*{\marginparsep}%
        \raisebox{-\height+\eqevtranstotbox}[0pt][0pt]{%
            \marginboxdesign[\insertTotalsBoxPDF]{\totalsboxtext}%
        }%
    }\hfil
}
%    \end{macrocode}
% We begin the document by declaring the current part is part 0 with 0 points. This is the default,
% in case the author writes a document with no parts or points!
%    \begin{macrocode}
\ifeqe@nopoints\else
\AtBeginDocument{\mark{0\csname eqExam\endcsname0}}
\fi
%    \end{macrocode}
% At the end of the document, we write out the number of parts for this test, and
% the names of the parts the author has given each part.
%    \begin{macrocode}
\AtEndDocument{%
    \eqe@IWO\@auxout{\string\expandafter\string\gdef
        \string\csname\space NumberOfParts\string\endcsname
            {\arabic{eq@numparts}}}%
    \eqe@IWO\@auxout{\string\expandafter\string\gdef
        \string\csname\space thePartNames\string\endcsname
            {\the\partNames}}%
    \clearpage\addtocounter{page}{-1}\writelastpage\relax
    \addtocounter{page}{1}%
}
%    \end{macrocode}
%    \begin{macro}{\theGrandTotal}
% When an exam has multiple parts, the total of each part is computed
% and the grand total is computed with \cs{theGrandTotal}.
%    \begin{macrocode}
\def\theGrandTotal{\csname eqeGrandTotal\endcsname}
%    \end{macrocode}
%    \end{macro}
%    \begin{macro}{\totalForPart}
% User access to the total for a exam environment, one argument, the name of the exam.
%    \begin{macro}{\percentForPart}
% Attempts to calculate the percent of the total that the referenced exam (\texttt{\#1})
% contributes to the grand total.
%    \begin{macrocode}
\newcommand{\totalForPart}[1]{\csname#1total\endcsname}
\newcommand{\percentForPart}[1]{\csname#1percent\endcsname\%}
%    \end{macrocode}
%    \end{macro}
%    \end{macro}
%
%\subsection{Computing Number of Points within a Segment of the Exam}
%
% We add some commands for calculating number of points in a segment of the exam.
% A segment being defined subset of consecutive problems within an exam. We can
% define segments by placing markers (using \cs{placeMarkerHere}) between problems. We can
% then calculate the total number of points between markers.
%
% The counter below is a scratch counter for making the calculations. We assume the
% \texttt{calc} package is loaded, it does the work for us.
%
%    \begin{macrocode}
\newcounter{markerCnt}
%    \end{macrocode}
%    \begin{macro}{\placeMarkerHere}
% Placed outside of any \texttt{problem}/\texttt{problem*} environment, this command
% takes one argument, the symbolic name for this marker. We write to the auxiliary file
% and define a command whose name is based on \texttt{\#1}, and whose value is
% the cumulative total \cs{theeqpointvalue}.
%\changes{v1.8}{2008/11/02}
%{
%  Added a set of commands \cs{placeMarkerHere}, \cs{calcFromMarkers},
%  and \cs{markerTotalFmt} to enable the calculation of totals of segments
%  of the exam.
%}
%    \begin{macrocode}
\def\placeMarkerHere#1{%
    \eqe@IWO\@auxout{\string\expandafter\string\gdef
    \string\csname\space#1SaveTotalHere\string\endcsname
        {\theeqpointvalue}}%
}
%    \end{macrocode}
%    \end{macro}
%    \begin{macro}{\calcFromMarkers}
% Once the markers are in place, we can calculate the number of points defined between
% two such markers. The \cs{calcFromMarkers} takes three arguments, \texttt{\#2} and
% \texttt{\#3} are the symbolic names of the two markers. While, \texttt{\#1} is an optional
% argument for formatting the calculation. The default is \cs{@markerTotalFmt}, defined
% below in \cs{markerTotalFmt}. The value of the total is \cs{themarkerCnt}.
%    \begin{macrocode}
\newcommand{\calcFromMarkers}[3][\@markerTotalFmt]{%
    \@ifundefined{#2SaveTotalHere}{}%
        {\@ifundefined{#3SaveTotalHere}{}{\setcounter{markerCnt}%
        {\@nameuse{#2SaveTotalHere}-\@nameuse{#3SaveTotalHere}}%
        \ifnum\value{markerCnt}<0\relax
        \setcountertotalHereCnt{-\value{markerCnt}}\fi#1}}%
}
%    \end{macrocode}
%    \end{macro}
%    \begin{macro}{\markerTotalFmt}
% Used to set the global format of the marker totals. The value of the marker total
% is \cs{themarkerCnt}. The default follows the definition of \cs{markerTotalFmt}
%    \begin{macrocode}
\newcommand{\markerTotalFmt}[1]{\def\@markerTotalFmt{#1}}
\markerTotalFmt{ (\themarkerCnt\space points)}
%    \end{macrocode}
%    \end{macro}
%    \begin{macro}{\calcQsBtwnMarkers}
% \changes{v3.1}{2012/05/16}{Added \cs{calcQsBtwnMarkers}}
% \cs{calcQsBtwnMarkers}[\meta{Mrk2}]\meta{Mrk1} does a number of things; primarily,
% it determines the range of the questions between the two marks. The names of the
% commands produced are all based in the first marker name \meta{Mrk1}.
% \DescribeMacro{\ \unskip\meta{Mrk1}Start}\cs{\meta{Mrk1}Start} is the first question number that follows the
% the placement of \cs{calcQsBtwnMarkers}. Similarly,
% \DescribeMacro{\ \unskip\meta{Mrk1}End}\cs{\meta{Mrk1}End}
% is the last question number between the two marks \meta{Mrk1} and \meta{Mrk2}.
%
% We also calculate \DescribeMacro{\ \unskip\meta{Mrk1}nQs}\cs{\meta{Mrk1}nQs}, the number of questions appearing between
% \meta{Mrk1} and \meta{Mrk2}.
%    \begin{macrocode}
%\newcommand{\calcQsBtwnMarkers}[2]{\setcounter{markerCnt}{0}%
\newcommand{\calcQsBtwnMarkers}[2][]{\setcounter{markerCnt}{0}%
    \def\eqe@argi{#1}%
%    \end{macrocode}
% \cs{\meta{Mrk1}thisQnum} is the question number in effect at the point where
% the command \cs{calcQsBtwnMarkers} is inserted.
%    \begin{macrocode}
    \eqe@IWO\@auxout{\string\expandafter\string\gdef
    \string\csname\space#2thisQnum\string\endcsname
        {\theeqquestionnoi}}%
%    \end{macrocode}
% \cs{\meta{Mrk1}Start} is 0 if undefined.
%    \begin{macrocode}
    \@ifundefined{#2thisQnum}{%
        \expandafter\xdef\csname #2Start\endcsname{0}%
        \eqe@IWO\@auxout{\string\expandafter\string\gdef
        \string\csname\space#2Start\string\endcsname{0}}%
    }{%
%    \end{macrocode}
% \cs{\meta{Mrk1}Start} is the current question number plus 1.
%    \begin{macrocode}
        \setcounter{markerCnt}{\value{eqquestionnoi}+1}\expandafter
        \xdef\csname #2Start\endcsname{\the\value{markerCnt}}
        \eqe@IWO\@auxout{\string\expandafter\string\gdef
        \string\csname\space#2Start\string\endcsname
            {\@nameuse{#2Start}}}%
    }
%    \end{macrocode}
% Using now both \meta{Mrk1} and \meta{Mrk2} we calculate the difference
% in the two. Result held in \texttt{markerCnt}.
%    \begin{macrocode}
    \ifx\eqe@argi\@empty\else
    \@ifundefined{#2thisQnum}{\setcounter{markerCnt}{0}}%
        {\@ifundefined{#1thisQnum}{\setcounter{markerCnt}{0}}{%
        \setcounter{markerCnt}%
            {\@nameuse{#1thisQnum}-\@nameuse{#2thisQnum}}%
    }}%
    \fi
%    \end{macrocode}
% \cs{\meta{Mrk1}nQs} is the number of questions that appear between
% \meta{Mrk1} and \meta{Mrk2}.
%    \begin{macrocode}
    \expandafter\xdef\csname #2nQs\endcsname{\the\value{markerCnt}}%
    \eqe@IWO\@auxout{\string\expandafter\string\gdef
    \string\csname\space#2nQs\string\endcsname
        {\the\value{markerCnt}}}%
%
%    \end{macrocode}
% \cs{\meta{Mrk1}End} is 0 if undefined.
%    \begin{macrocode}
    \ifx\eqe@argi\@empty\else
    \@ifundefined{#1thisQnum}{%
        \expandafter\xdef\csname #2End\endcsname{0}%
        \eqe@IWO\@auxout{\string\expandafter\string\gdef
        \string\csname\space#2End\string\endcsname{0}}%
    }{%
%    \end{macrocode}
% \cs{\meta{Mrk1}End} is \cs{\meta{Mrk2}thisQnum}.
%    \begin{macrocode}
    \expandafter\xdef\csname #2End\endcsname{\@nameuse{#1thisQnum}}%
    \eqe@IWO\@auxout{\string\expandafter\string\gdef
    \string\csname\space#2End\string\endcsname{\@nameuse{#2End}}}%
    }%
    \fi
}
%    \end{macrocode}
%    \begin{macro}{\markStartFor}
%    \begin{macro}{\markEndtFor}
%    \begin{macro}{\markNumQsFor}
% We provide a user interface to the three macros defined above. The required
% parameter is a mark (a name that was used as the first argument of
% \cmd{\calcQsBtwnMarkers}).
%    \begin{macrocode}
\newcommand{\markStartFor}[1]{\@nameuse{#1Start}}
\newcommand{\markEndFor}[1]{\@nameuse{#1End}}
\newcommand{\markNumQsFor}[1]{\@nameuse{#1nQs}}
%    \end{macrocode}
%    \end{macro}
%    \end{macro}
%    \end{macro}
%    \end{macro}
%
% \subsection{Useful Commands to Write Problems}
%
%    \begin{macro}{\placeAtxy}
%    Use to place material within the solutions area that is visible to the student
%    on the test document. Syntax:
%    \begin{quote}\switchCats\ttfamily
%    \string\placeAtxy{\meta(x\_dim)}{\meta(y\_dim)}{\meta(content)}
%    \end(quote)
% \cs{placeAtxy} goes immediately after \verb!\end{solution}!
%    \begin{macrocode}
\newcommand{\placeAtxy}[3]{%
    \par\nointerlineskip
    \ifdisplayworkarea
        \ifeq@nosolutions\ifx\eq@insertverticalspace y%
            \ifvmode\makebox[0pt][l]{%
                \hspace*{-\parindent}\hspace*{#1}%
                    \raisebox{#2}[0pt][0pt]{#3}}\fi
        \fi\fi
    \fi
}
%    \end{macrocode}
%    \end{macro}
%    \begin{environment}{workarea}
% A work area is an environment used when we display vertical space such as when the
% \texttt{nosolutions} and \texttt{vspacewithsolns} options are in effect.
% \DescribeMacro{\workareasb}\cmd{\workareasb} is a save box used in the
% argument of \texttt{lrbox} to save the contents of the \texttt{minipage}.
%    \begin{macrocode}
\newsavebox{\workareasb}
\newenvironment{workarea}[2][\linewidth]
{%
    \edef\workareadepth{\if\currhideopt H0pt\else#2\fi}%
    \begin{lrbox}{\workareasb}\setlength{\eqetmplengtha}{#1}%
    \begin{minipage}[b][\workareadepth][t]{\eqetmplengtha}\vspace*{3pt}%
}%
{%
    \end{minipage}\end{lrbox}%
%    \end{macrocode}
% \texttt{4/5/11} Changed the logic here, so the \texttt{workarea} is available
% for the \texttt{vspace\-with\-solns} option.
%    \begin{macrocode}
    \par\ifdisplayworkarea
            \ifx\eq@insertverticalspace y%
            \vspace{-\baselineskip}\fi
            \ifx\eq@insertverticalspace y%
            \if\currhideopt H\else
                \noindent\strut\smash{\usebox{\workareasb}}%
            \fi\fi
    \fi
}
%    \end{macrocode}
%    \end{environment}
%    \begin{environment}{splitsolution}
% This is a special solution environment designed for use with the \texttt{online} or \texttt{email}
% options.
%    \begin{macrocode}
\newcommand\panelgap{3pt}
\newsavebox{\eqpanelbox}
\def\panelwidth{\the\wd\eqpanelbox}\def\panelheight{\the\ht\eqpanelbox}%
\newcounter{panel@cnt}
\newwrite\panel@write
\let\usepanelwidth=\relax
%    \end{macrocode}
% \DescribeMacro{panel} environment goes inside the \texttt{splitsolution}
% environment.
%    \begin{macrocode}
\newenvironment{panel}[2][l]
{%
    \gdef\ss@Argi{#1}\gdef\ss@Argiii{#2}% #1 r or l #2 width of panel
    \stepcounter{panel@cnt}%
    \immediate\openout \panel@write panel\thepanel@cnt.cut
    % need to manage the h, H, and global overrides.
    \if\currhideopt H%
        \eqe@IWO\panel@write{\vfill}%
        \immediate\closeout\panel@write
        \gdef\ss@Argii{0pt}%
    \else
        \if\currhideopt h\ifeq@solutionsafter\else
            \ifeq@globalshowsolutions\else
                \eqe@IWO\panel@write{\vfill}%
                \immediate\closeout\panel@write
                \gdef\ss@Argii{0pt}%
            \fi\fi
        \fi
    \fi
    \begingroup
    \let\verbatim@out=\panel@write
    \verbatimwrite
}
{%
    \endverbatimwrite
    \immediate\closeout\panel@write
    \endgroup
    \xdef\mp@Width{\ifeq@solutionsafter\noexpand\linewidth
        \else\noexpand\linewidth-\ss@Argiii-\panelgap\fi}%
}
\long\def\eqe@IW#1{%
    \ifeq@solutionsafter
        \let\eqe@next=\@empty
    \else
        \def\eqe@next{\eqe@IWO\verbatim@out{#1}}%
        \ifeq@nosolutions\else
            \if\currhideopt H%
                \gdef\ss@Argii{0pt}\let\eqe@next=\@empty
            \else
                \if\currhideopt h%
                    \ifeq@globalshowsolutions\else
                        \gdef\ss@Argii{0pt}\let\eqe@next=\@empty
                    \fi
                \fi
            \fi
        \fi
    \fi
    \eqe@next
}
\newenvironment{@ssSolution}[1][\ss@Argii]%
{%
    \xdef\eqe@depth{\ifeq@nosolutions\ss@Argii\else
        \ifeq@solutionsafter\ss@Argii\else0pt\fi\fi}%
    \xdef\mp@Width{\ifeq@solutionsafter\noexpand\linewidth\else
        \noexpand\linewidth-\ss@Argiii-\panelgap\fi}%
    \if\ss@Argi l%
        \ifeq@solutionsafter
            \def\eqe@lPanel{\parbox[b][\eqe@depth][t]{\ss@Argiii}
                {\input{panel\thepanel@cnt.cut}\vfill}\hfill}%
            \def\eqe@rPanel{\@empty}%
        \else
            \def\eqe@lPanel{\string\parbox[b][\string\eqe@depth]%
                {\ss@Argiii}%
                {\string\input{panel\thepanel@cnt.cut}\vfill}\hfill^^J}%
            \def\eqe@rPanel{\@empty}%
        \fi
    \else
        \ifeq@solutionsafter
            \def\eqe@lPanel{\@empty}%
            \def\eqe@rPanel{\hfill\parbox[b][\eqe@depth][t]{\ss@Argiii}
                {\hfill\input{panel\thepanel@cnt.cut}\vfill}}%
        \else
            \def\eqe@lPanel{\@empty}%
            \def\eqe@rPanel{\hfill\string\parbox[b][\string\eqe@depth]%
                {\ss@Argiii}%
                {\endgraf\string\noindent\hfill%
                    \string\input{panel\thepanel@cnt.cut}\vfill}}%
        \fi
    \fi
    \let\verbatim@out=\ex@solns
    \par\ifeq@solutionsafter\smallskip\fi\noindent\minipage{\linewidth}%
    \if\ss@Argi l\noindent\parbox[b][\eqe@depth][t]{\ss@Argiii}%
        {\vfill}\hfill\fi
    \minipage[b][\eqe@depth][t]{\mp@Width}%
    {\lccode`C=`\%\lowercase{\eqe@IW{\def\string\panelgap{\panelgap}%
        \edef\string\eqe@depth
            {\ifeq@nosolutions\ss@Argii\else\ss@Argii\fi}C}}}%
    {\lccode`C=`\%\lowercase{\eqe@IW{%
            \string\strut\string\par\string\nobreak\string\noindent%
                \string\hbox\space to\string\linewidth\bgroup^^J%
                \eqe@lPanel\string\minipage[b][\string\eqe@depth][t]%
                {\string\linewidth-\ss@Argiii-\panelgap}C
     }}}%
     \ifeq@solutionsafter
        \noindent\strut\hbox to\linewidth\bgroup
        \eqe@lPanel
        \minipage[b][\eqe@depth][t]{\linewidth-\ss@Argiii-\panelgap}%
        \def\solutionsafterSkip{}%
     \fi
     \@sssolution[#1]%
}{%
    \eqe@IW{\noexpand\endminipage\eqe@rPanel\egroup}%
    \ifeq@solutionsafter\endminipage\eqe@rPanel\egroup\fi
    \end@sssolution
    \endminipage\endminipage
}
\newenvironment{splitsolution}[1] %  #1 depth
{%
    \gdef\ss@Argii{#1}%
    \let\@sssolution = \solution
    \let\end@sssolution = \endsolution
    \let\solution = \@ssSolution
    \let\endsolution = \end@ssSolution
    \par\noindent\ignorespaces
}{%
    \ifeq@solutionsafter\strut\par\fi
    \edef\eqe@workwidth{\if\ss@Argi l\noexpand\ss@Argiii
        \else\noexpand\linewidth\fi}%
    \edef\eqe@workfill{\if\ss@Argi l\else\hfill\fi}%
    \begin{workarea}[\eqe@workwidth]{\eqe@depth}%
    \eqe@workfill\input{panel\thepanel@cnt.cut}%
    \end{workarea}
    \strut\par
}
%    \end{macrocode}
%    \end{environment}
% Redefine the \cs{paragraph} command
%    \begin{macrocode}
\renewcommand{\paragraph}
    {\@startsection{paragraph}{4}{0pt}{12pt}{-3pt}{\bfseries}}
%    \end{macrocode}
%    \begin{macro}{\defaultInstructions}
% For the \texttt{instructions} environment, defined next, the default string
% for the instructions is \cs{eq@default@Instructions}, this command is
% defined using \cs{defaultInstructions}.
%    \begin{macrocode}
\newcommand{\defaultInstructions}[1]{\def\eq@default@Instructions{#1}}
\defaultInstructions{Instructions.}
%    \end{macrocode}
%    \end{macro}
%
%    \begin{environment}{instructions}
% Each test, or a part of a test usually have instructions. This \texttt{instruction}
% environment is used in this purpose. Normally, the number of points for the part is
% displayed following the heading (the default is \textbf{Instructions.}). You can
% eliminate the total points from the instructions by taking the \texttt{nosummarytotals}
% option.
%
%    \begin{macrocode}
\newcommand{\beforeInstrSkip}{2.25ex}
\newcommand{\afterInstrSkip}{-1em}
\newenvironment{instructions}[1][\eq@default@Instructions]
{%
    \def\eq@argi{#1}%
%    \end{macrocode}
% (2011/05/08) We also replace \cs{paragraph*} with just \cs{textbf}, within
% a list environment \cs{paragraph*} causes problems.
%    \begin{macrocode}
    \@startsection{paragraph}{4}{\z@}%
    {\beforeInstrSkip\space\@plus1ex \@minus.2ex}%
    {\afterInstrSkip}{\normalfont\normalsize\bfseries}*%
    {\textcolor{\@instructionsColor}{#1}\normalcolor%
    \ifx\eq@nosummarytotals y\else\ifx\eq@argi\@empty\else\ \fi
        {\normalfont\summaryTotalsTxt}\ \fi}\hskip-\lastskip
    \ifx\eq@nosummarytotals y\ifx\eq@argi\@empty\else\ \fi\fi
    \ignorespaces}{}
%    \end{macrocode}
%    \begin{macrocode}
\newcommand\summaryPointTotal{\csname\thisexamlabel total\endcsname}
\newcommand\summaryTotalsTxt{($\summaryPointTotal\,\text{%
    \@ifundefined{\thisexamlabel total}{\eq@pointsLabel}%
    {\ifnum\summaryPointTotal=1\relax\eq@pointLabel\else
    \eq@pointsLabel\fi}}$)}
\newcommand{\nQuesInExam}[1][\thisexamlabel]{%
        \csname#1nQuestions\endcsname}
%    \end{macrocode}
%    \end{environment}
% Make this definition for \texttt{hyperref}, so its anchors will be unique.  Useful
% when there are multiple parts of the test.
%    \begin{macrocode}
\ifx\hyper@anchor\@undefined\else
    \renewcommand\theHeqquestionnoi
            {\curr@quiz.\theeqquestionnoi}
    \renewcommand\theHquizno{%
        \if\probstar*\curr@quiz.%
            \theeqquestionnoi.part\thepartno.\arabic{quizno}%
        \else
            \curr@quiz.%
                \theeqquestionnoi.\arabic{quizno}%
        \fi
    }
    \renewcommand\theHpartno{\curr@quiz.%
        \theeqquestionnoi.part\thepartno}
\fi
%    \end{macrocode}
%    \begin{environment}{eqComments}
% Often, I want to make additional instructions between problems, you can use
% this comment environment.
%    \begin{macrocode}
\newcommand{\beforeCommentSkip}{1.25ex}
\newcommand{\afterCommentSkip}{-1em}
\newenvironment{eqComments}[1][\strut]
{%
    \removelastskip\removelastskip\def\eqe@argi{#1}%
    \def\eqe@Strut{\strut}%
%    \end{macrocode}
% (2011/05/08) We also replace \cs{paragraph*} with just \cs{textbf}, within
% a list environment \cs{paragraph*} causes problems.
%    \begin{macrocode}
    \@startsection{paragraph}{4}{\z@}%
    {\beforeCommentSkip\space\@plus1ex \@minus.2ex}%
    {\afterCommentSkip}{\normalfont\normalsize\bfseries}*%
    {\textcolor{\@eqCommentsColor}{#1}}%
    \ifx\eqe@argi\eqe@Strut\hskip-\lastskip\else\space\strut\fi
    \color{\@eqCommentsColorBody}\ignorespaces
}{}
%    \end{macrocode}
%    \end{environment}
% \subsection{The \texttt{exam} Environment}
% Each part of the exam is enclosed in an \texttt{exam} environment. The environment is
% a customized version of the \texttt{shortquiz} environment.
%    \begin{macro}{\exambegdef}
% Some definitions that are executed at the beginning of each exam environment.
%    \begin{macrocode}
\let\tb@beginexam@code\relax
\def\partialspillovertotals{0}
\def\exambegdef
{%
    \csname\thisexamlabel pagemark\endcsname
    \@ifundefined{partialtotalpg}{}{%
        \ifnum\partialtotalpg=\arabic{page}%
            {\count0=\partialspillovertotals
             \advance\count0by\partialtotaleoe
             \xdef\partialspillovertotals{\the\count0}%
             \eqe@IWO\@auxout{\string\expandafter\string\gdef
                \string\csname\space%
                 Page\partialtotalpg spilltotal\string\endcsname
                    {\partialtotaleoe}}%
            }%
        \fi
    }
    \expandafter
    \ifx\csname\thisexamlabel pageno\endcsname\relax
    \else
        \expandafter
        \ifx\csname pagenofirstprob\thisexamlabel\endcsname\relax
        \else
            \ifnum\csname\thisexamlabel pageno\endcsname
                <\csname pagenofirstprob\thisexamlabel\endcsname
            \else
                \expandafter
                \ifx\csname\thisexamlabel pagemark\endcsname\relax
                \else
                    \ifnum\value{page}=%
                        \csname\thisexamlabel pageno\endcsname
                        \eqe@IWO\@auxout{%
                            \string\expandafter\string\gdef
                            \string\csname\space
                            \thisexamlabel pagemark\string\endcsname
                                {\string\newpage}}%
                    \fi
                \fi
            \fi
        \fi
    \fi
    \setcounter{eqquestionnoi}{0}\setcounter{eqpointvalue}{0}%
    \setcounter{eqpointsofar}{0}\setcounter{eqpointsthispage}{0}%
    \setcounter{eq@count}{0}%
%    \end{macrocode}
% We wrote \verb!\begin{eqequestions}! to the top of the solutions file (\cs{jobname.sol}.
%    \begin{macrocode}
    \writeBeginEqeQuestions
%    \end{macrocode}
%    \begin{macrocode}
    \label{\thisexamlabel PageBegin}%
    \eqe@IWO\@auxout{\string\expandafter\string\gdef
    \string\csname\space\thisexamlabel pageno\string\endcsname
        {\thepage}}%
%    \let\sq@priorhook\@empty
    \ifeqfortextbook
        \global\examenvtrue\tb@beginexam@code
    \fi
}
%    \end{macrocode}
%    \end{macro}
%    \begin{macro}{\examenddef}
% Some definitions that are executed at the end of each exam environment.
% We place a totals box to report the total since the last page.
%    \begin{macrocode}
\def\tb@insmargmark{\ifisinstred\ifismarginans
    \insMidMarg{\mark{}}\fi\fi}
\def\examenddef
{%
    \global\let\partialtotaleoe\relax
    \global\let\partialtotalpg\relax
    \expandafter\ifx\csname NumberOfParts\endcsname\relax
    \else
        \ifnum\value{eq@numparts}<\NumberOfParts
            \setcounter{eq@count}{\value{eqpointvalue}}%
            \addtocounter{eq@count}{-\value{eqpointsofar}}%
            \xdef\partialtotaleoe{\arabic{eq@count}}%
            \xdef\partialtotalpg{\arabic{page}}%
            \if\eq@parttotals y%
%    \end{macrocode}
% See if there is enough room at the bottom of the page to place the end of exam
% totals, if not, forget it, and start a new page.
%    \begin{macrocode}
                \@actionsAtPageBreak{\global\let\@spacetobreak1}%
                    {\global\let\@spacetobreak0}%
                \ifx\@spacetobreak0\relax
                    \bgroup\@tempdima=\pagetotal
                    \advance\@tempdima32pt\relax
                    \ifdim\@tempdima>\pagegoal\aftergroup\newpage
                    \else
                        \ifnum\arabic{eq@count}>0\relax
                            \ifx\@reportpoints1\else\@checkSpacing\fi
                                \textcolor{\endexamtotal@color}%
                                {\eqeomarginbox{\arabic{eq@count}}{0}}%
                        \fi
                        \ifx\eqx@separationrule y\separationrule\fi
                    \fi\egroup
                \fi
            \else\ifx\eqx@separationrule y\separationrule\fi
            \fi
        \fi
    \fi
    \ifeqfortextbook\global\examenvfalse\fi
%    \end{macrocode}
% (2011/05/08) Just before the file is closed and input, we write the end
% of the \texttt{eqequestions} environment, \verb!\end{eqequestions}!.
%    \begin{macrocode}
    \writeEndEqeQuestions
%    \end{macrocode}
%    \begin{macrocode}
    \writetotalstoaux
    \addtocounter{page}{-1}%
    \writelastpage[\thisexamlabel]\addtocounter{page}{1}%
    \ifeqfortextbook\tb@insmargmark\fi
}
\def\@actionsAtPageBreak#1#2{%
    \bgroup\@tempdima\pagegoal\advance\@tempdima-\pagetotal
    \@tempdimb\@fvsizeskip\vsize
    \ifdim\@tempdima < \@tempdimb #1\else #2\fi\egroup
}
%    \end{macrocode}
%    \end{macro}
%    \begin{macro}{\separationrule}
% For an exam with multiple parts, a separation rule is created, unless absorbed
% into a page break. The command \cs{separationrule} defines this separation rule,
% it can be redefined as desired.
% \changes{v1.7a}{2007/12/10}
%{
%   Added \cs{separationrule} so user's can redesign the separation rule
%   that is created between two parts of an exam.
%}
%    \begin{macrocode}
\newcommand{\separationrule}{\makebox[\linewidth-\eqemargin]%
    {\centering\rule{.67\linewidth}{.4pt}}}
%    \end{macrocode}
%    \end{macro}
% (2011/05/08) This is a new environment that makes an exam into a list of problems.
% This is an attempt to expand the use of \textsf{eqexam} to {\LaTeX}
% documents. We give control over the page layout so an eqexam document
% can be used within a textbook.
%    \begin{macrocode}
\def\eqe@hspannerPrb{\ }
\newif\ifwithsoldoc \withsoldocfalse
\newcommand{\eqequestopsep}[1]{\def\eqeques@topsep{#1}}
\newcommand{\eqequesparsep}[1]{\def\eqeques@parsep{#1}}
\newcommand{\eqequesitemsep}[1]{\def\eqeques@itemsep{#1}}
\eqequestopsep{3pt}
\eqequesparsep{0pt}
\eqequesitemsep{3pt}
\newenvironment{eqequestions}{%
    \begin{list}{}{%
    \setlength{\labelwidth}{\eqemargin}%
    \setlength{\topsep}{\eqeques@topsep}%
    \setlength{\parsep}{\eqeques@parsep}%
    \setlength{\itemsep}{\eqeques@itemsep}%
    \setlength{\itemindent}{0pt}
    \ifwithsoldoc\settowidth{\labelsep}{\eqe@hspannerSoln}\else
    \settowidth{\labelsep}{\eqe@hspannerPrb}\fi
    \setlength{\leftmargin}{\labelwidth}%
    }\item\relax}{\end{list}}
%    \end{macrocode}
%    \begin{environment}{exam}
% Each part of the exam is enclosed in an \texttt{exam} environment. The one
% required parameter is the name of the part, for example, `Part1', `Part2'.
% These should be one word, no white spaces, just letters and possibly numbers.
%    \begin{macrocode}
\def\setDefaultfvsizeskip#1{\def\default@fvsizeskip{#1}%
    \def\@fvsizeskip{#1}}
\def\default@fvsizeskip{.3}
\edef\@fvsizeskip{\default@fvsizeskip}
\def\fvsizeskip#1{\def\@fvsizeskip{#1}}
\def\autoExamName{exam\the\value{eq@numparts}}
\def\nNumberOfP@rts{\csname NumberOfParts\endcsname}
\newenvironment{exam}[2][]
{%
    \makeRoomForProb{\@fvsizeskip\textheight}{0}%
%    \xdef\@fvsizeskip{\default@fvsizeskip}%
    \stepcounter{eq@numparts}%
    \xdef\thisexamlabel{#2}\xdef\curr@quiz{#2}%
    \edef\eq@tmp{\the\partNames}%
    \global\partNames=\expandafter{\eq@tmp\\{#2}}%
    \expandafter\ifx\csname NumberOfParts\endcsname\relax\else
    \ifnum\nNumberOfP@rts=1\relax\else
    \def\eqe@argi{#1}\ifx\eqe@argi\@empty
        \eqe@writetoSolns{#2}\eqe@writetoAux{%
            \string\expandafter\string\gdef
            \string\csname\space userFriendly#2\string\endcsname{#2}}
        \else
        \eqe@writetoSolns{#1}\eqe@writetoAux{%
            \string\expandafter\string\gdef
            \string\csname\space userFriendly#2\string\endcsname{#1}}
        \fi
    \fi\fi
    \exambegdef
    \expandafter\shortquiz\sqstar[#2]%
}{%
    \examenddef
    \endshortquiz
    \par\penalty-1000\vskip0pt
}
%    \end{macrocode}
%    \end{environment}
%    \begin{macro}{\EQEcalculateAllTotals}
% (4/22/11) Added the command \cs{EQEcalculateAllTotals}. The command is executed
% as part of the \cs{maketitle} command. If \cs{maketitle} is not used for some reason
% \cs{EQEcalculateAllTotals} can be executed just after \verb~\begin{document}~.
%    \begin{macrocode}
\newcommand{\EQEcalculateAllTotals}{%
    \begingroup
%    \end{macrocode}
% We calculate the grand total of all the parts of the \texttt{exam} environments,
% and we define \cs{eqeGrandTotal}, which contains the total.
%    \begin{macrocode}
        \count0=0\relax
        \def\\##1{\expandafter
            \ifx\csname##1total\endcsname\relax\else
            \advance\count0by\csname##1total\endcsname
%    \end{macrocode}
% \cs{thePartNames} list all named exam environments in the document, e.g.,
%\begin{verbatim}
%   \\{Part1}\\{Part2}...\\{LastPart}
%\end{verbatim}
%    \begin{macrocode}
        \fi}\csname thePartNames\endcsname
        \xdef\eqeGrandTotal{\the\count0 }%
        \ifnum\eqeGrandTotal=0 \else
%    \end{macrocode}
% If there is a nonzero grandtotal, we move on to calculate
% the percentages.
%    \begin{macrocode}
        \def\\##1{\eqe@calc@percent{##1}}%
        \csname thePartNames\endcsname\fi
    \endgroup
}
%    \end{macrocode}
%
%    \begin{macro}{\eqe@calc@percent}
% We go through the parts listed in \cs{thePartNames} and create a
% calculation of the percentage for that part, and leave it in
% \verb!\csname#1percent\endcsname!, which can be accessed
% through the \cs{percentForPart} command, for example
% \verb!\percentForPart{<part_name>}! might expand to \texttt{45.6\%}.
%    \begin{macro}{\nPctDecPts}
% The number of decimal points to carry in the representation of the
% percentage.
%    \begin{macrocode}
\newcommand{\nPctDecPts}{1}
\def\eqe@calc@percent#1{\@ifundefined{#1total}{%
        \expandafter\gdef\csname#1percent\endcsname{??}}{%
%    \end{macrocode}
% If the \textsf{fp} package is not loaded, we use register arithmetic,
% percentages are truncated to integers.
%    \begin{macrocode}
        \expandafter\ifx\csname FPdiv\endcsname\relax
        \count2=\totalForPart{#1}%
        \edef\expGT{\csname eqeGrandTotal\endcsname}%
        \multiply\count2by100\relax\divide\count2by\expGT\relax
        \expandafter\xdef\csname#1percent\endcsname{\the\count2 }\else
%    \end{macrocode}
% If the \textsf{fp} package is loaded, we use this package to calculate
% the percentage, accurate to one decimal place.
%    \begin{macrocode}
        \FPdiv{\eqe@pForPart}{\csname#1total\endcsname}%
            {\csname eqeGrandTotal\endcsname}%
        \FPmul{\eqe@pForPart}{\eqe@pForPart}{100}%
        \FPround{\eqe@pForPart}{\eqe@pForPart}{\nPctDecPts}%
        \expandafter\xdef\csname#1percent\endcsname{\eqe@pForPart}\fi
    }%
}
%    \end{macrocode}
%    \end{macro}
%    \end{macro}
%    \end{macro}
%    \begin{macrocode}
\def\writetotalstoaux{%
    \eqe@IWO\@auxout{\string\expandafter\string\gdef
    \string\csname\space\thisexamlabel total\string\endcsname
        {\theeqpointvalue}}%
    \eqe@IWO\@auxout{\string\expandafter\string\gdef
    \string\csname\space\thisexamlabel nQuestions\string\endcsname
        {\theeqquestionnoi}}%
}
\newcommand{\writelastpage}[1][]{\def\eqe@argi{#1}%
   \ifx\eqe@argi\@empty\else\label{#1PageEnd}\fi
   \eqe@IWO\@auxout{\string\expandafter\string\gdef
   \string\csname\space eqExamLastPage\string\endcsname{\arabic{page}}}%
}
\def\exlabel{}
\def\sqlabel{}
\def\exsolafter{\textit{Solution}:}
\def\sqsolafter{\textit{Solution}:}
%    \end{macrocode}
% The exercise labels in the body of the text
%    \begin{macrocode}
\def\exlabelformat{\textbf{\theeqquestionnoi.\ }}
\def\exlabelformatwp{\exlabelformat}%
%    \end{macrocode}
% The exercise labels for solutions at the end of the document
%    \begin{macrocode}
\def\exsllabelformat
    {\string\llap{\string\textbf{\theeqquestionnoi.\ }}}
\def\exsllabelformatwp
     {\string\llap{\string\textbf{\theeqquestionnoi.\ }}(\thepartno)\ }%
\ifeq@solutionsafter
    \def\exrtnlabelformat{}
    \def\exrtnlabelformatwp{}
    \def\eq@sqslrtnlabel{}
\else
    \def\exrtnlabelformat{$\square$}
    \def\exrtnlabelformatwp{$\square$}
    \def\eq@sqslrtnlabel{$\square$}
\fi
\def\sqslrtnlabel{\eq@sqslrtnlabel}
%    \end{macrocode}
% (2010/08/21) Enable some localizations of strings
%    \begin{macrocode}
\newcommand{\exsectitletext}{Solutions to \webtitle}
\def\exsectitle{\normalsize\exsectitletext}
%\def\exsectitle{\normalsize\hspace*
%    {-\oddsidemargin}\exsectitletext}
\@ifpackageloaded{exerquiz}{%
    \renewcommand{\exsecrunhead}{Solutions to \websubject}%
}{\newcommand{\exsecrunhead}{Solutions to \websubject}}
%\providecommand{\exsecrunhead}{Solutions to \websubject}%
\def\eq@sqslsectitle{}
\def\eq@sqslsecrunhead{}
\def\eq@sqsllabel{{\string\llap{\string\textbf{\theeqquestionnoi.\ }}}}
\def\eq@sqlabel{}
\let\include@quizsolutions=\relax
\let\solnhspace\@empty
%    \end{macrocode}
% \subsection{\texttt{problem} Environments}
% A single question is posed with the \texttt{problem} environment, and a question with
% multiple parts with the \texttt{problem*} environment.
%    \begin{macro}{\fillin}
% This macro is used for fill-in type questions. The first argument is the length
% of the underline blank to leave to fill-in, the second argument is the correct answer.
%    \begin{macrocode}
\newcommand{\fillin}[3][u]{%
    \ifx#1u\let\@fillinFmt\underbar
    \else\ifx#1b\let\@fillinFmt\relax
    \else\let\@fillinFmt\relax\fi\fi
    \ifeq@proofing
        \@fillinFmt{\makebox[#2]{%
            \strut\hfil\bfseries\color{red}#3\hfil}}%
    \else
        \@fillinFmt{\makebox[#2]{\strut\hfil}}%
        \@ifundefined{@quiz}{}{%
            \ifx\eq@online y\relax
                \ifeq@nosolutions
                    \ifeq@solutionsafter\else
                        \ifx\eq@insertverticalspace y\relax
                            \stepcounter{@cntfillin}%
                            \edef\fieldName{%
                                \if\probstar*eqexam.\curr@quiz.fillin.%
                                    \theeqquestionnoi.part\thepartno.%
                                    fi\the@cntfillin
                                \else
                                    eqexam.\curr@quiz.fillin.%
                                    \theeqquestionnoi.fi\the@cntfillin
                                \fi
                            }\makebox[0pt][r]{\textField[\BC{}]{%
                                \fieldName}{#2}{11bp}}%
                        \fi
                    \fi
                \fi
            \fi
        }%
    \fi\space\ignorespaces}
%    \end{macrocode}
%    \end{macro}
%    \begin{macro}{\TF}
% A specialized version of \cs{fillin} for True/False questions.
%    \begin{macrocode}
\newcommand\defaultTFwidth{30pt}
\newcommand\TF[2][\defaultTFwidth]{%
    \def\eqe@next{\fillin{#1}{#2}}%
    \ifdim\eq@extralabelsep=0pt\relax\else
        \if\probstar*\relax\if\exerwparts@cols x
            \def\eqe@next{\makebox[0pt][r]{%
                \fillin{#1}{#2}}\ignorespaces}%
    \fi\fi\fi
\eqe@next}
\def\fillinWidth#1{%
    \if\probstar*
        \settowidth{\eq@tmplengthA}{\normalfont\ }%
        \addtolength{\eq@tmplengthA}{#1}%
        \edef\eq@extralabelsep{\the\eq@tmplengthA}%
    \fi
}
%    \end{macrocode}
%    \end{macro}
%    \begin{macro}{\Do<num>}
% The following commands supports the optional argument \texttt{{\string\Do<num>}}.
% When I teach senior or graduate-level classes, I often give a problem with
% multiple parts (each of equal value) and ask them to ``do 3 of the following 5''
% parts.
%    \begin{macrocode}
\def\makeDoNum#1{\xdef\nDoNum{#1}%
    \def\ifc@sewrap{\ifcase#1??\or}%
    \xdef\DoNum{\expandafter\ifc@sewrap\eqe@wordNums\else
    \eqe@wordNumbsError\fi}}
\def\makeOutOfNum#1{\xdef\nOutOfNum{#1}%
    \def\ifc@sewrap{\ifcase#1??\or}%
    \xdef\OutOfNum{\expandafter\ifc@sewrap\eqe@wordNums\else
    \eqe@wordNumbsError\fi}}
%    \end{macrocode}
% \DescribeMacro{\eqe@wordNums} is used to typeset the English word for
% the numbers (1--10). This command may be redefined to other languages.
%    \begin{macrocode}
\newcommand{\eqe@wordNums}{one\or two\or three\or
    four\or five\or six\or seven\or eight\or nine\or ten}
\newcommand{\eqe@wordNumbsError}{\noexpand\PackageError{eqexam}%
    {Number out of range, 1--10}%
    {Use a smaller number, or redefine the command
    \string\eqe@wordNums.}}
%    \end{macrocode}
%    \end{macro}
%    \begin{macrocode}
\def\makeRoomForProb#1#2{\par %\endgraf % dps 11/11/10
    \bgroup\@nobreakfalse\addpenalty{-500}%
    \setlength{\@tempdimb}{#1}%
    \@tempdima \pagegoal \advance \@tempdima -\pagetotal
    \ifdim \@tempdima<\@tempdimb\ifnum\col@number>\@ne\columnbreak
        \else\newpage\fi\fi\egroup
    \ifx1#2\ifnum\@reportpoints>1 \@checkSpacing\fi\fi
}
%    \end{macrocode}
%    \begin{macro}{\promoteNewPage}
%\changes{v2.0n}{2011/05/13}{%
% A simple variation on \cs{makeRoomForProb} designed for user use.
%}
% A simple variation on \cs{makeRoomForProb} designed for user use.
%    \begin{macrocode}
\newcommand{\promoteNewPage}[1][\@fvsizeskip\textheight]{%
    \makeRoomForProb{#1}{0}}
%    \end{macrocode}
%    \end{macro}
%    \begin{macrocode}
\def\@checkSpacing{\bgroup
    \@tempdima = \lastPageTotal
    \@tempdimb = \pagetotal
    \ifdim\@tempdima < \@tempdimb
        \advance\@tempdimb by-\@tempdima
        \ifdim\@tempdimb < \eq@pointboxtotalheight
            \@tempdima=\eq@pointboxtotalheight
            \advance\@tempdima3pt\relax
            \advance\@tempdima by-\@tempdimb
            \vspace*{\@tempdima}%
         \fi
    \fi
\egroup}
\def\default@nbaselineskip{6}
\edef\@nbaselineskip{\default@nbaselineskip}
\def\nbaselineskip#1{\def\@nbaselineskip{#1}}
%    \end{macrocode}
%    \begin{environment}{problem}
% The \texttt{problem} is used to pose a single---non-multi-part---question.
% The optional argument is the number of points for this problem.
%    \begin{macrocode}
\def\@gobbletoend#1\end{}
\def\@grabarg#1\end{\def\numpoints{#1}}
\newenvironment{problem}[1][]{%
    \makeRoomForProb{\@nbaselineskip\baselineskip}{1}%
%    \xdef\@nbaselineskip{\default@nbaselineskip}%
    \gdef\probstar{x}\let\afterlabelhskip\@empty
    \ifx\marginpoints\@empty\else\def\numpoints{#1}%
        \ifx\numpoints\@empty\let\marginpoints\@empty\else
%    \end{macrocode}
%\changes{v2.0n}{2011/05/13}{%
% Added \texttt{*<num>} to signal in-line display of points.
%}
% (2011/5/13) We add a \texttt{*} feature. When the author types
% \texttt{[*3]}, it is a three point problem, but the value is expressed
% in-line, not in the margins.
%    \begin{macrocode}
        \@ifstar{\let\@isitstar=1\@grabarg}%
            {\let\@isitstar=0\@gobbletoend}#1\end
        \if\@isitstar1\addtocounter{eqpointvalue}{\numpoints}%
        \@marktotalvalue
        \def\marginparafterhook{\PTs{\numpoints}\space}\else
        \def\marginparpriorhook{\noindent\probvalue{#1}{0}}\fi\fi
    \fi\setcounter{eq@count}{\value{eqquestionnoi}}%
    \addtocounter{eq@count}{1}%
    \ifnum\value{eq@count}=1\relax
        \eqe@IWO\@auxout{\string\expandafter\string\gdef\string
        \csname\space pagenofirstprob\thisexamlabel\string\endcsname
        {\thepage}}%
    \fi
    \proofingsymbol{\ding{52}}%
    \begin{eqequestions}%
    \begin{exercise}[eqquestionnoi]%
}{\end{exercise}%
  \end{eqequestions}%
    \ifeqlocalversion\ifeqglobalversion
        \xdef\eqe@tmp{\noexpand\forVersion{\eq@selectedVersion}}%
        \aftergroup\eqe@tmp
    \fi\fi
    \global\eqlocalversionfalse
}
%    \end{macrocode}
%    \end{environment}
%    \begin{macro}{\PTs}
%    \begin{macro}{\itemPTsTxt}
%    \begin{macro}{\itemPTsFormated}
%  When you specify \cs{auto} for the optional argument of the \texttt{problem*}
%  environment, when each item must have the command \cs{PTs} to assign the
%  value of that question.  The \cs{PTs} has one optional star-parameter, and one
%  required parameter. The required parameter is the number of points for this item,
%  if the \texttt{*} is specified, then the point value is not typeset in the document.
%
%  The command \cs{itemPTsTxt} has one argument, the number of points for this item. This
%  argument is passed from the \cs{PTs} command.  You can redefine the way the points appear
%  in the document using \cs{itemPTsTxt}.  As separate command \cs{itemPTsFormated} is used to
%  put parentheses around \cs{itemPTsTxt}. If the \texttt{*} option is taken with \cs{PTs}, then
%  you are free to place \cs{itemPTsTxt} anywhere in the problem statement.
% \changes{v1.6g}{2006/11/29}
%{
% Added \cs{itemPTsTxt} and \cs{itemPTsFormated} to work with \cs{PTs}.
% Also added a \texttt{*} option, to \cs{PTs}, in this case the points
% are not typeset.
%}
%    \begin{macrocode}
\newcommand\itemPTsTxt[1]{$#1\,\text{%
    \ifnum#1=1\relax\eqptLabel\else\eqptsLabel\fi}$}
\newcommand\itemPTsEaTxt[1]{$#1\,\text{%
    \ifnum#1=1\relax\eqptLabel\else\eqptsLabel\fi\space\eq@eachLabel}$}
\newcommand{\itemPTsFormated}[1]{(#1)}
\def\PTs{\@ifstar{\@PTs{*}}{\@PTs{x}}}
\def\@PTs#1#2{%
    \if\@reportpoints0\else
        \if\eqe@pointsPartsId1%
        \addtocounter{eqpointvalue}{#2}\@marktotalvalue
        \addtocounter{eq@count}{#2}\fi\if#1*\else
%    \end{macrocode}
%    (2012/04/26) Wrapped |\itemPTsFormated{\itemPTsTxt{#2}}| as the argument
%    of \cs{eqe@movePTs}. \cs{eqe@movePTs} does nothing by default, but may be
%    redefined, for example, to place the value of each part on the margin.
%    The default definition of \cs{eqe@movePTs} follows.
%    \begin{macrocode}
        \eqe@movePTs{\itemPTsFormated{\itemPTsTxt{#2}}}\fi
    \fi
}
\def\eqe@movePTs#1{#1}
%    \end{macrocode}
%    \end{macro}
%    \end{macro}
%    \end{macro}
% \DescribeMacro{\Do<num>} The \cs{isItD@} tests to see if the next token
% is \cs{Do}, if yes, it marks it and calls \cs{y@st@Do}, which gets the argument
% if the \cs{Do} token.
%    \begin{macrocode}
\let\auto\relax
\def\isItD@{\@ifnextchar\Do{\let\yest@D@=y\y@st@Do}
    {\let\yest@D@=n\@gobblet@end}}
\let\yest@D@=n
\def\y@st@Do\Do#1\end{\gdef\D@Num{#1}}
\def\@gobblet@end#1\end{}
%    \end{macrocode}
%
%    \begin{environment}{problem*}
%
% The \texttt{problem*} environment is used to pose a multi-part question.
% The \texttt{parts} environment is used to enumerate the parts.
%\changes{v2.0c}{2011/01/11}{%
%    Changed \cs{@next} to \cs{eqe@next}. There was conflict in
%    the use of this command with one of the float environments. When user
%    used the table environment inside the problem* environment, the
%    compiled stopped because \cs{@next} was overwritten.
%}
%    \begin{macrocode}
%    \end{macrocode}
% We create a Id for the points specified by the first (and second) optional parameters:
% 0 (total points specified); 1 (\cs{auto} specified); 2 (points each
% specified); 4 (\cs{Do} second optional parameter); a value of \cs{relax} means
% no points specified (the default).
%    \begin{macrocode}
\let\eqe@pointsPartsId\relax
%    \end{macrocode}
% We now begin the code for the \texttt{problem*} environment.
%    \begin{macrocode}
\expandafter\def\csname problem*\endcsname{%
    \@ifnextchar[{\pr@bl@m@star}{\pr@bl@m@star[]}}
\def\pr@bl@m@star[#1]{%
    \@ifnextchar[{\pr@blem@star{#1}}{\pr@blem@star{#1}[]}}
\def\pr@blem@star#1[#2]{%
    \makeRoomForProb{\@nbaselineskip\baselineskip}{1}%
    \proofingsymbol{\ding{52}}\gdef\probstar{*}%
    \gdef\pr@b@secondarg{#2}\setcounter{eq@count}{0}%
    \let\afterlabelhskip\@empty
    \global\let\probpointseach\@empty\def\numpoints{#1}%
    \@ifstar{\let\@isitstar=1\@grabarg}%
        {\let\@isitstar=0\@gobbletoend}#1\end
%    \end{macrocode}
% Check for the \cs{auto} keyword
%    \begin{macrocode}
    \expandafter\ifx\numpoints\auto\let\eqe@pointsPartsId1%
        \global\let\probpointseach\relax
%    \end{macrocode}
% The author has requested \cs{auto}
%    \begin{macrocode}
        \def\eqe@next{\autocalcparts}%
    \else
        \ifx\pr@b@secondarg\@empty\else
        \let\eqe@pointsPartsId4%
%    \end{macrocode}
% Second Optional Argument not empty
%    \begin{macrocode}
            \setcounter{eq@count}{\value{eqquestionnoi}}%
            \addtocounter{eq@count}{1}%
                \@ifundefined{nPartsThisProb\thisexamlabel.\theeq@count}
                {\makeOutOfNum{0}\makeDoNum{0}}{%
                    \expandafter\makeOutOfNum{%
                        \csname nPartsThisProb\thisexamlabel.%
                            \theeq@count\endcsname}%
                    \expandafter\makeDoNum{%
                        \csname DoNumThisProb\thisexamlabel.%
                            \theeq@count\endcsname}%
                }%
        \fi
        \def\eqe@next{\manualcalcparts{\numpoints}}%
    \fi\eqe@next
    \ifeqfortextbook
    \writeToSolnFile{\global\protect\frstProbNumShownfalse}\fi
    \begin{eqequestions}%
    \begin{exercise}[eqquestionnoi]*}%
%    \end{macrocode}
%    \begin{macrocode}
\def\ftb@endprobstarCks{%
    \ifWithinANSGrp
        \PackageError{eqexam}{\string\bGrpANS\space is still open}
        {You need to match it with an \string\bGrpANS,
         or remove it.}%
    \fi
}
%    \end{macrocode}
%\DescribeMacro{\endproblem*} begins here.
%    \begin{macrocode}
\expandafter\def\csname endproblem*\endcsname{%
    \eqe@IWO\@auxout{\string\expandafter\string\gdef\string
    \csname\space nPartsThisProb\thisexamlabel.\theeqquestionnoi
    \string\endcsname{\arabic{partno}}}%
    \ifx\probpointseach\@empty\else
    \ifx\probpointseach\auto
        \eqe@IWO\@auxout{%
            \string\expandafter\string\gdef\string
            \csname\space prob\thisexamlabel.\theeqquestionnoi
            \string\endcsname{\theeq@count}}%
    \else
        \setcounter{eq@count}{\value{partno}}%
        \ifx\pr@b@secondarg\@empty\else
            \bgroup\toks0=\expandafter{\pr@b@secondarg}%
            \expandafter\isItD@\the\toks0 \end
%    \end{macrocode}
% If there is a |\Do|, we write this info to AUX.
%    \begin{macrocode}
            \ifx\yest@D@ y
                \eqe@IWO\@auxout{%
                    \string\expandafter\string\gdef\string
                    \csname\space DoNumThisProb\thisexamlabel.%
                    \theeqquestionnoi\string\endcsname{\D@Num}}%
                \@tempcnta = \value{eq@count}%
                \advance\@tempcnta -\D@Num
                \global\advance\value{eq@count}-\@tempcnta
% 3.0k
                \@tempcnta=\value{partno}%
                \advance\@tempcnta -\D@Num
                \multiply\@tempcnta by\argi
                \addtocounter{eqpointvalue}{-\@tempcnta}%
% end
            \fi
            \egroup
        \fi
        \multiply\value{eq@count}\argi
        \eqe@IWO\@auxout{%
            \string\expandafter\string\gdef\string
            \csname\space prob\thisexamlabel.\theeqquestionnoi
            \string\endcsname{\theeq@count}}%
%        \bgroup
%        \egroup
    \fi\fi
    \end{exercise}%
    \end{eqequestions}%
    \ifeqfortextbook\ftb@endprobstarCks\fi
    \ifeqlocalversion\ifeqglobalversion
        \xdef\eqe@tmp{\noexpand\forVersion{\eq@selectedVersion}}%
        \aftergroup\eqe@tmp
    \fi\fi
    \global\eqlocalversionfalse
}
%    \end{macrocode}
%    \end{environment}
%    \begin{macro}{\pushProblem}
%    \begin{macro}{\popProblem}
% There may be an occasion when a multi-part question needs to be broken between parts.
% use the \cs{pushProblem} and \cs{popProblem} for this purpose. The push saves the
% counter value, and ends the \texttt{parts} environment. The pop restarts the
% \texttt{parts}, and resets the parts counter.
% \changes{1.6b}{2006/3/12}
% {
%   Added \cs{pushProblem} and \cs{popProblem} to grant the ability to interrupt
%   a parts environment for, for example, a multicolumn environment.
% }
%\par\medskip\noindent
%In the example below, we have our parts in a \texttt{multicols} environment, we
%\cs{pushProblem}, close \texttt{multicols}, \cs{popProblem} and continue with
% the multi-parts in single column.
%\begin{verbatim}
%\item Compute $\lim_{x\to2^{\text{$-$}}} f(x)$
%\begin{solution}[1in]\end{solution}
%\pushProblem
%\end{multicols}
%\popProblem
%\item What value(s) of $c$ make the function $f$
%continuous at $x=2$?
%\begin{solution}[.5in]\end{solution}
%\end{parts}
%\end{verbatim}
%    \begin{macrocode}
\def\pushProblem{\xdef\nlastItem{\arabic{partno}}\end{parts}}
\def\popProblem{\begin{parts}\setcounter{partno}{\nlastItem}}
%    \end{macrocode}
%    \end{macro}
%    \end{macro}
%    \begin{macrocode}
\def\lastPageTotal{0pt}
\def\marginparafterhook{\xdef\lastPageTotal{\the\pagetotal}}
%    \end{macrocode}
% \DescribeMacro{\manualcalcparts} is the command calculates points
% when the argument is \emph{not} \cs{auto}. The macro \cmd{\prob@Arg}
% determines if the points passed are of the form \texttt{<num>ea}.
%    \begin{macrocode}
\def\prob@Arg#1ea#2\end{\def\argi{#1}\def\argii{#2}}
%    \end{macrocode}
% Now begin \cs{manualcalcparts}; \texttt{\#1} is the number of points, which may be
% of the form \texttt{<num>ea}, or just \texttt{<num>}.
%    \begin{macrocode}
\def\manualcalcparts#1{%
    \expandafter\prob@Arg#1ea\end
    \ifx\argii\@empty\edef\numpoints{#1}%
%    \end{macrocode}
% Total points specified, we should ignore any \cs{PTs} commands.
%    \begin{macrocode}
        \let\eqe@pointsPartsId0%
    \else\gdef\probpointseach{x}\let\eqe@pointsPartsId2%
%    \end{macrocode}
% Points each specified
%    \begin{macrocode}
        \setcounter{eq@count}{\value{eqquestionnoi}}%
        \addtocounter{eq@count}{1}\expandafter
         \ifx\csname prob\thisexamlabel.\theeq@count\endcsname\relax
            \def\numpoints{\argi}\else
            \def\numpoints{\expandafter
                \csname prob\thisexamlabel.\theeq@count\endcsname}%
        \fi
    \fi
    \ifx\marginpoints\@empty
%    \end{macrocode}
% no points for this exam
%    \begin{macrocode}
    \else
        \ifx\argi\@empty
%    \end{macrocode}
% no points for this problem specified
%    \begin{macrocode}
            \let\marginpoints\@empty
        \else
%    \end{macrocode}
% points to be displayed in margins or inline, if \cs{NoPoints} not specified
%    \begin{macrocode}
            \ifx\argii\@empty
%    \end{macrocode}
% \paragraph*{Total points specified}
%    \begin{macrocode}
                \ifx\marginpoints\@empty\else
                    \if\@isitstar1%
%    \end{macrocode}
% Points to appear ``in-line'' rather than in the margins
%    \begin{macrocode}
                        \addtocounter{eqpointvalue}{#1}%
                        \@marktotalvalue
                        \def\marginparafterhook{\itemPTsFormated{%
                            \itemPTsTxt{\numpoints}}\space}\else
                        \def\marginparpriorhook{\noindent
                            \probvalue{\numpoints}{0}}%
                    \fi
                \fi
            \else
%    \end{macrocode}
% \paragraph*{Points each specified}
%    \begin{macrocode}
                \ifx\marginpoints\@empty\else
                    \if\@isitstar1%
%    \end{macrocode}
% Points to appear ``in-line'' rather than in the margins
%    \begin{macrocode}
                        \def\marginparafterhook{%
                        \itemPTsFormated{\itemPTsEaTxt{\argi}}\space}%
                    \else
                        \def\marginparpriorhook{\noindent
                           \marginpoints{\numpoints}{\argi}}%
                    \fi
                    \edef\eqp@rtc@lcm@rk{\noexpand
                    \addtocounter{eqpointvalue}{\argi}%
                        \noexpand\@marktotalvalue}%
                \fi
            \fi
        \fi
    \fi
    \ifnum\value{eq@count}=1\relax
        \eqe@IWO\@auxout{\string\expandafter\string\gdef\string
        \csname\space pagenofirstprob\thisexamlabel\string\endcsname
            {\thepage}}%
    \else\goodbreak\fi
}
%    \end{macrocode}
% \DescribeMacro{\autocalcparts} is the command that computes the total points when
% the author specifies \cs{auto} as the optional argument of \texttt{problem*}. The
% commands \cmd{\acp@mpah} and \cmd{\acp@mpph} were recently (2012/04/21) separated
% out to allow for additional customization, without re-defining the whole of
% \cs{autocalcparts}.
%    \begin{macrocode}
\def\acp@mpah{\itemPTsFormated{\itemPTsTxt{\numpoints}}\space}
\def\acp@mpph{\noindent\marginpoints{\numpoints}{0}}
\def\autocalcparts{%
    \setcounter{eq@count}{\value{eqquestionnoi}}%
    \addtocounter{eq@count}{1}%
    \expandafter
    \ifx\csname prob\thisexamlabel.\theeq@count\endcsname\relax
        \def\numpoints{0}% assume zero points until we get the total
    \else
        \edef\numpoints{\expandafter
            \csname prob\thisexamlabel.\theeq@count\endcsname}%
    \fi
    \ifx\marginpoints\@empty\else
        \if\@isitstar1%
%    \end{macrocode}
% If we have \texttt{*\cs{auto}}, the total is to appear inline.
%    \begin{macrocode}
            \def\marginparafterhook{\acp@mpah}\else
%    \end{macrocode}
% Otherwise, the total will appear in the margin.
%    \begin{macrocode}
            \def\marginparpriorhook{\acp@mpph}\fi
    \fi
    \setcounter{eq@count}{0}%
}
%    \end{macrocode}
%    \begin{macro}{\forproblem}
%    \begin{macro}{\foritem}
% When typing solutions from assigned problems in a textbook, the problems
% assigned are not consecutive.  You can set the problem number before the problem
% environments by using the \cs{forproblem} command. The one required argument is
% the problem number: \verb!\forproblem{10)!.
% \changes{v1.6h}{2007/01/24}
% {
%   Added \cs{forproblem}, \cs{foritem}, \cs{aNewPage}
% }
%    \begin{macrocode}
\newcommand{\forproblem}[1]{\setcounter{eqquestionnoi}{#1 - 1}}
%    \end{macrocode}
% A similar comment for \cs{item}. These are useful for making out solution
% sets to homework assignments where problems are assigned from the textbook
% and you want to give a solution to problem 12, part (b),  An example of usage is
%\begin{verbatim}
%\forproblem{12}
%\begin{problem*}
%Factor each.
%\begin{parts}
%   \foritem{b} $ x^2 + 2x + 1 = ( x + 1 )^2 $
%   \item       $ x^2 - x - 2 = ( x - 2 )( x + 1) $ % this is part (c)
%   \foritem{e} $ x^2 + 7x + 10 = ( x - 2 )( x + 7) $
%   \item ...   % this will be part (f)
%\end{parts}
%\end{problem*}
%\end{verbatim}
%    \begin{macrocode}
\newcommand{\foritem}[1]{%
    \setcounter{partno}{0}%
    \lowercase{\def\eq@selectedItem{#1}}%
    \let\eq@initLoop=0
    \loop
        \stepcounter{partno}%
        \expandafter\if\alph{partno}\eq@selectedItem
            \let\eq@initLoop=1
        \fi
    \if\eq@initLoop0\repeat
    \addtocounter{partno}{-1}%
    \item
}
%    \end{macrocode}
%    \end{macro}
%    \end{macro}
% The command \cs{eqe@insertContAnnot} attempts to insert a string just prior
% to a part, if that part begins a new page. To get it right, it promotes a new
% page using the default of .25in. The optional parameter allows you to insert
% a new value; this may be needed to get the string \cs{annotContStr} placed properly.
% The commands
%    \begin{macro}{\annotContStr}
% \changes{v3.0l}{2011/08/22}{Defined \cs{eqe@insertContAnnot} and related commands}
% The string that is typeset by the \cs{eqe@insertContAnnot} command.
%    \begin{macro}{\acvspace}
% User access to changing the vertical spacing \cs{promoteNewPage} uses within
% \cs{eqe@insertContAnnot}.
%    \begin{macro}{\resetacvspace}
% Resets the vertical spacing back to its default.
%    \begin{macrocode}
\newcommand{\annotContStr}{%
    \textbf{Problem~\eqeCurrProb\space continued.}}
\newcommand{\acvspace}[1]{\def\ic@vspace{#1}}
%    \end{macrocode}
% \DescribeMacro{\ic@vspacedefault} is the default vertical spacing used
% by \cs{eqe@insertContAnnot}
%    \begin{macrocode}
\newcommand{\ic@vspacedefault}{.25in}
\newcommand{\resetacvspace}{\edef\ic@vspace{\ic@vspacedefault}}
\resetacvspace
%    \end{macrocode}
% \cs{eqe@insertContAnnot} promotes a new page, and if the current
% page is different than the starting page, \cs{eq@currProbStartPage}, we
% insert \cs{annotContStr}.
%    \begin{macrocode}
\newcommand{\eqe@insertContAnnot}{\promoteNewPage[\ic@vspace]%
    \ifnum\arabic{page}>\eq@currProbStartPage
        \xdef\eq@currProbStartPage{\arabic{page}}%
        {\settowidth{\eq@tmplength}{\parts@indent\eqe@prtsepPrb}%
        \xdef\eqe@partsIndent{\the\eq@tmplength}}%
        \noindent\hspace*{-\eqemargin}\hspace{-\eqe@partsIndent}%
        \annotContStr\medskip
    \fi
}
\let\insertContAnnot\eqe@insertContAnnot
%    \end{macrocode}
%    \end{macro}
%    \end{macro}
%    \begin{macro}{\turnContAnnotOff}
%    \begin{macro}{\turnContAnnotOn}
% Turn off and on this feature. The default is on.
%    \begin{macrocode}
%\newcommand{\@gobbloptone}[1][]{}
\newcommand{\turnContAnnotOff}{\global\let\eq@insertContAnnot\relax}
\newcommand{\turnContAnnotOn}{%
    \global\let\eq@insertContAnnot\eqe@insertContAnnot}
\turnContAnnotOff
%    \end{macrocode}
%    \end{macro}
%    \end{macro}
%    \end{macro}
%    \begin{macro}{\aNewPage}
%    \begin{macro}{\qNewPage}
% A simple command for inserting \cs{newpage}, only if the \cs{answerkey}
% option has been taken.
% \changes{v1.6h}{2007/01/24}
% {
%   Added \cs{aNewPage} and \cs{qNewPage}
% }
%    \begin{macrocode}
\newcommand\aNewPage{\ifanswerkey\newpage\fi}
\newcommand\qNewPage{\ifanswerkey\else\newpage\fi}
%    \end{macrocode}
%    \end{macro}
%    \end{macro}
%    \begin{macro}{\OnBackOfPage}
%
% In an effort to make maximum use of the paper, I sometimes ask the
% students to solve the problem on the back of a page. The following
% command is an automated instruction. Generally, we work on the back
% of the previous page, unless we are on page 1, in this case we work
% on the back of page 1.
%
%    \begin{macrocode}
\newcounter{backofpage}
\newcommand\bopText{on the back of page~\boPage}
\newcommand\bopCoverPageText{\space(the cover page)}
\newcommand\OnBackOfPage[1][\bopText]{%
    \refstepcounter{backofpage}\label{bop\thebackofpage}
    \begingroup
    \expandafter\ifx\csname r@bop\thebackofpage\endcsname\relax
        \def\boPage{??}%
    \else
        \edef\temp{\csname r@bop\thebackofpage\endcsname}%
        \ifx\hyper@anchor\@undefined
            \edef\boPage{\expandafter\@secondoftwo\temp}%
        \else
            \edef\boPage{\expandafter\@secondoffive\temp}%
        \fi
        \c@eq@count\boPage
        \advance\c@eq@count-1\relax
%    \end{macrocode}
% If on page 1, we work on the back of page 1, otherwise, we work on the
% back of the previous page.
%    \begin{macrocode}
        \edef\boPage
        {%
            \ifx\eqex@coverpage\relax
                \expandafter\ifnum\value{eq@count}=0\relax
                    1%
                \else
                    {\theeq@count}%
                \fi
            \else
                \ifnum\value{eq@count}=1\relax
                    {\theeq@count}\bopCoverPageText
                \else
                    {\theeq@count}%
                \fi
            \fi
        }%
    \fi
    #1%
    \endgroup
}
%    \end{macrocode}
%    \end{macro}
%
%\subsection{Vertical Space Filling Options}
%
% When the \texttt{nosolutions} or the \texttt{vspacewithsolns} is used, a
% vertical space is generated by the \texttt{solution} environment.
% Previously, this has just been a vertical white space, now, we provide
% the ability to fill the space with horizontal rules of different types.
% Below is the implementation of this.
%    \begin{macro}{\eqWriteLineColor}
% The color of the rule to use.
%    \begin{macrocode}
\newcommand{\eqWriteLineColor}[1]{\def\eq@WriteLineColor{#1}}
\eqWriteLineColor{gray}
%    \end{macrocode}
%    \end{macro}
%    \begin{macro}{\eqWLSpacing}
% The line spacing between the rules.
%    \begin{macrocode}
\newcommand{\eqWLSpacing}[1]{\def\eq@WLSpacing{#1}}
\eqWLSpacing{14pt}
%    \end{macrocode}
%    \end{macro}
% \paragraph*{Fill Types.} We have three types of line fill:
% \cs{hrulefill}, \cs{dotfill}, and a custom rule \cs{eqdashrulefill}.
% The commands three \cs{eqWriteLineFill}, \cs{eqWriteLineDots}, and
% \cs{eqWriteLineDashFill} implements these three types. They are
% \cs{let} to \cs{eqWriteLine}, which is used in \cs{vspaceFillerLines}.
%    \begin{macrocode}
\newcommand{\eqWriteLineFill}{%
    \textcolor{\eq@WriteLineColor}{\hrulefill}}
\newcommand{\eqWriteLineDots}{%
    \textcolor{\eq@WriteLineColor}{\dotfill}}
\def\eqdashrulefill{\leavevmode
    \cleaders\hb@xt@ .44em{\hss\rule{.22em}{.4pt}\hss}\hfill\kern\z@}
\newcommand{\eqWriteLineDashFill}{%
    \textcolor{\eq@WriteLineColor}{\eqdashrulefill}}
\newcommand{\vspaceFillerLines}[1]{\offinterlineskip
    \parindent0pt\relax\parskip0pt\relax
    \@tempdima\eq@WLSpacing\relax
    \@whiledim\@tempdima<#1\relax\do{\vspace{\eq@WLSpacing}\eqWriteLine
    \addtolength{\@tempdima}{.4pt+\eq@WLSpacing}\par}%
    \vfill
}
%    \end{macrocode}
%    \begin{macro}{\useFillerLines}
% When used, the vertical space is written with lines (rules, dashes, dots).
%\changes{v2.0e}{2011/03/07}
%{%
%  Added the feature of filling the vertical space with ruled lines of
%  different types. This feature is available for paper options and for
%  \texttt{nosolutions} and \texttt{vspacewithsolution} options.
%}
%    \begin{macrocode}
\newcommand{\useFillerLines}{\let\vspaceFiller\vspaceFillerLines}
%    \end{macrocode}
%    \end{macro}
%    \begin{macro}{\useFillerDefault}
% Resets the vertical space to the original white space.
%    \begin{macrocode}
\newcommand{\useFillerDefault}{\let\vspaceFiller\vspaceFillerDefault}
%    \end{macrocode}
%    \end{macro}
%    \begin{macro}{\fillTypeHRule}
% Writes the line as a solid line (\cs{hrulefill}).
%    \begin{macrocode}
\newcommand{\fillTypeHRule}{\let\eqWriteLine\eqWriteLineFill}
%    \end{macrocode}
%    \end{macro}
%    \begin{macro}{\fillTypeDots}
% Writes the line as a dotted line (\cs{dotfill}).
%    \begin{macrocode}
\newcommand{\fillTypeDots}{\let\eqWriteLine\eqWriteLineDots}
%    \end{macrocode}
%    \end{macro}
%    \begin{macro}{\fillTypeDashLine}
% Writes the line as a dotted line (\cs{eqdashrulefill}).
%    \begin{macrocode}
\newcommand{\fillTypeDashLine}{\let\eqWriteLine\eqWriteLineDashFill}
%    \end{macrocode}
%    \end{macro}
%    \begin{macro}{\fillTypeDefault}
% Resets fill type back to the default, \cs{hrulefill}.
%    \begin{macrocode}
\newcommand{\fillTypeDefault}{\let\eqWriteLine\eqWriteLineFill}
%    \end{macrocode}
%    \end{macro}
% Set the \textsf{eqexam} page style.
%    \begin{macrocode}
\ifeqfortextbook\else\pagestyle{eqExamheadings}\fi
%    \end{macrocode}
%    \begin{macrocode}
%</package>
%<*textbook>
\ProvidesFile{eqtextb.def}
 [2012/25/01 v3.0t Cmds used by the fortextbook option (dps)]
%    \end{macrocode}
%
% \section{Concerning the \protect\texttt{fortextbook} option}\label{fortextbook}
%
% What are my goals/desired features? Modern (U.S.) textbooks---at least
% the ones I'm familiar with---consist of some or all of the following
% resources:
% \begin{itemize}
%    \item \textbf{Student Edition}: Answers to odd-numbered problems
%    appear in the back of the text.
%    \item[] We need to have a scheme where odd-numbered problems, under suitable options,
%    are compiled. \textbf{Goal:} It does not need to be restricted to odd-numbered, however, need to latex only
%    those problems that meet the ``include'' criteria.
%    \item[] For chapter review problem sets, odd-number problems are have solutions in the
%    back of the book.
%    \item[] For chapter quizzes, odd-numbered (optionally all) solutions
%       are in the back of the book.
%
%    \item \textbf{Instructor Edition}: Answers to all problems appear in
%    the back of the book. Answers may also appear in the body of the text,
%    in the margins of the text, or immediately after the statement of the
%    problem. If the answer is too long, there is a cross-reference to the
%    solution in the appendix.
%    \item[] Some publishers I've seen have wide margins where additional material can be inserted
%    (historical sketches, instructor notes, pictures, etc.). In these margins, the answers to the problems
%    can appear. Other publishers put answer immediately
%    following the questions. The latter is easy to do; just have a macro, say \verb!\ANS{$12.5$}!, which only
%    expands when the ``instructor'' option is used.
%    \item \textbf{Student Solution Manual}: Contains solutions to all
%    odd-problems, as well as any review problems and chapter quizzes.
%    Some publishers include all solutions to chapter quizzes.
%   \item[] A solution manual is a separate publication. This document would be
%       created by latexing one or more of the auxiliary file (\texttt{.sol})
%       These files might have to be edited before the final compile. We include only the solutions
%       that meet the include criteria (i.e., odd-numbered ones).
%   \item[] The current features of \textsf{eqexam} is what is needed here. The authors need only
%       include solutions to each problem in a \texttt{solution} environment. Now, I realize that often times
%       the authors create the solutions, but someone else, possibly a grade student or contractor, solves the problems.
%       In the latter case, the authors would probably not like to turn over the source files to the one solving and typesetting
%       the problems.
%    \item \textbf{Instructor Solution Manual}: Contains solutions to all
%    problems, review problems, chapter quizzes.
%   \item[] Similar comments for the instructor solution manual.
% \end{itemize}
% Some other thoughts by a contributor:
% \begin{itemize}
%\item Often there is a diagram or graphic within the problem -- this has
%    some figure caption and after the running counter of the figure it is
%    named the NUMBER of the problem (cross-reference to the problem number)
%
%\item Often the probs with soln are setup in two-column style.
%
%\item Often the PROBLEM NUMBER has a special formatting (not only bold and
%    black), maybe with a colorframebox around or some special formatting
%    from the author
%
%\item Of course I have seen in some EXAMPLES that there is a wide margin
%    to put in additional graphics etc. setup in two-side style -- wide left
%    margin on even pages, wide right margins on odd pages. Here as well
%    are captions setup and cross-references.
%\end{itemize}
%
%\subsection{Setting options with \texorpdfstring{\cs{textbookOpts}}{\textbackslash{textbookOpts}}}
%    \begin{macro}{\textbookOpts}
%    \begin{macro}{marginans}
%    \begin{macro}{inlineans}
%    \begin{macro}{marginsonleft}
%    \begin{macro}{ssols}
%    \begin{macro}{lsols}
% We set up a command for setting the options for the \texttt{fortextbook} option.
%    \begin{macrocode}
\define@boolkey{eqe@tbopts}[is]{instred}[true]{}
\define@boolkey{eqe@tbopts}[is]{studented}[true]{%
    \ifisstudented
%    \end{macrocode}
% Now let's try to filter out the even-numbered problems for the student edition.
%    \begin{macrocode}
    \tbfilterOutEvenNums
%    \end{macrocode}
% The above command is normally \cs{let} to \cs{@gobble}.
%    \begin{macrocode}
    \fi
}
%    \end{macrocode}
% This code is executed in \cs{exambegdef}, the start up code of the \texttt{exam} environment.
% This enables problems with fill-ins, true/false, or multiple choice, to have the answer appear
% in the space provided.
%    \begin{macrocode}
\def\tb@beginexam@code{%
    \ifisinstred\answerkeytrue\eq@proofingtrue\fi}
%    \end{macrocode}
% \verb!\eqEXt{\theeqquestionnoi}! and \verb!\endeqEXt\tok1\tok2! enclose each
% solution, \cs{tbfilterOutEvenNums} redefines \cs{eqEXt} to gobble everything,
% when the page number is even, through \cs{endeqEXt} and the two tokens it follows.
% This leaves only the odd-numbered problems.
%    \begin{macrocode}
\newcommand{\tbfilterOutEvenNums}{%
    \def\eqEXt##1{\ifodd##1\let\eqe@next\relax\else
    \def\eqe@next{\gobbletoEndeqExt}\fi\eqe@next}%
}
\newcommand{\tballowAllNums}{%
    \let\eqEXt\@gobble
    \let\endeqEXt\relax
}
\define@boolkey{eqe@tbopts}[is]{marginans}[true]{}
\define@boolkey{eqe@tbopts}[is]{inlineans}[true]{}
\define@boolkey{eqe@tbopts}[]{marginsonleft}[true]
%    \end{macrocode}
% If margins are always on left, we turn off switching of margin notes
% as placed by \cs{marginpar}, and use \cs{reversemarginpar} to get them
% on the left.
%    \begin{macrocode}
    {\@mparswitchfalse\reversemarginpar}
\define@boolkey{eqe@tbopts}[show]{ssols}[true]{}
\define@boolkey{eqe@tbopts}[show]{lsols}[true]{%
    \ifshowlsols\let\tb@soln@choice\tb@showlsols\fi}
%    \end{macrocode}
%    \end{macro}
%    \end{macro}
%    \end{macro}
%    \end{macro}
%    \end{macro}
% The default settings are \texttt{true} for \texttt{studented} and
% \texttt{false} for \texttt{instred}.
%
%    \begin{macrocode}
\newcommand{\textbookOpts}[1]{\setkeys{eqe@tbopts}{#1}%
%    \end{macrocode}
% We do not allow both \texttt{instred} and \texttt{studented} to be true.
%    \begin{macrocode}
    \ifisinstred\global\isstudentedfalse\else
        \ifisstudented\global\isinstredfalse
    \fi\fi
%    \end{macrocode}
% \changes{v3.0v}{2012/03/14}{Include \cs{tbMakeFinalCalcs} at end of
% \cs{textbookOpts}}
% Added this part in in case \cs{textbookOpts} comes after \cs{marparboxwidth}.
%    \begin{macrocode}
    \ifdim\tbmarparboxwidth=1sp\else
    \expandafter\tbMakeFinalCalcs\fi
}
%    \end{macrocode}
% As mentioned above, the default settings are \texttt{true} for \texttt{studented} and
% \texttt{false} for \texttt{instred}.
%    \begin{macrocode}
\isstudentedtrue
\isinstredfalse
\ismarginansfalse
\isinlineansfalse
%    \end{macrocode}
% The command is available only in the preamble.
%    \begin{macrocode}
\@onlypreamble{\textbookOpts}
%    \end{macrocode}
%    \end{macro}
%    \begin{macro}{\turnOffMarAnsOnAnsInline}
%    \begin{macro}{\turnOnMarAnsOffAnsInline}
%    \begin{macro}{\toggleInstrAns}
% These three command may not be useful in the creation of a textbook, but
% you never know, I used them in my demo doc \texttt{fortextbook.tex} to turn
% off and on the display of the answers (change margin to inline, change inline to margin,
% and toggle margin and inline).
%    \begin{macrocode}
\newcommand{\turnOffMarAnsOnAnsInline}{%
    \global\ismarginansfalse\global\isinlineanstrue
    \insMidMarg{\global\ismarginansfalse
        \global\isinlineanstrue}%
}
\newcommand{\turnOnMarAnsOffAnsInline}{%
    \global\ismarginanstrue\global\isinlineansfalse
    \insMidMarg{\global\ismarginanstrue
        \global\isinlineansfalse}%
}
\newcommand{\toggleInstrAns}{%
    \ifisinstred\ifismarginans
        \global\ismarginansfalse\global\isinlineanstrue
        \insMidMarg{\global\ismarginansfalse
            \global\isinlineanstrue}%
    \else
        \global\ismarginanstrue\global\isinlineansfalse
        \insMidMarg{\global\ismarginanstrue
            \global\isinlineansfalse}%
    \fi\fi
}
%    \end{macrocode}
%    \end{macro}
%    \end{macro}
%    \end{macro}
%
% \subsection{Macros to display answers/shortsolns}
% In this section, we develop some commands to display answers or short solutions. These
% would appear if \texttt{instred=true}, in-line, or in the margins.
%    \begin{macro}{\ANS}
% Let us begin by creating a simple macro for saving an answer. The answer is displayed ``in-line.''
% No verbatim-type text allowed, no unbalanced braces unless escaped. \cs{ANS} displays the answer
% if the \texttt{instred} option of the \texttt{eqe@tbopts} family, i.e., by executing
%
%    \begin{macro}{\bGrpANS}
%    \begin{macro}{\eGrpANS}
% Two macros used to group answers in the margins.
%\changes{v3.0q}{2011/14/22}{Added \cs{bGrpANS} and \cs{eGrpANS}}
%    \begin{macrocode}
\newif\ifWithinANSGrp\WithinANSGrpfalse
\newif\ifftb@isANSListOpen\ftb@isANSListOpenfalse
\newcommand{\bGrpANS}{%
    \if\probstar*\else
        \PackageError{eqexam}{Use of \string\bGrpANS\space
        only applies\MessageBreak to the problem* environment}{Please
        remove this \string\bGrpANS.}%
    \fi
    \ifWithinANSGrp
        \global\WithinANSGrpfalse
        \let\tb@next\relax
        \PackageError{eqexam}{\string\bGrpANS\space already open}
        {You issued an earlier \string\bGrpANS,
         but did not close it.}%
    \else
        \global\WithinANSGrptrue
        \global\ftb@isANSListOpenfalse
        \def\tb@next{\ANS}%
    \fi
    \tb@next
}
\newcommand{\eGrpANS}{%
    \if\probstar*\else
        \PackageError{eqexam}{Use of \string\eGrpANS\space
        only applies\MessageBreak to the problem* environment}{Please
        remove this \string\eGrpANS.}%
    \fi
    \ifWithinANSGrp
        \global\WithinANSGrpfalse
        \def\tb@next{\ANS}%
     \else
        \let\tb@next\relax
        \PackageError{eqexam}{\string\eGrpANS\space already closed}
        {You've issued two consecutive \string\eGrpANS\space
         commands,\MessageBreak either remove this one
        or the previous one.}%
     \fi
    \tb@next
}
%    \end{macrocode}
%    \end{macro}
%    \end{macro}
% \cs{ANS} begin by checking to see if there is a star that follows the command,
% this is used for inline answers. If \texttt{*} is present, we do not put the
% answer inline, but will put it in the margins if the option call for it.
%    \begin{macrocode}
\newcommand{\ANS}{\@ifstar{\let\tb@istart=1\tb@ANS}
    {\let\tb@istart=0\tb@ANS}}
%    \end{macrocode}
% (10/13/2011) The following is the original definition of \cs{tb@ANS} before the creation of
% the commands \cs{bGrpANS} and \cs{bGrpANS}. We keep this to revert to this definition
% if this new feature causes problems.
%\begin{verbatim}
%\newcommand{\tb@ANS}[1]{%
%    \ifisinstred
%        \ifisinlineans\if\tb@istart0\ANSFmt{\theeqquestionnoi}{#1}\fi\fi
%        \ifismarginans
%            \edef\eqe@prehold{\noexpand\par\kern0pt\noindent
%                \if\probstar*%
%                   \noexpand\begin{eqeList}[\tb@wparts@len]{%
%                   \noexpand\eqedsplyOnlyFrst{\theeqquestionnoi}%
%                   {\thepartno}\noexpand\eqe@hspannerMrg
%                   \noexpand\makebox[\noexpand\tbmrgpartwdth]%
%                        {\noexpand\tb@mrgPartFmt{\thepartno}}}%
%                \else
%                   \noexpand\begin{eqeList}%
%                    {\noexpand\tb@mrgDigitFmt{%
%                        \theeqquestionnoi\eqe@decPointMrg}}%
%                \fi
%            }\expandafter\insMidMarg%
%                \expandafter{\eqe@prehold#1\end{eqeList}}%
%        \fi
%    \fi
%}
%\end{verbatim}
% \DescribeMacro{\ftb@defineInsSpan} is used when there is an optional
% argument for \cs{ANS}. It formats the range of parts,
% for example, (a)--(c). This macro can be redefined, I suppose, to meet
% the needs of the author.
%    \begin{macrocode}
\def\ftb@defineInsSpan#1{\def\ftb@argi{#1}\ifx\ftb@argi\@empty
    \def\ftb@InsSpan{}\else\ftb@spanPrts{#1}%
    \def\ftb@InsSpan{\noexpand\hspace{-\labelsep}%
    \noexpand\textcolor{MRGPARTcolor}{--}\noexpand
    \makebox[\noexpand\tbmrgpartwdth]{\noexpand
    \tb@mrgPartFmt{\ftb@EndSpanPrts}}\eqe@hspannerMrg}\fi
}
%    \end{macrocode}
% \DescribeMacro{\ftb@spanPrts} calculates the letter of the end of the
% range. \texttt{\#1} is passed by \cs{ANS} (\cs{tb@ANS}, actually). For
% example if we have \verb!\ANS[2]{...}!, \texttt{\#1=2}.
%    \begin{macrocode}
\def\ftb@spanPrts#1{{%
    \advance\value{partno}by#1\relax
    \xdef\ftb@EndSpanPrts{\thepartno}}%
}
%    \end{macrocode}
% \DescribeMacro{\ftb@EqeListPrtsFmt} is the internal formatting used within the \texttt{eqeList} for the part letter.
%    \begin{macrocode}
\def\ftb@EqeListPrtsFmt{\noexpand
    \makebox[\noexpand\tbmrgpartwdth]{\noexpand
    \tb@mrgPartFmt{\thepartno}}\nobreak
}
%    \end{macrocode}
% \DescribeMacro{\ftb@OpenEqeListPrts} opens an \texttt{eqeList} environment,
% and displays the question number (optionally) and the part number.
%    \begin{macrocode}
\def\ftb@OpenEqeListPrts{\noexpand
   \begin{eqeList}[\tb@wparts@len]{\noexpand
   \eqedsplyOnlyFrst{\theeqquestionnoi}%
   {\thepartno}\noexpand\eqe@hspannerMrg\ftb@EqeListPrtsFmt}%
}
%    \end{macrocode}
% \DescribeMacro{\ftb@CloseEqeList} closes the \texttt{eqeList} after
% inserting \cs{qe@prehold} and the content, \texttt{\#1}.
%    \begin{macrocode}
\def\ftb@CloseEqeList#1{\expandafter\insMidMarg%
    \expandafter{\eqe@prehold#1\end{eqeList}}%
}
%    \end{macrocode}
% \DescribeMacro{\grpANSDelimiter} delimits the parts
% when \cs{bGrpANS}/\cs{eGrpANS} is used. May be redefined.
%    \begin{macrocode}
\newcommand{\grpANSDelimiter}{\textcolor{MRGPARTcolor}{,}\space}
%    \end{macrocode}
% \DescribeMacro{\tb@ANS} does the main work of \cs{ANS}.
%    \begin{macrocode}
\newcommand{\tb@ANS}[2][]{%
    \ifisinstred
        \ifisinlineans
            \if\tb@istart0\ANSFmt{\theeqquestionnoi}{#2}\fi
        \fi
        \ifismarginans
            \ftb@defineInsSpan{#1}%
%    \end{macrocode}
% We create the code that we will introduce into \cs{insMidMarg}, this will
% be introduced prior to \texttt{\#2}.
%    \begin{macrocode}
            \edef\eqe@prehold{%
                \if\probstar*%
%    \end{macrocode}
% If this question is one with parts...
%    \begin{macrocode}
                    \ifftb@isANSListOpen
%    \end{macrocode}
% If the list is already open (\cs{ifftb@isANSListOpen}),
% we just add content to the \texttt{eqeList} environment.
%    \begin{macrocode}
                       \ftb@EqeListPrtsFmt\noexpand\eqe@hspannerMrg
                    \else
%    \end{macrocode}
% If the list is not open, we start the \texttt{eqeList} environment
% in the usual way, this also includes the case where \cs{bGrpAns} is
% not uses, which is normally the case.
%
%    \begin{macrocode}
                        \noexpand\par\kern0pt\noindent
                        \ftb@OpenEqeListPrts\ftb@InsSpan
                    \fi
                \else
%    \end{macrocode}
% This is a question without parts.
%    \begin{macrocode}
                   \noexpand\begin{eqeList}%
                    {\noexpand\tb@mrgDigitFmt{%
                    \theeqquestionnoi\eqe@decPointMrg}}%
                \fi
            }%
%    \end{macrocode}
% We have finished constructing \cs{eqe@prehold}. We next set
% \cs{ftb@isANSListOpen} to \texttt{true}, if \cs{WithinANSGrp} is \texttt{true}.
%    \begin{macrocode}
            \ifWithinANSGrp\global\ftb@isANSListOpentrue\fi
%    \end{macrocode}
% If we are within an open group, we emit \cs{insMidMarg} with the
% \cs{eqe@prehold}, followed by \texttt{\#2}, and a comma-space combo,
% but \emph{we do not close} the \texttt{eqeList} environment.
%    \begin{macrocode}
            \ifWithinANSGrp
                \expandafter\insMidMarg\expandafter
                    {\eqe@prehold#2\grpANSDelimiter}%
            \else
%    \end{macrocode}
% This is the normal case, we insert \cs{eqe@prehold}, \texttt{\#2}, and
% close the \texttt{eqeList} environment.
%
%    \begin{macrocode}
                \expandafter\insMidMarg%
                \expandafter{\eqe@prehold#2\end{eqeList}}%
                \global\ftb@isANSListOpenfalse
            \fi
        \fi % \ifismarginans
    \fi %\ifisinstred
}
%    \end{macrocode}
% End (10/13)
%    \begin{environment}{eqeList}
% An environment used to format the answers in the margins, when
% \texttt{marginans} is in effect.
%    \begin{macrocode}
\newenvironment{eqeList}[2][\tb@woparts@len]{\begin{list}{#2}{%
    \def\argi{#1}\setlength{\labelwidth}{#1}%
    \ifx\argi\tb@wparts@len
    \settowidth{\labelsep}{\eqe@prtsepMrg}\else
    \settowidth{\labelsep}{\eqe@hspannerMrg}\fi
    \setlength{\leftmargin}{\labelwidth+\labelsep}%
    \setlength{\parskip}{0pt}\setlength{\partopsep}{0pt}%
    \setlength{\topsep}{1pt}\setlength{\parsep}{0pt}%
    \setlength{\itemindent}{0pt}\setlength{\itemsep}{3pt}%
}\item\relax}{\end{list}}
%    \end{macrocode}
%    \end{environment}
% \paragraph*{Formatting Answers and Solutions}
%    \begin{macro}{\mrgDigitFmt}
% Format of the digit (and the decimal point) for the answers in the margins.
%\begin{verbatim}
%\mrgDigitFmt{\textbf{#1}}
%\mrgPartFmt{\textbf{(\hfil#1\hfil)}}
%\setMarIndents[\bfseries\normalsize\normalfont]{00}{(d)}
%\end{verbatim}
%    \begin{macrocode}
\newcommand{\mrgDigitFmt}[1]{\def\tb@mrgDigitFmt##1{#1}}
\mrgDigitFmt{#1}
%    \end{macrocode}
%    \begin{macro}{\mrgPartFmt}
% Format of the part (including possibly the parentheses), example give above.
%    \begin{macrocode}
\definecolor{MRGPARTcolor}{named}{black}
\newcommand{\mrgPartFmt}[1]{\def\tb@mrgPartFmt##1{#1}}
\mrgPartFmt{\textcolor{MRGPARTcolor}{(\hfil#1\hfil)}}
%    \end{macrocode}
%    \end{macro}
%    \end{macro}
%    \end{macro}
%    \begin{macro}{ANScolor}
% The default color of the answers that appear in the margins or inline.
%    \begin{macrocode}
\definecolor{ANScolor}{rgb}{0,0,.8}
%    \end{macrocode}
%    \end{macro}
%    \begin{macro}{\ANSFmt}
% The command that sets the format, may be redefined as needed. Used in the
% \cs{ANS} command above.
%    \begin{macrocode}
\newcommand{\ANSFmt}[2]{\textcolor{ANScolor}{#2}}
%    \end{macrocode}
%    \end{macro}
% We have two environments that we use in three different situations:
%\begin{itemize}
%    \item \texttt{eqequestions} environment: (1) Used to control the display of the
%           \texttt{probset} environment within the body of the textbook; (2) used
%           to control the display of the solutions ``in the back of the book.''
%    \item \texttt{eqeList} environment: Used for displaying answers in the margin of
%           the book, when the appropriate options allow it.
%\end{itemize}
% We want to be able to manipulate some of the parameters of these three
% situation, independently of each other. There are several issues, setting
% what I have been calling the gutter width, and the display of the problem
% numbers.
%\par\medskip\noindent
% We define four commands for each of the three situations described above. The names
% have a pattern to them, and similarly named commands have the same use.
%
% The numbering of the problems has the pattern: \verb*!dd. (a) !
% We provide convenience commands to give these internal macros values
%    \begin{macro}{\prbDecPt}
%    \begin{macro}{\prbPrtsep}
%    \begin{macro}{\prbNumPrtsep}
% Basic parameters for the problems in the body of the text.
%    \begin{macrocode}
\def\eqe@decPointPrb{.}     % decimal point of prob number
\def\eqe@prtsepPrb{\ }      % prob with parts, space after part
\def\eqe@hspannerPrb{\ }    % space after prob number
\newcommand{\prbDecPt}[1]{\def\eqe@decPointPrb{#1}}
\newcommand{\prbPrtsep}[1]{\def\eqe@prtsepPrb{#1}}
\newcommand{\prbNumPrtsep}[1]{\def\eqe@hspannerPrb{#1}}
%    \end{macrocode}
%    \end{macro}
%    \end{macro}
%    \end{macro}
%    \begin{macro}{\solDecPt}
%    \begin{macro}{\solPrtsep}
%    \begin{macro}{\solNumPrtsep}
% Basic parameters for the problems in the solution sets.
%    \begin{macrocode}
\def\eqedecPointSoln{.}    % decimal point of prob number
\def\eqe@prtsepSoln{\ }     % prob with parts, space after part
\def\eqe@hspannerSoln{\ }   % space after prob number
\newcommand{\solDecPt}[1]{\def\eqedecPointSoln{#1}}
\newcommand{\solPrtsep}[1]{\def\eqe@prtsepSoln{#1}}
\newcommand{\solNumPrtsep}[1]{\def\eqe@hspannerSoln{#1}}
%    \end{macrocode}
%    \end{macro}
%    \end{macro}
%    \end{macro}
%    \begin{macro}{\mrgDecPt}
%    \begin{macro}{\mrgPrtsep}
%    \begin{macro}{\mrgNumPrtsep}
% Basic parameters for the problems in the margins.
%    \begin{macrocode}
\def\eqe@decPointMrg{.}     % decimal point of prob number
\def\eqe@prtsepMrg{\ }      % prob with parts, space after part
\def\eqe@hspannerMrg{\ }    % space after prob number
\newcommand{\mrgDecPt}[1]{\def\eqe@decPointMrg{#1}}
\newcommand{\mrgPrtsep}[1]{\def\eqe@prtsepMrg{#1}}
\newcommand{\mrgNumPrtsep}[1]{\def\eqe@hspannerMrg{#1}}
%    \end{macrocode}
%    \end{macro}
%    \end{macro}
%    \end{macro}
%    \begin{macro}{\setMarIndents}
% Sets some dimensions used by the \texttt{eqeList} environment.
% \cs{tb@woparts@len} is calculated and
% is used as the default gutter width in \texttt{eqeList}. \cs{tb@wparts@len}
% is used for the gutter width for the gutter width when there is a problem
% with parts. Finally, \cs{tbmrgpartwdth} in \cs{ANS} and is used for the width
% of a \cs{makebox} that enclosed the part letter.
%    \begin{macrocode}
\newcommand{\setMarIndents}[3][\normalsize\normalfont]{{%
    \settowidth{\@tempdima}{#1#2\eqe@decPointMrg}%
    \xdef\tb@woparts@len{\the\@tempdima}%
    \settowidth{\@tempdima}%
        {#1#2\eqe@decPointMrg\eqe@hspannerMrg#3}%
    \xdef\tb@wparts@len{\the\@tempdima}%
    \settowidth{\@tempdima}{#1#3}%
    \xdef\tbmrgpartwdth{\the\@tempdima}%
}}
\setMarIndents{00}{(d)}
%    \end{macrocode}
%    \end{macro}
%    \begin{macro}{\setSolnIndent}
% Used to set the some parameters used by \texttt{eqequestions}, in the solutions file.
%    \begin{macrocode}
\newdimen\solnGutter
\newcommand{\setSolnIndent}[3][\normalsize\normalfont\bfseries]{%
    {\settowidth{\@tempdima}{#1#2\eqedecPointSoln\eqe@hspannerSoln}%
    \global\solnGutter=\@tempdima
%    \xdef\solnGutter{\the\@tempdima}%
    \settowidth{\@tempdima}{#1#3}%
    \xdef\tbsolnpartwdth{\the\@tempdima}%
}}
\setSolnIndent{00}{(d)}
%    \end{macrocode}
%    \end{macro}
%    \begin{macro}{\setSolnMargins}
% This command is written to the solution file, and expanded when that file
% is input back in. If \cs{solnGutter} is not \texttt{0pt}, we set the length
% of \cs{eqemargin} using the current value; otherwise, we use the value
% determined by \cs{setSolnIndent}, above.
%    \begin{macrocode}
\renewcommand{\setSolnMargins}[1]{%
    \ifdim\solnGutter=0pt \setlength\eqemargin{#1}\else
    \setlength\eqemargin{\solnGutter}\fi}
%\newcommand{\defaultSolnIndent}{\gdef\solnGutter{0pt}}
\newcommand{\defaultSolnIndent}{\global\solnGutter=0pt}
\defaultSolnIndent
%    \end{macrocode}
%    \end{macro}
%    \begin{macro}{\prbNumFmt}
%    \begin{macro}{\solWoPrtsFmt}
%    \begin{macro}{\solWPrtsFmt}
% We redefine \cs{exlabelformat}, \cs{exsllabelformat}, and
% \cs{exsllabelformatwp}. They are defined in such a way as
% to simply their modification through a series of simple
% formatting commands. The defaults are
%\begin{verbatim}
%\prbNumFmt{\textbf{#1}}
%\solWoPrtsFmt{\textbf{#1}}
%\solWPrtsFmt{\textbf{#1}}{(\hfil#2\hfil)}
%\end{verbatim}
%    \begin{macrocode}
\renewcommand{\exlabelformat}{%
    \tbprbNumFmt{\theeqquestionnoi\eqe@decPointPrb}}
%    \end{macrocode}
% \cs{prbNumFmt} is the format for the number of the problems
% in the body of the text. The argument \texttt{\#1} is a symbolic
% argument for the question number.
%    \begin{macrocode}
\newcommand{\prbNumFmt}[1]{\def\tbprbNumFmt##1{#1\eqe@hspannerPrb}}
\prbNumFmt{\textbf{#1}}
%    \end{macrocode}
% Redefine \cs{exsllabelformat}, and \cs{exsllabelformatwp}
%    \begin{macrocode}
\renewcommand{\exsllabelformat}{\string\tbsolWoPrtsFmt{%
    \theeqquestionnoi\string\eqedecPointSoln}}
\renewcommand{\exsllabelformatwp}{\string\tbsolWPrtsFmt%
    {\string\eqedsplyOnlyFrst{\theeqquestionnoi}{\thepartno}}%
    {\thepartno}%
}
%    \end{macrocode}
% \cs{solWoPrtsFmt} is the format for the number of the problems
% in the solution set. The argument \texttt{\#1} is a symbolic
% argument for the question number. \cs{solWPrtsFmt} is the format
% for a problem with parts in the solution file. \texttt{\#1} is
% symbolically the question number, and \texttt{\#2} is a
% symbolic for the part letter.
%    \begin{macrocode}
\newcommand{\solWoPrtsFmt}[1]{\def\tbsolWoPrtsFmt##1{%
    \makebox[0pt][r]{#1\eqe@hspannerSoln}}}
\solWoPrtsFmt{\textbf{#1}}
\newcommand{\solWPrtsFmt}[2]{\def\tbsolWPrtsFmt##1##2{%
    \makebox[0pt][r]{#1\eqe@hspannerSoln}%
    \makebox[\tbsolnpartwdth][l]{#2}\eqe@prtsepSoln%
}}
\solWPrtsFmt{\textbf{#1}}{(\hfil#2\hfil)}
%    \end{macrocode}
% An alternate definition for \cs{solWPrtsFmt}, used by \cs{hangSolWPrtsFmt}.
%    \begin{macrocode}
\newcommand{\solWPrtsFmt@hang}[2]{%
        \def\tbsolWPrtsFmt##1##2{%
        \makebox[0pt][r]{#1\eqe@prtsepSoln%
        \makebox[\tbsolnpartwdth][l]{#2}\eqe@hspannerSoln}%
}}
%    \end{macrocode}
%    \end{macro}
%    \end{macro}
%    \end{macro}
%    \begin{macro}{\hangSolWPrtsFmt}
% The command takes two arguments, the same as \cs{solWPrtsFmt}. When this
% command is executed in the preamble, we get hanging indentation for problems
% with parts.
%\changes{v3.0g}{2011/08/15}{Use this to use ``hanging indentation'' for the
% parts for problems with parts in the solutions file.}
%    \begin{macrocode}
\let\bpartsmrk\relax
\let\epartsmrk\relax
\newcommand{\hangSolWPrtsFmt}[2]{%
%    \end{macrocode}
% At the beginning and ending of a parts environment, we begin and end
% a special \texttt{eqepartsquestions} environment, designed to give
% the desired indentation.
%    \begin{macrocode}
    \def\prior@parts@hook{%
        \writeToSolnFile{^^J\protect\bpartsmrk}}%
%        \writeToSolnFile{\protect\begin{eqepartsquestions}}}%
    \def\post@parts@hook{%
        \writeToSolnFile{\protect\epartsmrk^^J}}%
%        \writeToSolnFile{\protect\end{eqepartsquestions}}}%
    \def\bpartsmrk{\begin{eqepartsquestions}}%
    \def\epartsmrk{\end{eqepartsquestions}}%
%    \end{macrocode}
% We must also redefine \cs{solWPrtsFmt} by letting it to
% \cs{solWPrtsFmt@hang}, then executing it using the parameters passed.
%    \begin{macrocode}
    \let\solWPrtsFmt\solWPrtsFmt@hang
     \solWPrtsFmt{#1}{#2}%
}
\@onlypreamble\hangSolWPrtsFmt
%    \end{macrocode}
%    \end{macro}
% Define some switches, token registers, and boxes for managing
% the answers and marginal notes.
%    \begin{macrocode}
\newif\ifexamenv \examenvfalse
\newif\iffirstemit \firstemittrue
\newtoks\txtbkt@ks \txtbkt@ks={}
\newtoks\txtbkt@ksi \txtbkt@ksi={}
\newbox\txtbkb@xb@t
\newbox\txtbkb@xt@p
\newbox\txtbkb@xh@ld
\let\tbTopMargin\relax
\let\tbBotMargin\relax
\long\def\tb@addtoTopMargin#1{\txtbkt@ksi={#1}%
    \edef\eqe@tmphold{\the\txtbkt@ksi\the\txtbkt@ks}%
    \global\txtbkt@ks=\expandafter{\eqe@tmphold}%
}
\newcommand{\tb@addtoMargin}[1]{%
    \edef\eqe@tmphold{\the\txtbkt@ks}%
    \global\txtbkt@ks=\expandafter{\eqe@tmphold#1}%
}
%    \end{macrocode}
% As my first attempt, let's create two comment environmets to be used within
% the \texttt{solution} environment.
%\begin{verbatim}
%\begin{solution}
%\begin{ssol}
%   <short solution/answer>
%\end{ssol}
%\begin{lsol}
%   <long solution>
%\end{lsol}
%\end{solution}
%\end{verbatim}
% The control of these environments are made through
%    \begin{environment}{lsol}
%    \begin{macro}{\tb@showlsols}
% Place full (or long) solutions in this environment.
%    \begin{macrocode}
\newcommand{\tb@showlsols}{\includecomment{lsol}\excludecomment{ssol}}
%    \end{macrocode}
%    \end{macro}
%    \end{environment}
%    \begin{environment}{ssol}
%    \begin{macro}{\tb@showssols}
% Place short solutions in this environment.
%    \begin{macrocode}
\newcommand{\tb@showssols}{\includecomment{ssol}\excludecomment{lsol}}
%    \end{macrocode}
%    \end{macro}
%    \end{environment}
% The default is to show the short solutions.
%    \begin{macrocode}
\let\tb@soln@choice\tb@showssols
%\let\tb@sols@choice\tb@showssols
%    \end{macrocode}
% \subsection{Marginal Matter.} There are three levels in the margins:
% \begin{enumerate}
%   \item \textbf{Top level}: This is a command \cs{tbTopMargin} with may be
%          redefined between pages. It should have the width of the \cs{parbox}
%          that contains all the content of the margin, this width is \cs{tb@marparboxwidth}
%          (\texttt{\cs{oddsidemargin}-\cs{marginparsep}}). The content \emph{must} be
%          unbreakable across pages. The content of \cs{tbTopMargin} will appear \emph{on
%          every page subsequent to its definition}.
%    \item[] \DescribeMacro{\clearTopMargin} We can clear the top level using the following command
%    \begin{macrocode}
\newcommand{\clearTopMargin}{\global\let\tbTopMargin\relax}
%    \end{macrocode}
%    \item[] Clearing will take effect on the following page.
%    \item[] \DescribeMacro{\setTopMargin} As a convenience macro, we can create top margin content.
%            Redefinitions will appear on the next page from where the definition was made.
%    \begin{macrocode}
\newcommand{\setTopMargin}[1]{\gdef\tbTopMargin{#1}%
    \gdef\tbSaveTopMargin{#1}}
\let\tbTopMargin\@empty
\let\tbSaveTopMargin\@empty
%    \end{macrocode}
%    \item \textbf{Middle level}: The middle level is the most interesting. You write to it
%          using \cs{insMidMarg}. Normally, this is text. If there is too much text,
%          it will be split off and placed in the middle level of the next page. The command
%          \cs{ANS} also writes to the middle level when the \texttt{instred} and \texttt{marginans} options are
%           taken.
%    \item \textbf{Bottom level}: This is similar to the top level, but on the bottom. The
%         command is named \cs{tbBotMargin} and follows the same rules as \cs{tbTopMargin}.
%         Again, the content of \cs{tbBotMargin} will appear \emph{on every
%         page subsequent to its definition}.
%    \item[] \DescribeMacro{\clearBotMargin} We can clear the bottom level using the following command
%    \begin{macrocode}
\newcommand{\clearBotMargin}{\global\let\tbBotMargin\relax}
%    \end{macrocode}
%    \item[] Clearing will take effect on the following page.
%    \item[] \DescribeMacro{\setBotMargin} As a convenience macro, we can create bottom margin content.
%            Redefinitions will appear on the next page from where the definition was made.
%    \begin{macrocode}
\newcommand{\setBotMargin}[1]{\gdef\tbBotMargin{#1}%
    \gdef\tbSaveBotMargin{#1}}
\let\tbBotMargin\@empty
\let\tbSaveBotMargin\@empty
%    \end{macrocode}
%    \begin{macrocode}
\newcommand{\restoreLastTopMargin}{\expandafter\setTopMargin
    \expandafter{\tbSaveTopMargin}}
\newcommand{\restoreLastBotMargin}{\expandafter\setBotMargin
    \expandafter{\tbSaveBotMargin}}
%    \end{macrocode}
% \end{enumerate}
%
%    \begin{macro}{\insMidMarg}
% \cs{insMidMarg} is a \cs{parbox} that will hold the material in the margin.
%    \begin{macro}{\MarParBoxFmt}
% The formatting for the marginal \cs{parbox}
%    \begin{macrocode}
\newcommand{\MarParBoxFmt}{\normalsfcodes
    \normalfont\normalsize\normalbaselines\parindent0pt
    \vbadness\@Mi \hbadness5000 \tolerance9999
    \parskip0pt\raggedright %\spaceskip=0pt\xspaceskip=0pt
    \setlength{\linewidth}{\tbmarparboxwidth}%
}
%    \end{macrocode}
%    \end{macro}
%    \begin{macro}{\tbmarparboxwidth}
% The width of the margin box. Initial value of \texttt{1sp}, if the user does not
% reset the value, it is a package error.
%    \begin{macrocode}
\newlength\tbmarparboxwidth
\setlength\tbmarparboxwidth{1sp}
%    \end{macrocode}
%    \end{macro}
%    \begin{macro}{MidMargcolor}
% The default color of text of the middle level
%    \begin{macrocode}
\definecolor{MidMargcolor}{rgb}{0,0,.8}
\newcommand{\midMargFmt}[1]{%
    \def\tb@midMargFmt{\normalfont\normalsize\normalcolor#1}}
\midMargFmt{\color{MidMargcolor}}
%    \end{macrocode}
%    \end{macro}
%    \begin{macro}{\eqe@MarParBox}
% This is the actual \cs{parbox} that holds the marginal material. I have two versions
% of this box, the first one has height \cs{textheight}, the second one has height
% \texttt{\cs{textheight}+\cs{footskip}}.
%    \begin{macrocode}
%\def\eqe@MarParBox#1{\parbox[b][\textheight][t]%
%    {\tbmarparboxwidth}{\color{MidMargcolor}#1}}
\def\eqe@MarParBox#1{\lower\footskip\hbox{%
    \leavevmode\parbox[b][\textheight+\footskip][t]%
    {\tbmarparboxwidth}{\tb@midMargFmt#1}}} %
%    \end{macrocode}
%    \end{macro}
% Finally, we get to the \cs{insMidMarg}, this is used to write to the middle
% level.
%    \begin{macrocode}
\newcommand{\insMidMarg}[1]{%
    \let\eqe@margininsert\@empty
    \expandafter\tb@addtoMargin\expandafter{\eqe@margininsert#1}%
}
%    \end{macrocode}
%    \end{macro}
%    \begin{macro}{\tbPreMarginHeader}
% Executed prior to the marginal heading
%    \begin{macro}{\tbPostMarginHeader}
% Executed after the marginal heading
%    \begin{macro}{HEADERcolor}
% Default color of a marginal header
%    \begin{macro}{\cngMargHeadColorTo}
% Changes the marginal header to a named color
%\changes{v3.0o}{2011/09/20}{Added \cs{cngMargHeadColorTo} and
% \cs{resetMargHeadColor} to make it easier to change the color
% of the header globally, or just once.}
%    \begin{macro}{\resetMargHeadColor}
% Reset the marginal header color to the default, \texttt{HEADERcolor}
%    \begin{macro}{\tbMarginHeaderFmt}
% Formatting for a marginal header.
% Format the marginal header, the default is \texttt{HEADERcolor} in bold
%    \begin{macrocode}
\newcommand{\tbPreMarginHeader}{\par\penalty-200\vskip0pt plus 24pt
    \kern3pt\noindent\strut}
\newcommand{\tbPostMarginHeader}{\par\nobreak\vskip0pt}
\definecolor{HEADERcolor}{named}{black}
\newcommand{\cngMargHeadColorTo}[1]{\insMidMarg{\gdef\tb@MHC{#1}}}
\newcommand{\resetMargHeadColor}{\insMidMarg{\gdef\tb@MHC{HEADERcolor}}}
\resetMargHeadColor
\newcommand{\tbMarginHeaderFmt}[1]{\textcolor{\tb@MHC}{\textbf{#1}}}
%    \end{macrocode}
%    \end{macro}
%    \end{macro}
%    \end{macro}
%    \end{macro}
%    \end{macro}
%    \end{macro}
%    \begin{macro}{\insMargHead}
% Used to insert a general marginal heading into the middle level. The
% optional parameter allows you to set a mark.
%
%    \begin{macro}{\insProbHead}
% Used to insert a marginal heading for a problem set into the middle level.
% The optional parameter allows you to insert a mark, the default mark is
% \verb!#1 \tbcontinued!.
%    \begin{macrocode}
\newcommand{\insMargHead}[2][]{%
    \protected@edef\temp@exp{\noexpand
    \insMidMarg{\noexpand\tb@marginHeader{#1}{#2}}}\temp@exp
}
\newcommand{\insProbHead}[2][]{%
    \def\tb@argi{#1}\ifx\tb@argi\@empty
        \protected@xdef\currProbHead{#2 \tbcontinued}\else
        \protected@xdef\currProbHead{#1}\fi
    \ifisinstred\ifismarginans
        \insMidMarg{\tb@marginProbHeader{#1}{#2}}\fi\fi
}
\newcommand{\tb@marginProbHeader}[2]{%
    \def\tb@argi{#1}\ifx\tb@argi\@empty
    \tb@marginHeader{#2 \tbcontinued}{#2}\else
    \tb@marginHeader{#1}{#2}\fi
}
\newcommand{\tb@marginHeader}[2]{\tbPreMarginHeader
    \tbMarginHeaderFmt{#2}\def\tb@argi{#1}\ifx\tb@argi\@empty
    \mark{#2}\else\mark{#1}\fi\tbPostMarginHeader
}
%    \end{macrocode}
%    \end{macro}
%    \end{macro}
%    \begin{macro}{\tbcontinued}
% The continue annot that appear when a problem set flows over to the next page.
%    \begin{macrocode}
\newcommand{\tbcontinued}{(cont.)}
%    \end{macrocode}
%    \end{macro}
%    \begin{macro}{\tbplaceMargins}
% Redefine this macro to set the locations of the margins we are writing to.
%    \begin{macrocode}
\newcommand{\marparboxwidth}[1]{%
    \setlength\tbmarparboxwidth{#1}%
    \setlength{\marginparwidth}{\tbmarparboxwidth}%
    \tbMakeFinalCalcs
}
\@onlypreamble\marparboxwidth
\newcommand{\chkmarginboxwidth}{%
    \ifdim\tbmarparboxwidth=1sp \PackageError{eqexam}%
    {You have not set the value of\MessageBreak
        \string\marparboxwidth}%
    {Define the \string\marparboxwidth\space command}\fi
}
%    \end{macrocode}
%    \begin{macro}{\ifmarginsonleft}
% A Boolean switch, if true, all margins are on the left; otherwise, they alternate
%    \begin{macrocode}
\newif\ifmarginsonleft \marginsonleftfalse
%    \end{macrocode}
%    \end{macro}
%    \begin{macro}{\tbSetupForMargins}
% We compute \cs{oddsidemargin}, \cs{evensidemargin}, and \cs{textwidth}
%    \begin{macrocode}
\newcommand{\tbSetupForMargins}{%
    \ifmarginsonleft
        \setlength{\oddsidemargin}{\tbmarparboxwidth+\marginparsep}%
        \setlength{\evensidemargin}{\oddsidemargin}%
        \setlength{\textwidth}{\paperwidth-2in-\oddsidemargin}%
    \else
        \setlength{\oddsidemargin}{0pt}%
        \setlength{\evensidemargin}{\tbmarparboxwidth+\marginparsep}%
        \setlength{\textwidth}{%
            \paperwidth-2in-\oddsidemargin-\evensidemargin}%
    \fi
}
%    \end{macrocode}
%    \end{macro}
%    \begin{macro}{\tbplaceMargins}
% We calculate the coordinates of the lower left hand corner of the margin \cs{parbox}
% depending on the value of \cs{ifmarginsonleft}.
%    \begin{macrocode}
\newcommand{\tbplaceMargins}{{%
    \setlength{\@tempdima}{%
        \paperheight-1in-\topmargin-\headheight-\headsep-\textheight}%
    \xdef\@evenlly{\strip@pt\@tempdima}%
    \xdef\@oddlly{\@evenlly}%
    \setlength{\@tempdima}{1in}%
    \xdef\@evenllx{\strip@pt\@tempdima}%
    \ifmarginsonleft\else
        \setlength{\@tempdima}{1in+\textwidth+\marginparsep}\fi
    \xdef\@oddllx{\strip@pt\@tempdima}%
}}
%    \end{macrocode}
%    \end{macro}
%    \end{macro}
%    \begin{macro}{\tbMakeFinalCalcs}
% Executed by \cs{marparboxwidth}
%    \begin{macrocode}
\newcommand{\tbMakeFinalCalcs}{%
    \tbSetupForMargins
    \tbplaceMargins
}
%    \end{macrocode}
%    \end{macro}
%    \begin{macro}{\tbminskipbtnlayers}
% \cs{tbminskipbtnlayers} is the minimum skip between layers (top, middle, bottom)
% Executed by \cs{marparboxwidth}
%    \begin{macrocode}
\newlength\tbminskipbtnlayers
\setlength{\tbminskipbtnlayers}{6pt}
%    \end{macrocode}
%    \end{macro}
%    \begin{macrocode}
\newif\ifiscarryover \iscarryoverfalse
%    \end{macrocode}
%    \begin{environment}{carryoverFmt}
% is a work-around for the color problem
% experienced with carry over text: Suppose there is a change of
% color of the text on the previous page, the carry over text will naturally be
% colored the default color, \texttt{MidMargcolor}. To continue the text with the same color
% as the one the previous page, we enclose the text in the
% \texttt{carryoverFmt} environment.
%    \begin{macrocode}
\newenvironment{carryOverFmt}[1]{#1\c@rryoverFmt{#1}}{}
%    \end{macrocode}
%    \end{environment}
% This command is called by the \texttt{carryOverFmt} environment.
% it takes its argument, which is a change in color or style, and
% defines \cs{tb@carryoverFmt}, which will be executed on the next page.
%    \begin{macrocode}
\def\c@rryoverFmt#1{%
    \ifx\tb@carryoverFmt\@empty
        \global\let\tb@carryoverFmt\@empty
        \xdef\tb@co@page{\thepage}%
        \gdef\tb@carryoverFmt{\ifnum\thepage>\tb@co@page
        #1\global\let\tb@carryoverFmt\@empty\fi}%
    \fi
}
\let\tb@carryoverFmt\@empty
%    \end{macrocode}
% \cs{tb@insertCarryOver} takes its argument, that is always
% \verb!\unvbox\txtbkb@xb@t}!, and if there is any carryover
% content, will insert its argument followed by a copy,
% \cs{tb@rest@reMarginFmt} of the default margin format. This seems
% to work for recovering from a change of text or style over a page
% boundary.
%    \begin{macrocode}
\let\tb@rest@reMarginFmt\relax
\def\tb@insertCarryOver#1{%
    \let\tb@rest@reMarginFmt\relax
    \ifiscarryover\ifx\tb@carryoverFmt\@empty\else
        \let\tb@rest@reMarginFmt\tb@midMargFmt
        \tb@carryoverFmt\fi\fi
    #1 \tb@rest@reMarginFmt
}
%    \end{macrocode}
%    \begin{macro}{\eqe@tb@shipout}
% We define the shipout to the margins.\par\medskip\noindent
% \textbf{Bug:} When I use \textsf{graphicxsp}, embed the picture (such as a logo),
% and use that picture as the \cs{setTopMargin}, the shipout routine is executed
% twice for each page. I haven't figured out what causes this, but here is a
% work around. We record the most recent page number, if it equals the page number
% of the last iteration of \cs{eqe@tb@shipout}, we do nothing; otherwise,
% execute the shipout code.
%    \begin{macrocode}
\newcommand{\eqe@tb@shipout}{%
    \ifnum\arabic{page}=\tblastpageshipped
    \let\tb@so@next\relax\else
    \xdef\tblastpageshipped{\arabic{page}}%
    \def\tb@so@next{\eqe@tb@ship@ut}\fi\tb@so@next
}
\def\tblastpageshipped{-100}
%    \end{macrocode}
% Here is the actual shipout code for writing to the margins.
%    \begin{macrocode}
\newcommand{\eqe@tb@ship@ut}{%
    \fboxsep=0pt\setlength{\unitlength}{1pt}%
    \global\setbox\txtbkb@xb@t=\vbox\bgroup
         \color@begingroup
         \hsize=\tbmarparboxwidth\vsize=\textheight
         \MarParBoxFmt
         \csname tbTopMargin\endcsname
         \vskip\tbminskipbtnlayers
\set@typeset@protect
         \the\txtbkt@ks
         \color@endgroup
    \egroup
    \global\setbox\txtbkb@xt@p=\vsplit\txtbkb@xb@t to\textheight
    \ifvoid\txtbkb@xb@t\global\iscarryoverfalse
    \else\global\iscarryovertrue\fi
%    \end{macrocode}
% We have three levels the top (\cs{tbTopMargin}), the bottom (\cs{tbBotMargin}),
% and the middle (\cs{txtbkt@ks}). \cs{tbTopMargin} is no problem but \cs{tbBotMargin}
% requires some special attention.
%    \begin{macrocode}
    \ifx\tbBotMargin\relax\else
%    \end{macrocode}
% If \cs{tbBotMargin} is not \cs{relax}, we begin by putting \cs{tbBotMargin}
% into a \cs{vbox} under the same assumptions, and get its height.
%    \begin{macrocode}
        \bgroup\setbox2=\vbox{%
            \hsize=\tbmarparboxwidth\kern0pt
            \MarParBoxFmt\csname tbBotMargin\endcsname
            \kern0pt
        }%
%    \end{macrocode}
% We reduce \cs{textheight} by the height of \cs{tbBotMargin}
%    \begin{macrocode}
        \dimen0=\textheight
        \advance\dimen0-\ht\txtbkb@xh@ld
        \advance\dimen0-\tbminskipbtnlayers
%    \end{macrocode}
% We split off the top material by this amount, the new bottom
% is in \cs{txtbkb@xt@p} the new top is in \cs{box0}
%    \begin{macrocode}
        \setbox0=\vsplit\txtbkb@xt@p to \dimen0
%    \end{macrocode}
% The new bottom (which will overflow to the next page) is the content
% we clipped off bottom of \cs{txtbkb@xt@p} and the original overflow
% material still in \cs{txtbkb@xb@t}.
%    \begin{macrocode}
        \global\setbox\txtbkb@xb@t=\vbox{%
            \unvbox\txtbkb@xt@p\unvbox\txtbkb@xb@t}%
%    \end{macrocode}
% We then patch everything together the new top
% is in \cs{txtbkb@xt@p} the new top is in \cs{@tempboxa} followed by
% \cs{tbBotMargin} (in \cs{box}\cs{txtbkb@xh@ld}).
%    \begin{macrocode}
        \global\setbox\txtbkb@xt@p=\vbox{\unvbox0
            \vfill\vskip\tbminskipbtnlayers
            \unvbox2\relax}%
        \egroup
    \fi
    \ifodd\value{page}
        \put(\@oddllx,\@oddlly){%
            \eqe@MarParBox{\unvbox\txtbkb@xt@p}}\else
        \put(\@evenllx,\@evenlly){%
            \eqe@MarParBox{\unvbox\txtbkb@xt@p}}\fi
%    \end{macrocode}
% We see if there is any carry over, if yes, we insert into
% \cs{txtbkt@ks} for use on the next page, along with a heading,
% if any.
%    \begin{macrocode}
    \global\txtbkt@ks={}\ifvoid\txtbkb@xb@t\else
%    \end{macrocode}
% We test whether these is a \cs{splitbotmark}, if yes, then we will
% insert it at the top of the next page with formatting.
%    \begin{macrocode}
    \if!\splitbotmark!\global\let\tb@sbm@exp\relax\else
        \xdef\tb@sbm@exp{\noexpand\tbPreMarginHeader
            \noexpand\tbMarginHeaderFmt{\splitbotmark}%
            \noexpand\tbPostMarginHeader
            \noexpand\par\kern3pt}%
    \fi
%    \end{macrocode}
% Here is the content that will be carried over to the next page,
% we insert a \cs{splitbotmark} if it is non-empty (\cs{tb@tmp@exp}).
%    \begin{macrocode}
        \global\txtbkt@ks=\expandafter{\tb@sbm@exp
        \tb@insertCarryOver{\unvbox\txtbkb@xb@t}}%
    \fi
}
%    \end{macrocode}
%    \end{macro}
%    \begin{macro}{\insertpageifcarryover}
% This macro is use to generate a blank page if there is carry over from the
% previous page. It is place just after the exercises, and before a new chapter of section.
% The optional argument allows you to insert something into the new page, if  one is
% automatically created. The default is \cs{null}.
%    \begin{macrocode}
\newcommand{\insertpageifcarryover}[1][\null]{%
%    \end{macrocode}
% We begin by starting a new page, the shipout routine of previous page
% will be initialized and can then get an accurate result for
% \cs{ifiscarryover}.
%    \begin{macrocode}
    \newpage
%    \end{macrocode}
% If there is carryover, we create a new page by inserting
% a content into the page.  If there is no carry over, we do
% now insert any content, and the page will not be created.
%    \begin{macrocode}
    \ifiscarryover\def\eqeifnext{\csname iftrue\endcsname}%
    \PackageInfo{eqexam}{Carry over of content in margin
    from page \thepage.\MessageBreak Creating a blank page}\else
    \def\eqeifnext{\csname iffalse\endcsname}\fi\eqeifnext#1\fi}
%    \end{macrocode}
%    \end{macro}
%    \begin{macro}{\setFullWidthHeader}
% Makes the running header full width.
%    \begin{macrocode}
\newcommand{\setFullWidthHeader}{%
    \setlength{\@tempdima}{%
        \evensidemargin+\tbmarparboxwidth+\marginparsep}%
    \edef\@headoffset{\the\@tempdima}%
    \def\@evenhead{\makebox[0pt]{\makebox[0pt][l]
        {\thepage}\hspace{\@headoffset}}\hfil\slshape\leftmark}%
    \ifmarginsonleft
        \def\@oddhead{\makebox[0pt]{\makebox[0pt][l]
            {\slshape\rightmark}\hspace{\@headoffset}}\hfil\thepage}%
    \else
        \def\@oddhead{{\slshape\rightmark}\hfil\makebox[0pt]
            {\hspace{\@headoffset}\makebox[0pt][r]{\thepage}}}%
    \fi
}
%    \end{macrocode}
%    \end{macro}
%
% \subsection{In support of solutions at end of document and chapter}
%
% A feature that may not be used much is to have solutions at the end of each chapter.
%    \begin{macro}{\chaptersolutions}
% If \cs{tb@EndOfChapterExercises} is executed, and
% \cs{chaptersolutions} is placed between chapters, we can generate
% solutions at the end of the chapters, instead of at the end of the book.
% \cs{chaptersolutions} is \cs{let} to \cs{relax} unless
% \cs{tb@EndOfChapterExercises} is executed. In this case
% \cs{chaptersolutions} inputs the the \texttt{.sol} file, then
% then opens it
%    \begin{macrocode}
\newif\ifchapterexercises \chapterexercisesfalse
\let\chaptersolutions\relax
\def\tb@EndOfChapterExercises{%
    \let\include@solutions@chapter\include@solutions
    \def\includeexersolutions{%
        \include@solutions@chapter
          \global\let\include@solutions\relax
    }%
%    \end{macrocode}
% \cs{chaptersolutions} is redefined from \cs{relax}. Input current solutions,
% close stream, open stream.
%    \begin{macrocode}
    \def\chaptersolutions{%
        \includeexersolutions
        \immediate\closeout\ex@solns
        \newwrite \ex@solns \global\let\quiz@solns\ex@solns
        \immediate\openout \ex@solns \jobname.sol
        \ifvspacewithsolns\writeAllAnsAtEnd\fi
    }%
}
%    \end{macrocode}
%    \end{macro}
%    \begin{macrocode}
\def\writeallsolutions{\let\chaptersolutions\relax}
%    \end{macrocode}
%    \begin{macro}{\exercisesAtEndOfChapter}
% If you want solutions at the end of each chapter, you'll have to
% execute this command in the preamble. See \cs{initChapAfterSolns} for an
% example of usage.
%    \begin{environment}{afterChapSolns}
% This comment environment is a convenience for placing content between
% chapters.
%    \begin{macrocode}
\excludecomment{afterChapSolns}
\includecomment{solnsAtEnd}
\newcommand{\exercisesAtEndOfChapter}{%
    \ifeq@nosolutions\else
        \typeout{^^J!!!!!Executing in chapter solutions!!!!!^^J}
        \chapterexercisestrue\tb@EndOfChapterExercises
        \ifchapterexercises
        \csarg\let{solnsAtEnd}\@gobble
        \excludecomment{solnsAtEnd}%
        \csarg\let{AftersolnsAtEndComment}\relax
        \includecomment{afterChapSolns}\else
        \excludecomment{afterChapSolns}\fi
    \fi
}
\@onlypreamble\exercisesAtEndOfChapter
%    \end{macrocode}
%    \end{environment}
%    \end{macro}
% \subsection{Modifying and restoring the Layout}
% The book may need a wide page format and use multi-columns to display homework sets, or
% solutions at the end if the book.
%    \begin{macro}{\setFullWidthLayout}
% A command to set the page layout for the solutions in the back of the book. Typically,
% we do away with the wide margins. We also save the current values of the parameters
% we are changing so we can restore them later.
%    \begin{macrocode}
\newcommand{\setFullWidthLayout}{%
    \saveBasicLayoutParams
    \setlength{\oddsidemargin}{0in}%
    \setlength{\evensidemargin}{\oddsidemargin}%
    \setlength{\textwidth}{\paperwidth-2in}%
    \setlength{\linewidth}{\paperwidth-2in}%
    \setlength{\columnseprule}{0pt}%
    \def\@evenhead{\thepage\hfil\slshape\leftmark}%
    \def\@oddhead{{\slshape\rightmark}\hfil\thepage}%
}
%    \end{macrocode}
%    \begin{environment}{fullwidthtext}
% When \cs{setFullWidthLayout} is in effect, we have the problem of writing text.
% Originally, I used a \cs{parbox} with width of \cs{linewidth}, but this has it problems
% when breaking across pages. We have instead an environment for writing; the list environment
% obeys the current \cs{linewidth}, which is set to \cs{paperwidth-2in}, this latter value
% may not always be correct (especially when the margins are smaller than 2in.
%    \begin{macrocode}
\newenvironment{fullwidthtext}{%
\begin{list}{}{%
    \setlength{\labelwidth}{0pt}\setlength{\labelsep}{0pt}%
    \setlength{\itemindent}{0pt}\setlength{\itemsep}{0pt}%
    \setlength{\topsep}{0pt}\setlength{\parsep}{0pt}%
    \setlength{\listparindent}{\parindent}%
    \setlength{\leftmargin}{0pt}\setlength{\rightmargin}{0pt}
}\item\relax}{\end{list}}
%    \end{macrocode}
%    \begin{macro}{\restorePageLayout}
% Restore the last saved page parameters.
%    \begin{macrocode}
\newcommand{\restorePageLayout}{\newpage
    \setlength{\oddsidemargin}{\tb@osms}
    \setlength\evensidemargin{\tb@esms}
    \setlength{\textwidth}{\tb@tws}
    \setlength{\linewidth}{\tb@lws}
    \setlength{\columnseprule}{\tb@csr}
}
%    \end{macrocode}
% Used by \cs{setFullWidthLayout} just before the page layout parameters are changed.
%    \begin{macrocode}
\newcommand{\saveBasicLayoutParams}{%
    \xdef\tb@osms{\the\oddsidemargin}%
    \xdef\tb@esms{\the\evensidemargin}%
    \xdef\tb@tws{\the\textwidth}%
    \xdef\tb@lws{\the\linewidth}%
    \xdef\tb@csr{\the\columnseprule}%
}
%    \end{macrocode}
%    \begin{macro}{\initChapAfterSolns}
% Initializes the environment when solutions appear after each chapter.
% Example of usage, taken from fortextbook.ltx,
%\begin{verbatim}
%\begin{afterChapSolns}
%\initChapAfterSolns
%\section{Solutions to Chapter Exercises}
%\begin{fullwidthtext}
%We present short solutions to the problems.
%We present short solutions to the problems.
%We present short solutions to the problems.
%We present short solutions to the problems.
%\end{fullwidthtext}
%\bigskip
%\begin{multicols}{2}\forceNoColor
%\chaptersolutions
%\end{multicols}
%\restoreFromChapAfterSolns
%\end{afterChapSolns}
%\end{verbatim}
%    \begin{macrocode}
\newcommand{\initChapAfterSolns}{\newpage
    \clearTopMargin\clearBotMargin
    \setFullWidthLayout
}
%    \end{macrocode}
%    \begin{macro}{\restoreFromChapAfterSolns}
% Restores the saved parameters at the end of the chapter solutions, see
% above for an example.
%    \begin{macrocode}
\newcommand{\restoreFromChapAfterSolns}{\newpage
    \restorePageLayout\setFullWidthHeader
}
%    \end{macrocode}
%    \end{macro}
%    \end{macro}
%    \end{macro}
%    \end{environment}
%    \end{macro}
%    \begin{macrocode}
%</textbook>
%<*package>
%    \end{macrocode}
% \subsection{We shipout in support of \texttt{fortextbook}}
% We shipout \cs{eqe@tb@shipout} to be placed in the margins on every page.
%    \begin{macrocode}
\ifeqfortextbook
\AtBeginDocument{\tb@soln@choice
    \ifeqwritetomargins\chkmarginboxwidth
    \AddToShipoutPicture{\eqe@tb@shipout}\fi}
\fi
%</package>
%<*textbook>
%    \end{macrocode}
%
% \subsection{Modify \texttt{eqequestions} environment}
%
% We adjust the \texttt{eqequestions} environment to minimize spacing between problems.
%    \begin{macrocode}
\eqequestopsep{0pt}
\eqequesparsep{0pt}
\eqequesitemsep{3pt}
\renewenvironment{eqequestions}{%
    \begin{list}{}{%
    \setlength{\labelwidth}{\eqemargin}%
    \setlength{\topsep}{\eqeques@topsep}%
    \setlength{\parsep}{\eqeques@parsep}%
    \setlength{\itemsep}{\eqeques@itemsep}
    \setlength{\itemindent}{0pt}%
    \ifwithsoldoc\settowidth{\labelsep}{\eqe@hspannerSoln}\else
    \settowidth{\labelsep}{\eqe@hspannerPrb}\fi
    \setlength{\leftmargin}{\labelwidth}%
    }\item\relax}{\end{list}}
%    \end{macrocode}
%    \begin{environment}{eqepartsquestions}
%    \begin{macrocode}
\newenvironment{eqepartsquestions}{%
    \begin{list}{}{%
    \settowidth{\labelwidth}{\eqe@prtsepSoln\hspace{\tbsolnpartwdth}}
    \setlength{\topsep}{\eqeques@topsep}%
    \setlength{\parsep}{\eqeques@parsep}%
    \setlength{\itemsep}{\eqeques@itemsep}%
    \setlength{\itemindent}{0pt}%
    \settowidth{\labelsep}{\eqe@hspannerSoln}
    \setlength{\leftmargin}{\labelwidth}%
    }\item\relax}{\end{list}}
%    \end{macrocode}
%    \end{environment}
%
% \subsection{Modifications for solutions page}
%
% \DescribeMacro{\gobbletoEndeqExt} is a command to gobble all content from the current position \cs{eqEXt}
% down to \cs{endeqEXt}. In the solutions file ends with \cs{par}\cs{medskip}, which
% we gobble up too. We define \DescribeMacro{\eqExtArg}\cs{eqExtArg} to \cs{thequestionno} so we can use the problem
% number to filter out the even-problems.
%    \begin{macrocode}
\long\def\gobbletoEndeqExt#1\endeqEXt{\@gobbletwo}
\def\eqExtArg{\theeqquestionnoi}
%    \end{macrocode}
%    \begin{macrocode}
\if\load@exerquiz n\DoNotFitItIn\fi
\if\eq@usexkeys y\newcommand{\fillInFormatDefault}{}\fi
\def\exerSolnsHeadnToc{}
\renewcommand{\exerSolnInput}{%
    \let\webnewpage\relax
    \ifsolutionsonly\else\immediate\closeout\ex@solns\fi
    \ifeq@nosolutions\else\newpage % 2012-03-14
        \eqsolutionshook
        \iftherearesolutions\ifsolutionsonly\else\newpage\fi
            \ifx\webnewpage\relax
                \def\webnewpage{\let\webnewpage\newpage}%
            \fi
            \priorexsectitle\exerSolnsHeadnToc\priorexslinput
            \InputIfFileExists{\jobname.sol}{}{\PackageWarning{exerquiz}
            {!!! Solutions to exercises not found}}%
        \fi
    \fi
}
%    \end{macrocode}
%    \begin{macro}{\eqedsplyOnlyFrst}
% The default listing of a problem with multiple parts is to typeset
% \texttt{<num>.}~\texttt{(<part>)}. Here, we do not typeset the number after
% the first time.
%    \begin{macrocode}
\setcounter{partno}{1}\edef\firstPartLtr{\thepartno}
%    \end{macrocode}
% ???? 6/2/11 When part (a) is hidden we need to generate the questions number
% for the the first non-hidden part. Created \cs{iffrstProbNumShown} to help
% but it not working yet.
%    \begin{macrocode}
\newif\iffrstProbNumShown\frstProbNumShownfalse
\def\tb@insertDecPoint{\ifwithsoldoc\eqedecPointSoln\else
    \eqe@decPointMrg\fi}
\newcommand{\eqedsplyOnlyFrst}[2]{\def\thisPart{#2}%
    \ifx\thisPart\firstPartLtr\global\frstProbNumShowntrue
        \tb@mrgDigitFmt{#1}\tb@insertDecPoint\else
        \iffrstProbNumShown\tb@GenProbNum{#1}\else
        \global\frstProbNumShowntrue\tb@mrgDigitFmt{#1}%
        \tb@insertDecPoint\fi\fi\global\eqeGenProbNumfalse
}
%    \end{macrocode}
%    \end{macro}
%    \begin{macro}{\displayProbNumOnce}
% If a part is carried over to the next page, it may be necessary to manually
% force the display of the first digit.
%\begin{verbatim}
%\insMidMarg{\displayProbNumOnce}
%\end{verbatim}
%    \begin{macrocode}
\newif\ifeqeGenProbNum \eqeGenProbNumfalse
\newcommand{\displayProbNumOnce}{\global\eqeGenProbNumtrue} %
%\def\tb@GenProbNum#1{\ifeqeGenProbNum#1\eqe@decPointMrg\else
\def\tb@GenProbNum#1{\ifeqeGenProbNum#1\tb@insertDecPoint\else
    \phantom{#1\tb@insertDecPoint}\fi}%
%    \end{macrocode}
%    \end{macro}
%    \begin{macrocode}
\def\sq@priorhook{\medskip}
%    \end{macrocode}
% Adjustments of spacing between problems \cs{eqexerskip}, and the check for enough
% room for the next problem.
%    \begin{macrocode}
\def\default@fvsizeskip{.1}
%    \end{macrocode}
% The skip prior to the beginning of an exercise
%    \begin{macrocode}
\priorexskip{0pt}
%    \end{macrocode}
% The skip after the end of an exercise
%    \begin{macrocode}
\eqexerskip{0pt}
%    \end{macrocode}
% The skip in the solutions file following an exercise OR a part of an exercise
% The text of this command should be a single token, that's why I've enclosed
% it in braces. (There is a \cs{@gobbletwo} that gobbles it up for the \texttt{studented} option.)
%    \begin{macrocode}
\renewcommand\eqafterexersolnskip{{}}
%    \end{macrocode}
% We remove the \cs{mark} from this definition, see original definition in \texttt{eqexam.def}
%    \begin{macrocode}
\renewcommand\exerSolnHeader[3]{%
    \ifeqforpaper\else\webnewpage\fi\par
    \noindent\@ifundefined{hypertarget}
    {#3}{\hypertarget{#2}{#3}\relax}\solnhspace
}
%    \end{macrocode}
% This causes the \texttt{eqexam} environment to write the user friendly name of the exam
% even if there is only one exam.
%    \begin{macrocode}
\def\nNumberOfP@rts{0}
%    \end{macrocode}
%
% \subsection{Some Convenience/Formatting Commands}
%
%    \begin{macro}{\preExamSolnHead}
%    \begin{macro}{\examSolnHeadFmt}
%    \begin{macro}{\postExamSolnHead}
% These are redefinitions of commands defined in \texttt{eqexam},
% They control the vertical spacing before and after a heading in the
% solutions at the end of the book, as well as the formatting.
%    \begin{macrocode}
\renewcommand{\preExamSolnHead}{\medbreak\noindent}
\renewcommand{\examSolnHeadFmt}[1]{\textbf{#1}}
\renewcommand{\postExamSolnHead}{\smallbreak}
%    \end{macrocode}
%    \end{macro}
%    \end{macro}
%    \end{macro}
%    \begin{macro}{\wrtChapSolnHead}
% Writes a chapter heading to the solution file, usage,
%\begin{verbatim}
%\wrtChapSolnHead{The New {\eqexam}}
%\end{verbatim}
%    \begin{macrocode}
\newcommand{\wrtChapSolnHead}[1]{%
    \writeToSolnFile{%
    \protect\preChapSolnHead
    \protect\chapHeadSolnFmt{\protect\ftbFmtChapter{\thechapter}#1}%
    \protect\postChapSolnHead
}}
%    \end{macrocode}
%    \begin{macro}{\preChapSolnHead}
%    \begin{macro}{\chapHeadSolnFmt}
%    \begin{macro}{\postChapSolnHead}
% Same as above, except for chapter headings.
%    \begin{macrocode}
\newcommand{\preChapSolnHead}{\bigbreak\noindent}
\newcommand{\chapHeadSolnFmt}[1]{{\large\textbf{#1}}}
\newcommand{\postChapSolnHead}{\medbreak}
%    \end{macrocode}
%    \begin{macro}{\ftbFmtChapter}
% This command may (optionally) insert the chapter number into the chapter title passed
% to \cs{wrtChapSolnHead}. The default is to pass the chapter name (``Chapter'') and chapter number.
% If you say \verb!\let\ftbFmtChapter\@gobble!, the chapter name and number will not appear.
% You can redefine this command as desired.
%    \begin{macrocode}
\newcommand{\ftbFmtChapter}[1]{\chaptername\space#1.\space\ignorespaces}
%    \end{macrocode}
% In the solution manual, all these
%chapter commands may be redefined like so
%\begin{verbatim}
%\let\preChapSolnHead\relax
%\let\chapHeadSolnFmt\chapter
%\let\ftbFmtChapter\@gobble
%\let\postChapSolnHead\relax
%\end{verbatim}
% In fact, let's make this into a command.
%    \begin{macro}{\convertChapHeadToChapters}
%In the solutions manual, the chapter headings will become chapters of the manual, rather than
%just a bold faced heading.
%    \begin{macrocode}
\newcommand{\convertChapHeadToChapters}{%
    \let\preChapSolnHead\relax
    \let\chapHeadSolnFmt\chapter
    \let\ftbFmtChapter\@gobble
    \let\postChapSolnHead\relax
}
%    \end{macrocode}
%    \end{macro}
%    \end{macro}
%    \end{macro}
%    \end{macro}
%    \end{macro}
%    \end{macro}
%    \begin{macro}{\probSet}
% A simple command to announce the problem set.
%\begin{verbatim}
%\subsection*{\probSet{\thesection}}
%\end{verbatim}
% See also the definition for the \texttt{probset} environment below.
%    \begin{macrocode}
\newcommand{\probSet}[1]{Problem Set #1}
%    \end{macrocode}
%    \end{macro}
%    \begin{macro}{\annotPage}
% Use to annotation the page number onto a solution heading, for example,
%\begin{verbatim}
%\begin{exam}[\thesection. Another Section\annotPage]{\autoExamName}
%\end{verbatim}
% or using the \texttt{probset} environment defined below
%\begin{verbatim}
%\begin{probset}{{\thesection} Setting the page layout\annotPage}
%\end{verbatim}
%    \begin{macrocode}
\newcommand{\annotPage}{\protect\annotThePage{\thepage}}
\newcommand{\annotThePage}[1]{\space(page\protect~#1)}
%    \end{macrocode}
%    \end{macro}
% \subsection{The \texttt{probset} and \texttt{example} environments}
% We define two environments based. The first (\texttt{probset}) is based on the
% \texttt{exam} environment; the second (example) is based on the \texttt{exercise} environment.
%    \begin{environment}{probset}
% A convenience environment, it is the \texttt{exam} environment, renamed, with
% different arguments. \texttt{\#1} is the heading that will appear in the margins,
% and \texttt{\#2} is the heading to appear in the back of the book.
%\changes{v3.0n}{2011/09/18}{Added an \cs{edef} in case \cs{thesection} does not get
% expanded early enough to display correctly in the margins.}
%    \begin{macrocode}
\def\noProbHeader{NPH}
\newenvironment{probset}[2][\probSet{\thesection}]{%
    \exam[#2]{\autoExamName}\ifx#1\noProbHeader\else
    \protected@edef\ftb@tmp@exp{\noexpand\insProbHead{#1}}%
    \ftb@tmp@exp\fi}{\endexam}
%    \end{macrocode}
%    \end{environment}
%    \begin{environment}{example}
% A simple example environment, based on the \texttt{exercise} environment.
%    \begin{macrocode}
\newcounter{exampleno}[section]
\renewcommand{\theexampleno}{\arabic{section}.\arabic{exampleno}}
\newenvironment{example}{\medskip
    \renewcommand\exlabel{Example}%
    \renewcommand\exlabelformat{\textbf{\exlabel~\theexampleno.}}%
    \let\eq@fititin\eqfititin
    \renewcommand\exrtnlabelformat{$\square$}%
    \def\eqexheader@wrapper{\eqexheader}%
    \SolutionsAfter
    \begin{exercise}[exampleno]}{\end{exercise}}
%    \end{macrocode}
%    \end{environment}
%    \begin{environment}{example*}
% An example environment with parts.
%    \begin{macrocode}
\newenvironment{example*}{\medskip
    \renewcommand\exlabel{Example}%
    \renewcommand\exlabelformat{\textbf{\exlabel~\theexampleno:}}%
    \let\eq@fititin\eqfititin
    \renewcommand\exrtnlabelformat{$\square$}%
    \def\eqexheader@wrapper{\eqexheader}%
    \SolutionsAfter
    \begin{exercise*}[exampleno]}{\end{exercise*}}
%    \end{macrocode}
%    \end{environment}
% We set some parameters, to values better suited for the option.
%    \begin{macrocode}
\setDefaultfvsizeskip{.1}
\nbaselineskip{4}
%    \end{macrocode}
% \subsection{Commands in support of Solution Manuals}
% Generally, the solution manual source file should have the same
% packages as the source file for the book itself, perhaps with a few exceptions,
% but definitely the \textsf{eqexam} package is required with its \texttt{fortextbook} option.
%
% At this time, we provide only two commands, these are \cs{ftbInputBookAux} and
% \cs{ftbInputSolnFiles}.
%    \begin{macro}{\ftbInputBookAux}
% This command is used to input the auxiliary files of the master source file. It takes
% one argument, the name of the master source file (\texttt{myBook.ltx} or \texttt{myBook.tex}).
% If the extension is not present, it is assumed to be \texttt{.tex}.
%\changes{v3.0f}{2011/08/13}{Added \cs{ftbInputBookAux} to support solution manual}
%    \begin{macrocode}
\newcommand{\ftbInputBookAux}[1]{%
    \filename@parse{#1}\@ifundefined{filename@ext}%
        {\def\filename@ext{tex}}{}%
    \xdef\tbBaseName{\filename@base}%
    \xdef\tbSourceFile{\filename@base.\filename@ext}%
%    \end{macrocode}
% In the next 4 lines, we save \cs{@writefile} and \cs{@setckpt}, and \cs{let}
% them to \cs{@gobbletwo}. We restore their definitions after we input the aux files.
% We include the aux files of the source file in case there are some cross references
% in the solution files, or the body of the text would like to refer back to the
% original book. (Seems unlikely.)
%    \begin{macrocode}
    \let\save@writefile\@writefile
    \let\@writefile\@gobbletwo
    \let\save@setckpt\@setckpt
    \let\@setckpt\@gobbletwo
    \makeatletter
    \InputIfFileExists{\tbBaseName.aux}{%
        \PackageInfo{eqexam}
            {Inputting auxiliary files of\MessageBreak\tbSourceFile}%
        }{%
        \PackageError{eqexam}
            {Auxiliary files for \tbSourceFile\space were not found}
            {Compile the source file \tbSourceFile\space
                three times\MessageBreak%
                to create the required auxiliary files.}%
        }%
    \makeatother
%    \end{macrocode}
% The solution files really shouldn't have a label, but if we do
% we'll save the {\LaTeX} definition of \cs{label}, and \cs{let}
% it two \cs{@gobble}. Within the body of the solutions, the
% command \cs{ftblabel} may be used to cross reference, if needed.
%
%    \begin{macrocode}
    \global\let\ftblabel\label
    \let\label\@gobble
    \let\@writefile\save@writefile
    \let\@setckpt\save@setckpt
}
\@onlypreamble\ftbInputBookAux
%    \end{macrocode}
%    \begin{macro}{\restorelabel}
%    \begin{macro}{\gobblelabel}
% These two are used to restore the usual definition of \cs{label}, and to
% cancel it out by letting it to \cs{@gobble}.
%    \begin{macrocode}
\newcommand{\restorelabel}{\global\let\label\ftblabel}
\newcommand{\gobblelabel}{\let\label\@gobble}
%    \end{macrocode}
%    \end{macro}
%    \end{macro}
%    \end{macro}
%    \begin{macro}{\ftbInputSolnFiles}
% In the body of the text, place \cs{ftbInputSolnFiles} in vertical mode.
% This will input the \texttt{.sol} file of the master source document.
% The optional argument is the name of the solution file. The default name
% is \verb!\tbBaseName.sol!, where \cs{tbBaseName} was defined in
% \cs{ftbInputBookAux}. If no extension is specified, an extension of \texttt{.sol}
% is assumed.  The original \texttt{.sol} may have changed its name, if someone
% renamed it (to keep it from being overwritten). The solution file may be editing (by hand)
% as needed.
%\changes{v3.0f}{2011/08/13}{Added \cs{ftbInputSolnFiles} to support solution manual}
%    \begin{macrocode}
\newcommand{\ftbInputSolnFiles}[1][\tbBaseName.sol]{%
    \filename@parse{#1}\@ifundefined{filename@ext}%
        {\def\filename@ext{sol}}{}%
    \xdef\tbBaseName{\filename@base}%
    \xdef\tbSourceFile{\filename@base.\filename@ext}%
    \InputIfFileExists{\tbBaseName.sol}{%
        \PackageInfo{eqexam}
            {Inputting solutions file \tbBaseName.sol\MessageBreak}%
        }{%
        \PackageError{eqexam}
            {Solutions file for \tbSourceFile\space was not found}%
            {Compile the source files three times}%
        }%
}
%    \end{macrocode}
%    \end{macro}
%    \begin{macrocode}
%</textbook>
%<*ftbsty>
%    \end{macrocode}
% \section{\textsf{fortextbook} Style File}\label{fortextbookstyle}
% One person said it would be nice to separate \textsf{eqexam} from the \texttt{fortextbook} option, and
% have \textsf{fortextbook} as a separate style (package). Rather than spending tens of hours separating them
% I create a simple ``wrapper'' package, which simply calls \textsf{eqexam} with the \texttt{fortextook} option along
% with all the recommended options.
%\changes{v3.0p}{2011/09/22}{Added the wrapper package fortextbook.}
%\par\medskip\noindent
% \textbf{Usage:}
%\begin{verbatim}
%\documentclass[twoside,letterpaper]{book}
%\usepackage[fleqn]{amsmath}
%\usepackage{fortextbook}
%...
%\end{verbatim}
% Below is the style.
%    \begin{macrocode}
\NeedsTeXFormat{LaTeX2e}
\ProvidesPackage{fortextbook}
 [2012/03/14 v1.0 A fortextbook Package (dps)]
\DeclareOption*{\PassOptionsToPackage{\CurrentOption}{eqexam}}
\ProcessOptions
\RequirePackage[%
    ftbsolns,fortextbook,usecustomdesign,
    forcolorpaper,noseparationrule,usexkv
]{eqexam}
%    \end{macrocode}
% In support of this style, I've also defined \cs{NoSolutions} to compile the document
% without creating the solutions at the end of the file (this reduces the amount if IO
% when compiling). I've also defined a special option \texttt{nocustomdesign} which
% cancels out the \texttt{usecustomdesign} option.
%    \begin{macrocode}
%</ftbsty>
%<*package>
%    \end{macrocode}
% \begin{center}
%   \rule{.67\linewidth}{.4pt}
% \end{center}
% \paragraph*{Input \texttt{eqtextb.def}.}
% Back in the main package, we choose this point to input the
% \texttt{fortextbook} code (\texttt{eqtextb.def}) if the
% \texttt{fortextbook} option is taken.
%    \begin{macrocode}
\edef\ftbInputEqTextb{\ifeqfortextbook\noexpand
    \InputIfFileExists{eqtextb.def}{}{}\fi}
\ftbInputEqTextb
%    \end{macrocode}
% \begin{center}
%   \rule{.67\linewidth}{.4pt}
% \end{center}
%
% \section{\textsf{xkeyval} Extensions}
%
% We load this material if \textsf{xkeyval} exists, and if the document author has specified
% the \texttt{usexkv} option.
%
%    \begin{macrocode}
\IfFileExists{xkeyval}{%
    \if\eq@usexkeys y\RequirePackage{xkeyval}\else
    \endinput\fi}{\endinput}
%    \end{macrocode}
% We redefine selected commands if the user has specified the \texttt{usexkv} option.
%
%\paragraph*{New options for \cs{fillin}}
%    \begin{macro}{underline}
% Underline the fillin
%    \begin{macro}{u,b}
% Legacy parameters, underlines (\texttt{u}) or leaves a blank space (\texttt{b})
%    \begin{macro}{boxed}
% Boxes in the response region
%    \begin{macro}{boxpretext}
% When boxed is use, use this to insert text in front of the answer, for example, \texttt{x=}
%    \begin{macro}{boxsize}
% When boxed is taken, use boxsize to set the size of the box; permissible choices are
% tiny, scriptsize, footnotesize, small, normalsize, large, Large, LARGE, huge, Huge
%    \begin{macro}{align}
% Align the answer within the response region, permissible values are \texttt{l}, \texttt{c}, \texttt{r}.
%    \begin{macro}{color}
% The color of the response (named color)
%    \begin{macro}{format}
% Special formatting for the answer, the default is \cs{bfseries}
%    \begin{macro}{enclosesoln}
% This Boolean key only takes effect when the
%      \texttt{boxed} key is used, and when either the \texttt{nosolutions} or the
%      \texttt{vspacewithsolns} option is taken. When these conditions are
%      met, a box is created around the solution (the third parameter of \cs{fillin}); the solution
%      is enclosed in a \cs{phantom} so it is not seen, but the dimensions of the solution are used.
%      This key allows you to create a box or arbitrary dimension.
%\changes{v2.0j}{2011/04/19}{added the \texttt{enclosesoln} key to \texttt{eqFillin} family.}
%    \begin{macro}{fitwidth}
%\changes{v3.0i}{2011/08/18}{added the \texttt{fitwidth} key to \texttt{eqFillin} family.}
% The \texttt{fitwidth} option uses the natural width of the answer to create the fillin
% when the \texttt{answerkey} option is in effect; otherwise it uses the second parameter \texttt{\#2}.
%    \begin{macro}{parbox}
%    The \texttt{parbox} parameter may be used to create a multiline \cs{fillin}
%    box. The value of \texttt{parbox} is the same as the first three parameters
%    of the {\LaTeX} command \cs{parbox}, e.g., \verb!parbox={[t][.5in][t]}!.
%    The value needs to be enclosed in braces.
% \changes{v3.0w}{2012/03/27}{Added \texttt{parbox}}%
%    \begin{macro}{hiddenbox}
%    When the \texttt{boxed} option is used, this option resets the \cs{fbox}
%    parameters to \texttt{0pt}, making the box ``hidden.''
%\changes{v3.0w}{2012/03/27}{Added \texttt{hiddenbox} options}
%\par\medskip\noindent
% Below are the \textsf{xkeyval} definitions of the keys recognized by \cs{fillin}.
%    \begin{macrocode}
\define@boolkey{eqFillin}{underline}[true]{}
\define@key{eqFillin}{u}[]{\KV@eqFillin@underlinetrue}
\define@key{eqFillin}{b}[]{\KV@eqFillin@underlinefalse}
\define@boolkey{eqFillin}{boxed}[true]{}
\define@key{eqFillin}{boxpretext}[]{\def\eq@fillintext{#1}}
\let\eq@fillintext\@empty
%    \end{macrocode}
%    If the user just says \texttt{parbox,...} the value of
%    \cs{eq@fillinparbox} is \cs{relax}. If \texttt{parbox}
%    does not appear in the option list, \cs{eq@fillinparbox}
%    has a default value of \cs{@empty}. In this way, we can
%    distinguish between \texttt{parbox} with the empty value,
%    and \texttt{parbox} not present at all.
%    \begin{macrocode}
\define@key{eqFillin}{parbox}[\relax]{\def\eq@fillinparbox{#1}}
\let\eq@fillinparbox\@empty
\define@key{eqFillin}{hiddenbox}[]{%
    \def\eq@fillinhiddenbox{%
        \setlength{\fboxrule}{0pt}\setlength{\fboxsep}{0pt}}}
\let\eq@fillinhiddenbox\@empty
\define@boolkey{eqFillin}{enclosesoln}[true]{}
\define@choicekey+{eqFillin}{boxsize}{tiny,scriptsize,footnotesize,%
    small,normalsize,large,Large,LARGE,huge,Huge}[normalsize]{%
    \def\eq@eqFillin@boxsize{\text{\csname#1\endcsname\strut}}%
}{\PackageWarning{eqexam}{Bad choice for boxsize, permissible values
    are tiny, scriptsize, footnotesize, small, normalsize,
    large, Large, LARGE, huge and Huge. Try again}}
\def\eq@eqFillin@boxsize{\text{\normalsize\strut}}
\define@key{eqFillin}{color}[\eq@fillinColor]{\edef\eq@fillin@color{#1}}
%    \end{macrocode}
% \changes{v3.0x}{2012/04/03}{Added \cs{eqe@align@hfill} to \texttt{align} property.
% used to set position of content when \texttt{parbox} is used.}
%    \begin{macrocode}
\define@choicekey+{eqFillin}{align}[\val\nr]%
    {l,r,c}[\eq@eqFillin@align@default]{%
    \def\eq@eqFillin@align{#1}%
    \ifcase\nr\relax
        \def\eqe@align@hfill{}\or
        \def\eqe@align@hfill{\hfill}\or
        \def\eqe@align@hfill{\hfil}\fi
    }{%
    \PackageWarning{eqexam}{Bad choice for align, permissible values
    are l, r, and c. Try again}}
\let\eqe@align@hfill\relax
%    \end{macrocode}
% \DescribeMacro{defaultalign} is used to change the values of the default macros
% \cs{eq@eqFillin@align@default} and \cs{eqe@align@hfill@default} for the \texttt{align} key together.
%    \begin{macrocode}
\define@choicekey+{eqFillin}{defaultalign}[\val\nr]{l,r,c}[c]{%
    \def\eq@eqFillin@align@default{#1}%
    \ifcase\nr\relax
        \def\eqe@align@hfill@default{}\or
        \def\eqe@align@hfill@default{\hfill}\or
        \def\eqe@align@hfill@default{\hfil}\fi
    }{%
    \PackageWarning{eqexam}{Bad choice for defaultalign, permissible
    values are l, r, and c. Try again}}
\setkeys{eqFillin}{defaultalign=c}
%    \end{macrocode}
% \DescribeMacro{\fillInFormatDefault} is the default fill-in format
%    \begin{macrocode}
\providecommand{\fillInFormatDefault}{\normalfont}
\define@key{eqFillin}{format}[\fillInFormatDefault]{%
    \def\eq@fillin@format{#1}}
\edef\eq@fillin@format{\bfseries}
\def\eqe@fbox@corr#1{#1-2\fboxsep-2\fboxrule}
\define@boolkey{eqFillin}{fitwidth}[true]{} %
%    \end{macrocode}
%    \end{macro}
%    \end{macro}
%    \end{macro}
%    \end{macro}
%    \end{macro}
%    \end{macro}
%    \end{macro}
%    \end{macro}
%    \end{macro}
%    \end{macro}
%    \end{macro}
%    \end{macro}
%    The macro \cs{eqe@getiiiOpts} is based on early parsing code of \cs{parbox}.
%    It picks up three optional parameters and saves their values under the
%    commands \cs{eqe@opts@argi}, \cs{eqe@opts@argii}, \cs{eqe@opts@argiii}.
%    We are interested in \cs{eqe@opts@argiii}, which specifies the depth of
%    the \cs{parbox}. If the \texttt{boxed} option is taken, we reduce the value
%    of \cs{eqe@opts@argiii} by \texttt{2\string\fboxsep+2\string\fboxrule} so
%    that the height will be exactly as specified. The macro \cs{eqe@getiiiOpts}
%    is used with the \texttt{parbox} option of \cs{fillin}.
%    The macro \cs{eqe@getiiiOpts} has syntax:
%\begin{quote}
%   \cs{eqe@getiiiOpts[\meta{pos}][\meta{height}][\meta{inner-pos}]}
%\end{quote}
%    \begin{macrocode}
\def\eqe@getiiiOpts{%
    \@ifnextchar[%]
    \i@eqe@getiiiOpts
    {\iii@eqe@getiiiOpts{c}{\relax}[s]}}
%    \end{macrocode}
% Get \meta{pos}
%    \begin{macrocode}
\def\i@eqe@getiiiOpts[#1]{%
    \@ifnextchar[%]
    {\ii@eqe@getiiiOpts{#1}}%
    {\iii@eqe@getiiiOpts{#1}{\relax}[s]}}
%    \end{macrocode}
% Get \meta{height}
%    \begin{macrocode}
\def\ii@eqe@getiiiOpts#1[#2]{%
    \@ifnextchar[%]
    {\iii@eqe@getiiiOpts{#1}{#2}}%
    {\iii@eqe@getiiiOpts{#1}{#2}[#1]}}
%    \end{macrocode}
% Get \meta{inner-pos}
%    \begin{macrocode}
\def\iii@eqe@getiiiOpts#1#2[#3]{%
    \def\eqe@opts@argi{#1}%
    \def\eqe@opts@argii{#2}%
    \def\eqe@opts@argiii{#3}}
%    \end{macrocode}
%    \paragraph*{Redefine the \cs{fillin} command}
%    \begin{macro}{\fillin}
%    Re-worked \cs{fillin} to have \textsf{xkeyval} in the optional first parameter.
%    The syntax is illustrated below.
%\begin{verbatim}
% \fillin[
%       underline=true|false,u,b,boxed=true|false,boxpretext=<text>,
%       align=l|r|c,boxsize=\tiny|..\normalsize|\large|...|\Huge,
%       color=<namedcolor>,format=<\bfseries|\ttfamily|\Large|whatever>
% ]{<width>}{<ans>}
%\end{verbatim}
%    \begin{macrocode}
\renewcommand{\fillin}[3][]{\begingroup%
%    \end{macrocode}
%    \cs{ifsp@expand} is defined in \textsf{spdef} package. This is a version
%    if \cs{ifsp} that expands correctly in an \cs{edef}.
%    \begin{macrocode}
    \expandafter\let\expandafter\ifsp\csname ifsp@expand\endcsname
%    \end{macrocode}
%    Get the keys indicated by the document author.
%    \begin{macrocode}
    \setkeys{eqFillin}{boxsize,underline=false,boxed=false,%
    boxpretext,color,format,enclosesoln=false,fitwidth=false}%
    \protected@edef\eq@temp@exp{\noexpand\setkeys{eqFillin}{#1}}%
    \eq@temp@exp
%    \end{macrocode}
% Get the second parameter.
%    \begin{macrocode}
    \edef\eqe@argii{#2}%
%    \end{macrocode}
% We reset \cs{fboxrule} and \cs{fboxsep} as needed.
%    \begin{macrocode}
    \eq@fillinhiddenbox
%    \end{macrocode}
%    If the document author uses the \texttt{hiddenbox} option,
%    this option assumes the \texttt{boxed} option as well so we'll
%    set \cs{KV@eqFillin@boxedtrue} to signal the \texttt{boxed} option.
%    \begin{macrocode}
    \ifx\eq@fillinhiddenbox\@empty\else
        \KV@eqFillin@boxedtrue\fi
%    \end{macrocode}
%    If the \texttt{parbox} option is taken, we define the third parameter
%    to be wrapped in a \cs{parbox}.
%    \begin{macrocode}
    \ifx\eq@fillinparbox\@empty\def\eqe@argiii{#3}\else
%    \end{macrocode}
%    If \texttt{parbox} is specified, we make \texttt{align=l} the default.
%    \begin{macrocode}
        \ifx\eqe@align@hfill\relax
        \def\eq@eqFillin@align{l}\def\eqe@align@hfill{}\fi
%    \end{macrocode}
%    If \texttt{parbox} is specified, we get its three optional
%    parameters so we can manipulate the width parameter.
%    \begin{macrocode}
        \expandafter\eqe@getiiiOpts\eq@fillinparbox\relax
%    \end{macrocode}
%    Now, if this is to be \texttt{boxed}, we reduce the height
%    of the box (\cs{boxed} increases the height by
%    \texttt{2\string\fboxrule+2\string\fboxrule}
%    \begin{macrocode}
        \ifKV@eqFillin@boxed
%    \end{macrocode}
%    \cs{eqe@opts@argii} has a value of \cs{relax} if the document author
%    did not specify a height for the box.
%    \begin{macrocode}
            \expandafter\ifx\eqe@opts@argii\relax\else
            \edef\eqe@opts@argii{\expandafter
                \eqe@fbox@corr\expandafter{\eqe@opts@argii}}\fi
        \fi
%    \end{macrocode}
%    We need to feed \cs{parbox} the parameters it expects, so, if the
%    height parameter is not given, we just pass the first argument;
%    otherwise, we pass all three parameters.
%    \begin{macrocode}
        \edef\eqe@parboxOptArgs{[\eqe@opts@argi]%
            \expandafter\ifx\eqe@opts@argii\relax\else
            [\eqe@opts@argii][\eqe@opts@argiii]\fi}%
%    \end{macrocode}
%    Now we build the third parameter, \cs{eqe@argiii}.
%    \begin{macrocode}
        \def\eqe@argiii{\expandafter\parbox\eqe@parboxOptArgs{\eqe@bw}%
%    \end{macrocode}
%    We insert \cs{eqe@align@hfill}, which is synchronized to the value of
%    the \texttt{align} key to move the \cs{parbox} contents to left aligned, centered,
%    or right aligned. \cs{eqe@align@hfill} will only be effective if \texttt{\#3} is
%    enclosed in a narrower box.
%    \begin{macrocode}
            {\eqe@align@hfill\ifKV@eqFillin@boxed\eq@fillintext\fi#3}}%
    \fi
%    \end{macrocode}
%    If \cs{eqe@align@hfill} is still equal to \cs{relax}, give it the default
%    value.
%    \begin{macrocode}
    \ifx\eqe@align@hfill\relax
        \def\eq@eqFillin@align{c}%
        \edef\eqe@align@hfill{\eqe@align@hfill@default}\fi
    \ifmmode\let\@eqmath\ensuremath\else\let\@eqmath\text\fi
%    \end{macrocode}
%    We re-calculate the width of the formatted box
%    \begin{macrocode}
    \ifx\eq@fillinparbox\@empty
        \ifx\eqe@argii\@empty
%    \end{macrocode}
%    If no \texttt{parbox} option and if the second argument is empty,
%    we set width based on the natural width of \texttt{\#3}
%    \begin{macrocode}
            \settowidth{\eqetmplengthb}{\@eqmath{\eq@fillin@format
            \ifKV@eqFillin@boxed\eq@fillintext\fi\eqe@argiii}}%
            \ifKV@eqFillin@boxed
                \setlength{\eqetmplengthb}{%
                \eqetmplengthb+2\fboxsep+2\fboxrule}%
            \fi
        \else
%    \end{macrocode}
%    If \texttt{\#2} is nonempty, we use this value.
%    \begin{macrocode}
            \setlength{\eqetmplengthb}{#2}%
        \fi
    \else
%    \end{macrocode}
%    \texttt{parbox} option with empty second argument, use \cs{linewidth}.
%    for width
%    \begin{macrocode}
        \ifx\eqe@argii\@empty
            \setlength{\eqetmplengthb}{\linewidth}%
            \PackageWarning{eqexam}{Parameter \#2
                is empty with parbox option,\MessageBreak
                using \string\linewidth\space for width%
            }%
        \else
%    \end{macrocode}
%    \texttt{parbox} option with second argument, use \texttt{\#2}
%    for width
%    \begin{macrocode}
            \setlength{\eqetmplengthb}{#2}%
        \fi
    \fi
%    \end{macrocode}
%    Return \cs{ifsp} to its default definition.
%    \begin{macrocode}
    \expandafter\let\expandafter\ifsp\csname ifsp@default\endcsname
%    \end{macrocode}
%    Save the final calculated width as \cs{eqe@bw}.
%    \begin{macrocode}
    \edef\eqe@bw{\the\eqetmplengthb}%
%    \end{macrocode}
%    Set the underline option, \dots
%    \begin{macrocode}
    \ifKV@eqFillin@underline\let\@fillinFmt\underbar
    \else\let\@fillinFmt\relax\fi
%    \end{macrocode}
%    however, if \texttt{parbox} is specified, we remove the underlining,
% if any.
%    \begin{macrocode}
    \ifx\eq@fillinparbox\@empty\else
        \ifx\@fillinFmt\underbar\let\@fillinFmt\relax
        \PackageInfo{eqexam}{Removing underline option, not permissible
        \MessageBreak with parbox option}%
    \fi\fi
%    \end{macrocode}
%    \paragraph*{Build the \cs{fillin} box.} After the preliminaries, we
%    create the requested answer field. We begin by building the answer
%    field for the case of \cs{ifeq@proofing} is true (which occurs when
%    the \texttt{answerkey} is used.
%    \begin{macrocode}
    \ifeq@proofing
        \ifKV@eqFillin@fitwidth
%    \end{macrocode}
%    If the \texttt{fitwidth} option is taken, we measure the width
%    of the box. Ignored when the \texttt{parbox} option is used.
%    \begin{macrocode}
            \settowidth{\eqetmplengthb}{\@eqmath{\eq@fillin@format
                \ifx\eq@fillinparbox\@empty\ifKV@eqFillin@boxed
                    \eq@fillintext\fi\fi\eqe@argiii}}%
%    \end{macrocode}
%    If boxed, we increase the width by
%    \texttt{2\string\fboxsep+2\string\fboxrule}; when content is \cs{boxed},
%    the dimensions are reduced.
%    \begin{macrocode}
            \ifKV@eqFillin@boxed
                \setlength{\eqetmplengthb}{%
                \eqetmplengthb+2\fboxsep+2\fboxrule}%
            \fi
            \edef\eqe@bw{\the\eqetmplengthb}%
        \fi
%    \end{macrocode}
%    We build the fill-in field for the case of \texttt{boxed}.
%    \begin{macrocode}
        \ifKV@eqFillin@boxed
            \ifmmode\let\@eqmath\ensuremath\else\let\@eqmath\text\fi
%            \mbox{\eq@fillin@format\ensuremath{\boxed{%
            \setbox0=\hbox{\eq@fillin@format\ensuremath{\boxed{%
            \eq@eqFillin@boxsize
            \@fillinFmt{%
                \ifKV@eqFillin@boxed
                    \edef\eqe@bw{\eqe@fbox@corr{\eqe@bw}}%
                \fi
%    \end{macrocode}
%    \changes{v2.0h}{2011/04/14}{Modified the calculation of the width
%    of \cs{fillin}, the width of enclosing box now equals the requested
%    width}
%    When the boxed option is taken, we adjust the width of the \cs{makebox}
%    to get the desired width \texttt{\#2}.
%    \begin{macrocode}
                \makebox[\eqe@bw][\eq@eqFillin@align]{\strut
                \eq@fillin@format\color{\eq@fillin@color}%
                \@eqmath{\ifx\eq@fillinparbox\@empty
                    \eq@fillintext\fi\eqe@argiii}}%
                }% end \@fillinFmt
            }}}% end \mbox
            \setlength{\@tempdima}{\ht0+\dp0}%
            \xdef\fillinTotalHeight{\the\@tempdima}%
            \mbox{\unhbox0}%
        \else
%    \end{macrocode}
%    The content is not to be boxed.
%    \begin{macrocode}
            \ifmmode\let\@eqmath\ensuremath\else\let\@eqmath\relax\fi
            \setbox0=\hbox{%
            \@fillinFmt{\makebox[\eqe@bw][\eq@eqFillin@align]{\strut
                \eq@fillin@format\color{\eq@fillin@color}%
                \@eqmath{\eqe@argiii}}}%
            }\setlength{\@tempdima}{\ht0+\dp0}%
            \xdef\fillinTotalHeight{\the\@tempdima}%
            \mbox{\unhbox0}%
        \fi
    \else
%    \end{macrocode}
%    We begin the case of not \cs{ifeq@proofing}, that is, the document author
%    is not compiling with the \texttt{answerkey} option.
%    \begin{macrocode}
        \ifKV@eqFillin@boxed
%            \mbox{\eq@fillin@format\ensuremath{\boxed{%
            \setbox0=\hbox{\eq@fillin@format\ensuremath{\boxed{%
            \eq@eqFillin@boxsize\eq@fillintext
            \@fillinFmt{%
%    \end{macrocode}
%    We do a similar thing if proofing is not active (\texttt{nosolutions} is taken).
%    \begin{macrocode}
                \ifx\eq@fillintext\@empty
                    \makebox[\eqe@fbox@corr{\eqe@bw}]{%
                        \ifKV@eqFillin@enclosesoln
                        \phantom{\eqe@argiii}\else
                        \strut\hfill\fi
                    }\else
                    \settowidth{\eqetmplengthb}{%
                        \ensuremath{\eq@fillintext}}%
                    \makebox[\eqe@fbox@corr{\eqe@bw}-\eqetmplengthb]{%
                        \ifKV@eqFillin@enclosesoln
                        \phantom{\eqe@argiii}\else
                        \strut\hfill\fi
                    }\fi
                }%end \@fillinFmt
            }}}%
            \setlength{\@tempdima}{\ht0+\dp0}%
            \xdef\fillinTotalHeight{\the\@tempdima}%
            \mbox{\unhbox0}%
        \else
%    \end{macrocode}
%    This is the case where the field is \emph{not} boxed.
%    \begin{macrocode}
            \setbox0=\hbox{%
            \@fillinFmt{\makebox[\eqe@bw]{\strut\hfil}}%
            }\setlength{\@tempdima}{\ht0+\dp0}%
            \xdef\fillinTotalHeight{\the\@tempdima}%
            \mbox{\unhbox0}%
        \fi
%    \end{macrocode}
%    \paragraph*{Online Code.} If the \texttt{quiz} environment is defined,
%    and the user has asked for \texttt{online} option we build a text field.
%    \begin{macrocode}
        \@ifundefined{@quiz}{}{%
            \ifx\eq@online y\relax
                \ifeq@nosolutions
                    \ifeq@solutionsafter\else
                        \ifx\eq@insertverticalspace y\relax
%    \end{macrocode}
%    OK, we get this far if we choose \texttt{online} (or higher) and
%    if \texttt{nosolutions} (which includes the \texttt{vspacewithsolns}
%    option). We require \cs{eq@insertverticalspace} to be \texttt{y}.
%    This last value is the default (\cs{SpaceToWork}).
%    \begin{macrocode}
                            \stepcounter{@cntfillin}%
                            \edef\fieldName{%
                                \if\probstar*eqexam.\curr@quiz.fillin.%
                                    \theeqquestionnoi.part\thepartno.%
                                    fi\the@cntfillin
                                \else
                                    eqexam.\curr@quiz.fillin.%
                                    \theeqquestionnoi.fi\the@cntfillin
                                \fi
                            }\ifx\eq@fillinparbox\@empty
%    \end{macrocode}
%    If the \texttt{parbox} option is not taken, we build a text field with height \texttt{11bp}
%    \begin{macrocode}
                            \makebox[0pt][r]{\textField[\BC{}]{%
                               \fieldName}{#2}{\fillinTotalHeight}}\else
%    \end{macrocode}
%    If the user has taken the \texttt{parbox} option, then the text field becomes a multiline
%    field, with height equal to the requested height.
%    \begin{macrocode}
                            \setlength{\@tempdima}%
                                {\eqe@opts@argii+2\fboxrule+2\fboxsep}%
                            \makebox[0pt][r]{\textField[\BC{}
                                \Ff{\FfMultiline}]{%
                                \fieldName}{#2}{\fillinTotalHeight}}\fi
                        \fi
                    \fi
                \fi
            \fi
        }%
    \fi\endgroup\space\ignorespaces}
%    \end{macrocode}
%    \end{macro}
%    \begin{macro}{\TF}
% The \cs{TF} command depends on \cs{fillin}, so we make the appropriate changes.
%    \begin{macrocode}
\renewcommand\TF[2][\defaultTFwidth]{%
    \def\eqe@next{\fillin[underline]{#1}{#2}}%
    \ifdim\eq@extralabelsep=0pt\relax\else
        \if\probstar*\relax\if\exerwparts@cols x
            \def\eqe@next{\makebox[0pt][r]{%
                \fillin[underline]{#1}{#2}}\ignorespaces}%
    \fi\fi\fi\eqe@next
}
%    \end{macrocode}
%    \end{macro}
% This marks the end of the \textsf{eqexam} package. dps
%    \begin{macrocode}
%</package>
%    \end{macrocode}
\endinput
